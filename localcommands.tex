\let\hspaceThis\hphantom

\captionsetup[sub]{labelformat=simple,labelsep=period}

\renewcommand{\lsCoverTitleFontSize}{48pt}
\renewcommand{\lsCoverTitleFontBaselineskip}{15mm}

\DefineBibliographyExtras{french}{\restorecommand\mkbibnamefamily}
\DefineBibliographyStrings{french}{ phdthesis = {Thèse de doctorat} }


\newcommand{\appref}[1]{Appendix \ref{#1}}
\newcommand{\fnref}[1]{Footnote \ref{#1}}
\renewcommand{\sectref}[1]{section~\ref{#1}}

\newcommand*{\Boite}{\bgroup\normalfont\ttfamily\char"25A2\egroup}
\setkomafont{descriptionlabel}{\normalfont}

\tcbuselibrary{theorems}

\newcommand{\chapfuturef}[1]{chapitre~#1 du vol.~2}
\renewcommand{\chapref}[1]{chapitre~\ref{#1}}
\newcommand{\hi}[1]{\textit{#1}}

\RedeclareSectionCommand[font={\sffamily},beforeskip=\baselineskip]{paragraph}

\newcommand*{\FurtherReading}[1]
  {%
    \vskip\baselineskip
    \begin{refsection}
    \nocite{*}
    \printbibliography[heading=none,keyword={Ch-#1}]
    \end{refsection}
  }

\newtcbtheorem[number within=chapter]
  {definition}
  {Définition}
  { 
    graphical environment = tikzpicture,
    boxsep = 0pt,
    fonttitle = \normalsize\sffamily\bfseries,
    toptitle = 5mm,
    top = 5mm,
    bottom = 5mm,
    left = 5mm,
    right = 5mm,
    borderline = {0.8mm}{0pt}{\lsSeriesColor},
    sharp corners = all,
    before upper = {\hspace*{-\parindent}},
    beforeafter skip balanced = \baselineskip
  }
  {def}

\DeclareTColorBox
  [auto counter,
   number within=chapter,
   list inside=encadre
  ]
  {encadre}{O{} m m}
  {
    label = #1,
    skin = enhanced,
    graphical environment = tikzpicture,
    title = {Encadré~\thetcbcounter: #2},
    list entry = {\protect\numberline{\thetcbcounter} #2},
    boxsep = 0pt,
    toptitle = 10mm,
    top = 5mm,
    bottom = 5mm,
    left = 5mm,
    right = 5mm,
    frame style = {black!12},
    interior hidden,
    sharp corners = all,
    enlarge top initially by = 5mm,
    underlay unbroken and first = 
      {
        \node 
          [
            anchor = base,
            xshift = 11.3mm,
            yshift = -5mm,
            fill = white,
            draw = black!12,
            inner sep = 1mm,
            line width = 0.8mm
          ] at (frame.north west) 
          {{\color{black!12}\includegraphics[height=10mm]{tbls-#3.pdf}}};
      },
    subtitle style = 
      {
        top = 3.25ex,
%         before skip = \baselineskip,
        boxrule = 0pt,
        colback = black!12,
        fontupper = \normalfont\sffamily\bfseries\large
      } 
  }


\newcommand{\Definition}[2]{%
  \begin{definition}{#1}{}
  #2
  \end{definition}%
}

\newfontfamily\xitsfont
	  [
	    Scale=MatchLowercase,
      ]
      {XITSMath-Regular.otf}
\AdditionalFontImprint{XITS Math}

\setlength{\epigraphwidth}{.618\textwidth}% (Golden ratio)
\tikzset
  {
    reset shape/.style = {
      shape=rectangle,
      draw=none,
      fill=none
    },
    AbsSet/.style = {
      draw,
      circle,
      inner sep=9pt
    },
    AbsEllipse/.style ={
      draw,
      ellipse,
      inner xsep=9pt
    },
    ConcSet/.style = {
      draw,
      ellipse,
      font = {\itshape\strut}
    },
    EdgeLabel/.style = {
      midway,
      above,
%       sloped,
      font=\small
    },
    every loop/.style={
        -{Triangle[]},
        out=120,
        in=60
    },
    FnEdgeLabel/.style={
      reset shape,
      font=\footnotesize
    },
    CircleNode/.style={
      circle, 
      draw,
      inner sep=2.5pt,
    },
    CircleFillNode/.style={
      circle, 
      fill=black,
      inner sep=2.5pt,
    },
    ConversionArrow/.style={
      double, 
      -{Triangle[open]}
    },
	DiamondNode/.style={
		diamond,
		draw,
		inner sep=2pt,
	},
  }% exceptions for the every node/.style key

\forestset
  {% Adapted from the decision tree example in the forest doc.
    anchors/.style={anchor=#1,child anchor=#1,parent anchor=#1},
    dot/.style={tikz+={\fill (.child anchor) circle[radius=#1];}},
    dot/.default=3pt,
    dot tree/.style = {
      anchors=south, outer sep=2pt,
      dot=3pt,for descendants=dot,
      for tree = {
        if n children=0{anchors=north}{
        if n=1{anchors=south east}{anchors=south west}},
      }
    }
  }  

\newenvironment{langscibars}{\begin{axis}[ybar,xtick=data, xticklabels from table={\mydata}{pos},
        width  = \textwidth,
	height = .3\textheight,
    	nodes near coords,
	xtick=data,
	x tick label style={},
	ymin=0,
        ]}{\end{axis}}

\newcommand{\langscibar}[1]{\addplot+ table [x=i, y=#1] {\mydata};\addlegendentry{#1};}

\newcommand{\langscidata}[1]{\pgfplotstableread{#1}\mydata;}


\newcommand{\textstylePhono}[1]{#1}
\newcommand{\textstylePhonoApprofondissement}[1]{#1}
\newcommand{\textstylest}[1]{#1}
\newcommand{\textstyleTermes}[1]{\is{#1}\textsc{#1}}
\newcommand{\textstyleTermesapprof}[1]{\textstyleTermes{#1}}
\newcommand{\textstyleTermesapprofondissement}[1]{\textstyleTermes{#1}}
\newcommand{\terme}[1]{\textstyleTermes{#1}}
\newcommand\phraseme[1]{$\ulcorner$\textsc{#1}$\urcorner$}


%%%%%%%%%%%%%%%%%%%%%
\setlength{\marginparwidth}{24pt}
\KOMAoptions{headings=optiontotocandhead}
\newcommand{\gkchapter}[2]{#1: #2}
\newcommand{\encadref}[1]{encadré~\ref{#1}}

\NewDocumentCommand{\loupe}{O{} m +m}
  {\begin{encadre}[#1]{#2}{glass}#3\end{encadre}}
\NewDocumentCommand{\eiffel}{O{} m +m}
  {\begin{encadre}[#1]{#2}{french}#3\end{encadre}}
\NewDocumentCommand{\globe}{O{} m +m}
  {\begin{encadre}[#1]{#2}{world}#3\end{encadre}}
\NewDocumentCommand{\chevalier}{O{} m +m}
  {\begin{encadre}[#1]{#2}{history}#3\end{encadre}}
\NewDocumentCommand{\maths}{O{} m +m}
  {\begin{encadre}[#1]{#2}{calc}#3\end{encadre}}

\newcommand{\corrections}[1]{\begin{tblsfilledsymbol}{Corrections des exercices}{checkmark}#1\end{tblsfilledsymbol}}
\newcommand{\exercices}[1]{\begin{tblsfilledsymbol}{Exercices}{pencil}#1\end{tblsfilledsymbol}}
\newcommand{\exercice}[1]{\vspace{\baselineskip}\noindent{\normalfont\sffamily\bfseries Exercice #1.}\hspace{.75em}}
\newcommand{\corrigé}[1]{\vspace{\baselineskip}\noindent{\normalfont\sffamily\bfseries Corrigé #1.}\hspace{.75em}}
\newcommand{\citations}[1]{\begin{tblsfilledsymbol}{Citations originales}{quote}#1\end{tblsfilledsymbol}}
\newcommand{\lecturesadditionnelles}[1]{\begin{tblsfilledsymbol}{Lectures additionnelles}{book}#1\end{tblsfilledsymbol}}


\newcommand{\HappySmiley}{\bgroup\fontspec[Scale=MatchUppercase]{DejaVuSansMono.ttf}\char"263A\egroup}
\newcommand{\SadSmiley}{\bgroup\fontspec[Scale=MatchUppercase]{DejaVuSansMono.ttf}\char"2639\egroup}
\newcommand{\CryingSmiley}{\bgroup\fontspec[Scale=MatchUppercase]{DejaVuSans.ttf}\char"1F62D\egroup}

\newcommand{\gkcaption}[1]{
	\draw ($(current bounding box.south west)+(0cm,-0.3cm)$) -- (current bounding box.south east);
	\node (caption) [anchor=south] at ($(current bounding box.south)+(0,-0.65cm)$) {#1};
}


\renewcommand{\lsDedicationFont}{\fontsize{13pt}{10mm}\selectfont}
