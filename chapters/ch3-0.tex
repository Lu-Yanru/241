\section*{Présentation}

Cette troisième partie est consacrée à la façon dont les signes linguistiques se combinent pour former des syntagmes et aux structures qui en découlent. Le \chapref{sec:3.1}, où nous donnons notre définition de la syntaxe, est consacré à distinguer l’unité minimale de la syntaxe, le syntaxème, du syntagme. Le \chapref{sec:3.2} introduit la structure de connexion en montrant comment la combinaison des unités syntaxiques définit un ensemble de connexions qui forment la charpente de la structure syntaxique. Le \chapref{sec:3.3} montre que cette structure est hiérarchique en introduisant les notions de tête et de dépendance. Le \chapref{sec:3.4} étudie les équivalences et différences entre les structure de dépendance, qui mettent en avant les relations syntagmatiques entre unités, et les arbres de constituants, qui mettent en avant les relations d’enchâssement entre les unités. Le \chapref{sec:3.5} étudie l’ordre des mots et la façon dont les syntaxèmes se regroupent lorsqu’ils sont dans l’ordre linéaire, définissant ainsi ce que nous appelons la structure topologique. Le \chapref{sec:13} présente la structure qui rend compte de la combinaison des sémantèmes et étudie les distorsions entre cette structure et les structures qui rendent compte de la combinaison des syntaxèmes. 
