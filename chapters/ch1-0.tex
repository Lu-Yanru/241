\section*{Présentation}

Cette première partie est consacrée à la modélisation des langues en général. Elle est divisée en trois chapitres. Le \chapref{1.1} essaye de définir la langue, notre objet d’étude, et précise les caractéristiques que nous retenons dans notre modélisation. Le \chapref{1.2} montre à travers l’étude de la production d’un énoncé où se situent la syntaxe et la grammaire dans un modèle complet de la langue. Le \chapref{1.3} caractérise le type de modèles que nous adoptons dans cet ouvrage.
