\section*{Présentation}

Cette première partie est consacrée à la modélisation des langues en général. Elle est divisée en trois chapitres. Le premier chapitre essaye de définir la langue, notre objet d’étude, et précise les caractéristiques que nous retenons dans notre modélisation. Le deuxième chapitre montre à travers l’étude de la production d’un énoncé où se situent la syntaxe et la grammaire dans un modèle complet de la langue. Le troisième chapitre caractérise le type de modèles que nous adoptons dans cet ouvrage.
