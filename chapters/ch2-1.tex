\chapter{\gkchapter{Trois types d’unités}{Morphème, syntaxème, sémantème}}\label{sec:2.1}

\section{Introduction}\label{sec:2.1.0}

\begin{quote}
    «~De même que le jeu d’échecs est tout entier dans la combinaison des différentes pièces, de même la langue a le caractère d’un système basé complètement sur l’opposition de ses unités concrètes. On ne peut ni se dispenser de les connaître, ni faire un pas sans recourir à elles ; et pourtant leur délimitation est un problème si délicat qu’on se demande si elles sont réellement données.~» (\citealt{saussure1916cours} : 149)
\end{quote}

Notre objectif est d’étudier la structure de la langue. Nous verrons dans la Partie 3 que, contrairement à ce qu’affirme Saussure dans la citation qui précède, on peut en grande partie \hi{s’abstraire} \hi{de la question des unités} lorsqu’on définit la structure syntaxique. Ce qui nous intéresse, ce sont les combinaisons entre les unités et non les unités elles-mêmes. Néanmoins, pour pouvoir parler des combinaisons d’unités, il nous faut dire avant un mot des unités. Nous profiterons de cette discussion sur les unités pour montrer que les unités de la langue, que l’on appelle des \textstyleTermes{signes linguistiques}, peuvent être appréhendées selon \hi{trois points de vue} (morphologique, syntaxique et sémantique) et groupées en ensembles de signes différents selon chacun de ces points de vue. Nous nous attarderons en particulier sur la question des \hi{unités minimales}. Cette question n’est pas fondamentale pour étudier la structure, mais elle n’est pas inutile non plus et elle a l’avantage de pointer les différences entre morphologie, syntaxe et sémantique.

\section{Double articulation du langage}\label{sec:2.1.1}

\begin{quote}
    «~L’intention et la capacité de signification […] sont constitutives du son articulé ; et on ne peut rien proposer d’autre pour le distinguer d’une part du cri animal et d’autre part du son musical.~» (\citealt{humboldt1836uber} : 60)
\end{quote}

Considérons la situation suivante : un acteur vient de faire une performance et vous souhaitez le féliciter, c’est-à-dire lui communiquer le plaisir que vous a procuré sa performance. Vous avez plusieurs moyens à votre disposition et notamment applaudir ou crier «~\textit{Bravo} !~». Chacune de ces deux réalisations possède une \textstyleTermes{signification} du type ‘je te félicite’, exprimant que celui qui en fait la réalisation souhaite féliciter celui à qui il l’adresse. Chacune des deux réalisations possède une \textstyleTermes{forme} spécifique : l’applaudis\-sement se réalise en frappant les deux mains à plat l’une sur l’autre et «~\textit{Bravo} !~» se réalise en produisant une séquence sonore particulière. Cette association entre une forme et une signification est appelée un \textstyleTermes{signe}. Ces deux actes de communication sont des signes \textstyleTermes{conventionnels} : la forme de ce signe obéit à une convention que s’est fixée un groupe particulier de personnes et qui leur sert à communiquer. De plus, la relation entre leur signification et leur forme est \textstyleTermes{arbitraire} : on pourrait tout aussi bien féliciter quelqu’un en levant les bras et en agitant les mains comme le font les sourds en langue des signes, ou bien prononcer une autre séquence sonore, comme le font par exemple les Chinois qui crient «~\textit{Hao} !~» pour communiquer leur appréciation. Inversement l’applaudissement et «~\textit{Bravo} !~» pourraient tout aussi bien signifier autre chose que ‘je te félicite’.

Au delà de leurs points communs, ces deux signes présentent des différences importantes qui tiennent à la \textstyleTermes{double articulation du langage} (voir \encadref{sec:2.1.2} sur l’origine du terme). La \textstyleTermes{première articulation} s’illustre par le fait que le signe \textit{bravo} peut être combiné à d’autres signes pour former des énoncés plus complexes : «~\textit{Bravo pour ton excellente prestation} !~», «~\textit{Alors là, excuse-moi, mais je ne te dis pas bravo pour ce que tu viens de faire.}~». En prenant la question à l’envers, on peut remarquer que la plupart des énoncés de la langue peuvent être décomposés en signes plus élémentaires et que la signification de ces énoncés est la combinaison des significations des signes élémentaires qui les composent. Rien de tel avec l’applaudissement qui ne peut pas être combiné avec un autre signe du même type pour exprimer une signification nouvelle.

La \textstyleTermes{deuxième articulation} tient au fait que la substance sonore des signes d’une langue peut être décomposée en la combinaison d’un tout petit nombre de sons élémentaires. Dans le signifiant de \textit{bravo}, tout locuteur du français reconnaît cinq sons et chacun de ces cinq sons peut être permuté avec un autre son du français pour donner des mots (potentiels) différents du français~\textit{:} \textbf{\textit{t}}\textit{ravo, b}\textbf{\textit{l}}\textit{avo, br}\textbf{\textit{i}}\textit{vo, bra}\textbf{\textit{c}}\textit{o, brav}\textbf{\textit{a}}. Dans toutes les langues du monde, il existe un ensemble fini de quelques dizaines de sons élémentaires qui permettent de construire les signifiants de tous les énoncés de cette langue.

Les unités de première articulation et leurs combinaisons sont les \textstyleTermes{signes linguistiques}. Les unités de deuxième articulation, les \hi{segments sonores minimaux}, sont les \textstyleTermes{phonèmes}. L’étude des phonèmes en tant que telle est en dehors du champ de cet ouvrage, bien que le \textstyleTermes{système phonologique} d’une langue présente des similitudes structurelles avec le système des signes de la langue et que les outils d’investigation des deux systèmes soient en partie similaires.

\chevalier[sec:2.1.2]{Découverte de la double articulation}{%
    L’histoire de la mise en évidence de la double articulation du langage est celle de l’\hi{invention de l’écriture}. Les étapes de l’invention de l’écriture peuvent être tracées ainsi : d’abord les hommes ont créé des \textstyleTermes{pictogrammes} isolés symbolisant par exemple la fécondité ou la chasse, puis des \textstyleTermes{idéogrammes} associés aux signes linguistiques et combinés pour faire des textes (comme les hiéroglyphes de l’égyptien ancien ou les sinogrammes), puis une \textstyleTermes{écriture syllabaire} avec un symbole par syllabe, et enfin un \textstyleTermes{écriture alphabétique} avec optimalement une lettre par son (comme en phénicien, puis en grec ancien). On peut dire avec certitude que les savants qui ont élaboré les premières écritures alphabétiques il y a plus de 6000 ans avaient compris la double articulation du langage.

    La double articulation du langage a été introduite par Ferdinand \citet[26]{saussure1916cours} :

    \begin{quote}
    «~En latin \textit{articulus} signifie «~membre, partie, subdivision dans une suite de choses~» ; en matière de langage, l’articulation peut désigner ou bien la subdivision de la chaîne parlée en syllabes, ou bien la subdivision de la chaîne des significations en unités significatives.~»
    \end{quote}

    On doit le terme de \textstyleTermes{double articulation} à André \citet[13--14]{martinet1960elements}:

    \begin{quote}
    «~La \textbf{\textit{première articulation}} du langage est celle selon laquelle tout fait d’expérience à transmettre, tout besoin qu’on désire faire connaître à autrui s’analyse en une suite d’unités douées chacune d’une forme vocale et d’un sens. Si je souffre de douleurs à la tête, je puis manifester la chose par des cris. Ceux-ci peuvent être involontaires ; dans ce cas ils relèvent de la physiologie. Ils peuvent être plus ou moins voulus et destinés à faire connaître mes souffrances à mon entourage. Mais cela ne suffit pas à en faire une communication linguistique. Chaque cri est inanalysable et correspond à l’ensemble, inanalysé, de la sensation douloureuse. Tout autre est la situation si je prononce la phrase \textit{j’ai mal à la tête}. Ici, il n’est aucune des six unités successives \textit{j’, ai, mal, à, la, tête} qui corresponde à ce que ma douleur a de spécifique. Chacune d’entre elle peut se retrouver dans de tout autres contextes pour communiquer d’autres faits d’expérience : \textit{mal}, par exemple, dans \textit{il fait le mal}, et \textit{tête} dans \textit{il s’est mis à leur tête}.~[…] Chacune de ces unités de première articulation présente, nous l’avons vu, un sens et une forme vocale (ou phonique). Elle ne saurait être analysée en unités successives plus petites douées de sens. L’ensemble \textit{tête} veut dire ‘tête’ et l’on ne peut attribuer à \textit{tê-} ou à \textit{{}-te} des sens distincts dont la somme serait équivalente à ‘tête’. Mais la forme vocale est, elle, analysable en une succession d’unités dont chacune contribue à distinguer \textit{tête}, par exemple, d’autres unités comme \textit{bête, tante} ou \textit{terre}. C’est ce qu’on désignera comme la \textbf{\textit{deuxième articulation}} du langage. Dans le cas de \textit{tête}, ces unités sont au nombre de trois ; nous pouvons les représenter au moyen des lettres t e t, placées par convention entre barres obliques, donc /tet/.~»
    \end{quote}

    Si nous reprenons à notre compte la double articulation telle que présentée par Martinet, nous voudrions ajouter une remarque sur ce qui est désigné ici par « cri ». Il est admis aujourd’hui qu’un cri de douleur comme «~\textit{Aïe !}~» est bien un mot du français. Un allemand ne dira pas «~\textit{Aïe} !~», mais «~\textit{Au!}~», (prononcé a-ou) et un anglais «~\textit{Ouch!~}» (prononcé a-outch). Ces signes linguistiques ont une combinatoire beaucoup plus limitée que la plupart des autres signes de la langue (voir la section sur les \textit{Locutifs et interjections} du \chapfuturef{17}), mais ils n’en sont pas moins analysés de façon comparable et «~\textit{Aïe} !~» n’a pas la même signification que «~\textit{Aïaïaïe} !~» (prononcé [a-ja-jaj]).
}
\section{Signifié, syntactique, signifiant}\label{sec:2.1.3}

\Definition{\textstyleTermes{signes linguistiques}, \textstyleTermes{signifié}, \textstyleTermes{signifiant}}
{Les unités de première articulation, qui sont les portions de texte porteuses de signification, sont appelées les \textstyleTermes{signes linguistiques}. (Rappelons que nous désignons par \textit{texte} aussi bien des textes écrits que des productions orales ou gestuelles.) La \hi{signification} portée par un signe est appelée son \textstyleTermes{signifié}. La \hi{forme} d’un signe linguistique, en général un segment de texte, est appelée son \textstyleTermes{signifiant}.}

Les signes linguistiques se distinguent des signes de la plupart des autres systèmes sémiotiques par le fait qu’ils peuvent \hi{se combiner entre eux pour former de nouveaux signes} qui expriment en général des sens qui ne peuvent être exprimés par aucun signe élémentaire. Les signes linguistiques ont donc un potentiel combinatoire qui fait qu’ils possèdent une \hi{troisième composante}. Cette troisième composante n’est pas réductible aux deux autres. Par exemple, les noms, lorsqu’ils se combinent à un article, déclenchent un accord en genre de cet article : \textbf{\textit{une}} \textit{chaise,} \textbf{\textit{un}} \textit{fauteuil,} \textbf{\textit{un}} \textit{tabouret,} \textbf{\textit{une}} \textit{table}, etc. Rien, ni dans la forme du nom, ni dans son sens, ne permet de prévoir quel sera cet accord — féminin ou masculin. Le locuteur ne peut utiliser correctement un nom en français que s’il a appris cette information en plus du signifiant et du signifié du nom.

\Definition{\textstyleTermes{syntactique}, \textstyleTermes{combinatoire}}
{Le signe linguistique possède donc une troisième composante, appelée son \textstyleTermes{syntactique}, qui contrôle sa \textstyleTermes{combinatoire} et qui n’est réductible ni aux propriétés du signifiant, ni à celles du signifié :

\begin{center} signe = 〈 signifié, syntactique, signifiant 〉\end{center}}

Nous aurons l’occasion de décrire en détail le syntactique des signes. Celui-ci comprend plus d’informations qu’on ne le pense en général. Il comprend d’abord ce qu’on appelle la \textstyleTermes{catégorie syntaxique} du signe. Pour un lexème, il s’agit de la \textstyleTermes{partie du discours} (verbe, nom, adjectif, etc.), mais aussi d’un certain nombre de traits qui contrôlent entre autres la combinatoire flexionnelle (comme le \textstyleTermes{groupe de conjugaison} pour les verbes) ou l’accord (comme le \textstyleTermes{genre} des noms). Le syntactique contient également la \textstyleTermes{valence}, c’est-à-dire la \hi{liste des compléments régis} par le signe avec le \textstyleTermes{régime} qui leur est imposé, c’est-à-dire les contraintes sur la nature du complément, sa place par rapport au gouverneur et les marques qui doivent l’accompagner (cas, préposition, etc.) (voir les sections \ref{sec:3.3.10} sur \textit{Distribution et valence} et \ref{sec:3.3.17} sur \textit{Tête interne et rection}). Enfin, le syntactique comprend la description de la \textstyleTermes{cooccurrence lexicale restreinte}, c’est-à-dire toutes les combinaisons qui sont lexicalement contraintes (par exemple le fait que \textit{amoureux} s’intensifie par \textit{follement}, \textit{blessé} par \textit{grièvement}, \textit{malade} par \textit{gravement} et \textit{improbable} par \textit{hautement}) (voir la \sectref{sec:2.3.10} sur \textit{Collocation et choix lié}).

L’étude des signes linguistiques, en raison de leurs trois composantes, relève de trois disciplines différentes : l’étude des signifiés relève de la \textstyleTermes{sémantique}, l’étude du syntactique est le domaine de la \textstyleTermes{syntaxe} et l’étude des signifiants de la \textstyleTermes{morphologie}.

\loupe[sec:2.1.4]{Le signe linguistique}{%
    La notion moderne de \textbf{signe linguistique} est généralement attribuée à \textcite[98]{saussure1916cours} :

    \begin{quote}
    «~Le signe linguistique unit non une chose et un nom, mais un concept et une image acoustique. Cette dernière n’est pas le son matériel, chose purement physique, mais l’empreinte psychique de ce son, la représentation que nous en donne le témoignage de nos sens ; elle est sensorielle, et s’il nous arrive de l’appeler «~matérielle~», c’est seulement dans ce sens et par opposition à l’autre terme de l’association, le concept, généralement plus abstrait. […] Nous proposons de conserver le mot \textit{signe} pour désigner le total, et de remplacer \textit{concept} et \textit{image acoustique} respectivement par \textit{signifié} et \textit{signifiant~}; ces derniers termes ont l’avantage de marquer l’opposition qui les sépare soit entre eux, soit du total dont ils font partie.~»
    \end{quote}

    Saussure n’indique pas clairement s’il appelle signe linguistique n’importe quel segment de texte possédant un sens, comme nous le faisons dans cet ouvrage, ou si le terme s’applique uniquement aux unités minimales. Le paragraphe suivant sur l’\textbf{arbitraire du signe} concerne clairement les signes minimaux, appelés chez nous les morphèmes :

    \begin{quote}
    «~ Le lien unissant le signifiant au signifié est arbitraire, ou encore, puisque nous entendons par signe le total résultant de l’association d’un signifiant à un signifié, nous pouvons dire plus simplement : \textit{le signe linguistique est arbitraire}. […] Le mot \textit{arbitraire} appelle aussi une remarque. Il ne doit pas donner l’idée que le signifiant dépend du libre choix du sujet parlant […] ; nous voulons dire qu’il est \textit{immotivé}, c’est-à-dire arbitraire par rapport au signifié, avec lequel il n’a aucune attache naturelle dans la réalité.~» (\citealt{saussure1916cours} : 101)
    \end{quote}

    Chez Saussure, le signe linguistique est un élément à deux faces, signifié et signifiant. La troisième composante du signe, son \textbf{syntactique}, n’est pas explicitement considérée. \citet{melcuk1993cours} est probablement un des premiers à insister sur la nécessité de cette composante dans la définition du signe linguistique.
}

\loupe[sec:2.1.5]{Signifié ou exprimende}{%
    La terminologie usuellement utilisée pour désigner la forme et le contenu d’un signe linguistique — \textit{signifiant} et \textit{signifié} — véhicule une certaine conception du signe. Le signe est vu comme quelque chose qui signifie, c’est-à-dire quelque chose dont on saisit le signifiant pour accéder à son signifié. C’est une vision du rôle du signe que conteste vivement Lucien \citet[36]{tesniere1959elements} :

    \begin{quote}
    «~Lorsque nous parlons, notre intention n’est pas de trouver après coup un sens à une suite de phonèmes qui lui préexistent, mais bien de donner une forme sensible aisément transmissible à une pensée qui lui préexiste et en est la seule raison d’être. En d’autres termes, le télégraphe est là pour transmettre les dépêches, non les dépêches pour faire fonctionner le télégraphe.~»
    \end{quote}

    Cela conduit Tesnière à inverser le point de vue et, pour mettre la terminologie en conformité avec la nécessité de partir du sens, à proposer de nommer \textstyleTermes{exprimende} le contenu du signe et \textstyleTermes{exprimé} la forme du signe. On souligne ainsi que le signe sert à \hi{exprimer} une pensée et non pas à donner sens à une forme. Bien que tout à fait sensible aux arguments de Tesnière (voir la \encadref{sec:1.1.12} \textit{Parler et comprendre}), nous continuerons à utiliser la terminologie de Saussure, qui est aujourd’hui universellement adoptée.
}
\section{Sémantème, syntaxème, morphème}\label{sec:2.1.6}

Le fait que le signe ait trois composantes implique qu’il y a trois façons de l’appréhender et autant de façons de définir les signes minimaux : les signes minimaux du point de vue du \hi{sens} seront appelés les \textstyleTermes{sémantèmes}, les signes minimaux du point de vue de la \hi{combinatoire} les \textstyleTermes{syntaxèmes} et les signes minimaux du point de vue de la \hi{forme} les \textstyleTermes{morphèmes}.

Le grand problème de la caractérisation des signes minimaux vient de la \hi{non-correspondance} entre les unités minimales de forme et les unités minimales de sens. Par exemple, dans \textit{Aya dort à poings fermés}, le segment \textit{à poings fermés} est une unité minimale de sens exprimant l’intensification du verbe \textsc{dormir}. Ce segment correspond à un choix unique et indivisible du locuteur et pourtant cette unité est construite en utilisant d’autres unités dont on peut entrapercevoir la contribution, puisqu’il s’agit d’un emploi métaphorique figé de la combinaison libre \textit{à poings fermés} ‘en ayant les poings fermés’. Un autre exemple : le mot \textit{décapsuleur} désigne un type particulier d’objet et est donc un choix unique et indivisible fait pour désigner un tel objet. Ce mot n’en est pas moins l’assemblage de trois unités, \textit{dé}+\textit{capsul}+\textit{eur}, que l’on retrouve dans d’autres mots (\textbf{\textit{capsule}}, \textbf{\textit{décapsul}}\textit{er,} \textbf{\textit{dé}}\textit{tacher,} \textbf{\textit{dé}}\textit{fibrillat}\textbf{\textit{eur}}, etc.). Les signes $⌜$\textsc{à} \textsc{poings} \textsc{fermés}$⌝$ ou \textsc{decapsuleur} sont des \textstyleTermes{sémantèmes}. Les composantes \textit{dé-}, \textit{capsule} et -\textit{eur} de \textit{décapsuleur} sont des \textstyleTermes{morphèmes~}: il s’agit bien de signes, car on peut leur attribuer une signification et déterminer leur contribution au sens de \textit{décapsuleur}, et il s’agit de signes minimaux, car on ne peut les décomposer à nouveau en des formes qui possèdent une signification.

Cette non-correspondance entre unités minimales de sens et de forme vaut aussi pour les unités minimales de combinatoire. Si un sémantème comme \textsc{décapsuleur} se comporte comme un nom simple — par exemple \textsc{bol} ou \textsc{couteau} —, un sémantème comme $⌜$\textsc{àvoir} \textsc{les} \textsc{pieds} \textsc{sur} \textsc{terre}$⌝$ (au sens de ‘avoir le sens des réalités’) ne se comporte pas comme un verbe simple, mais plutôt comme une combinaison libre du verbe \textsc{avoir} avec des compléments — par exemple \textit{avoir les mains sur la table}. Le verbe \textsc{avoir} se conjugue de la même façon dans les deux cas, il se combine de la même façon avec la négation, etc. Ceci a notamment pour conséquence que le sémantème est séparable : \textit{Il n’}\textbf{\textit{a}} \textit{vraiment pas} \textbf{\textit{les pieds sur terre}}. Ainsi le composant \textsc{avoir} du sémantème $⌜$\textsc{àvoir} \textsc{les} \textsc{pieds} \textsc{sur} \textsc{terre}$⌝$ est-il du point de vue de sa combinatoire le même verbe \textsc{avoir} que celui qui s’utilise librement dans \textit{avoir les mains sur la table}. C’est une telle unité que nous appelons un \textstyleTermes{syntaxème}. Les autres composantes du sémantème $⌜$\textsc{àvoir} \textsc{les} \textsc{pieds} \textsc{sur} \textsc{terre}$⌝$ sont aussi des syntaxèmes : \textsc{le,} \textsc{pied,} \textsc{sur,} \textsc{terre}, ainsi que les syntaxèmes flexionnels qui marquent le nombre.

Par contre, le morphème \textit{capsule} de \textsc{décapsuleur} n’est pas un syntaxème : \textit{capsule} dans \textsc{décapsuleur} n’a plus la combinatoire d’un nom. C’est \textit{décapsuleur} comme un tout qui rentre dans des combinaisons et, par exemple, qui impose les accord en genre (\textit{un décapsuleur blanc} vs \textit{une capsule blanche}).

\section{L’identification des unités de la langue}\label{sec:2.1.7}

Dans toute description analytique d’une chose, qu’il s’agisse d’un être vivant ou d’une machine, on cherche à identifier les \hi{parties constituantes}. Pour qu’une partie de la chose soit reconnue comme un constituant, on doit pouvoir identifier sa contribution au tout. On doit pouvoir en \hi{délimiter les contours} (le signifiant), en \hi{déterminer les possibilités de combinaison avec les autres éléments du système} (le syntactique) et en \hi{identifier la contribution au fonctionnement général du système} (le signifié). Il n’est réellement intéressant d’isoler un constituant que si l’on peut l’extraire pour le réutiliser ailleurs. Ceci est particulièrement vrai pour les signes de la langue. Ceux-ci ne deviennent identifiables que s’ils sont \hi{utilisés plusieurs fois dans des contextes différents} et que \hi{dans ces mêmes contextes d’autres} \hi{signes sont utilisées}. On peut alors les extraire pour les recombiner autrement et c’est ce qui fait l’infinie richesse de la langue.

Classifier les différents morceaux en fonction des éléments qui peuvent les remplacer (les \textstyleTermes{rapports paradigmatiques}) et des éléments avec lesquels ils peuvent se combiner (les \textstyleTermes{rapports syntagmatiques}) s’appelle l’\textstyleTermes{analyse} \textstyleTermes{distributionnelle}. La technique qui consiste à remplacer un segment par un autre pour voir s’il s’agit d’un signe s’appelle le \textstyleTermes{principe de commutation}. Un ensemble de segments qui peuvent commuter les uns avec les autres s’appelle un \textstyleTermes{paradigme de commutation}.

L’analyse distributionnelle s’est d’abord intéressée aux unités minimales de forme, en délaissant volontairement le sens (voir l’encadré qui suit) et sans clairement distinguer le syntactique du signifiant. Nous appliquerons pour notre part l’analyse distributionnelle aux trois composantes du signe linguistique. Nous considérons en particulier que \hi{le sens est un observable} dans la mesure où il est possible d’une part d’observer dans quel contexte d’énonciation un énoncé est produit et donc de faire des hypothèses sur les intentions du locuteurs qui le produit. Et d’autre part, de demander à un locuteur quels sont les énoncés qu’il aurait été possible de produire pour obtenir le même résultat, c’est-à-dire les énoncés qui possèdent le \hi{même sens} (voir l’\encadref{sec:1.2.1} sur \textit{Les observables : textes et sens}).

Les deux chapitres suivants seront consacrés aux deux types extrêmes de signes minimaux : les morphèmes et les sémantèmes. Les syntaxèmes seront introduits aussi dans cette partie, mais leur étude approfondie aura lieu dans la Partie 3.

\chevalier[sec:2.1.8]{Structuralisme et distributionnalisme}{%
    L’identification des phonèmes et des signes linguistiques d’une langue ne va pas de soi. Le courant \textstyleTermes{structuraliste}, qui s’est développé à la suite de la publication du \textit{Cours de linguistique générale} de Ferdinand de Saussure, propose de prendre comme seule base d’analyse l’~«~observable~» de la langue, c’est-à-dire les énoncés produits par les locuteurs ou acceptables pour eux, et élabore un certain nombre de méthodes basées sur la manipulation de ces énoncés. Suivant la tradition de la \textstyleTermesapprof{psychologie} \textstyleTermes{behavioriste}, le sens des énoncés, considéré comme non observable par essence, est utilisé aussi peu que possible : la langue, comme d’autres comportements (angl. \textit{behaviors}) humains, est analysée de l’extérieur, sans faire d’hypothèse a priori sur ce qui peut se passer dans le cerveau du locuteur.

    Le stade le plus abouti de cette approche basée uniquement sur l’observation des énoncés a été atteint pas l’école \textstyleTermes{distributionnaliste}, dont les principaux acteurs ont été Leonard Bloomfield, Edward Sapir, Benjamin L. Whorf, Charles F. Hockett et Zelig Harris. Ce courant est né de l’étude des langues amérindiennes, langues essentiellement orales et en voie de disparition pour la plupart, pour lesquelles il n’y avait ni écriture, ni description préalable d’aucune sorte. La simple transcription des énoncés produits par les locuteurs nécessitait d’établir très rapidement une liste de phonèmes et une segmentation en morphèmes et en mots quelque peu raisonnable.

    L’influence du distributionnalisme s’est estompée dans les années 1960 avec l’émergence de la «~nouvelle syntaxe~» de Noam Chomsky. En pointant les limites de l’analyse de corpus pour la compréhension du fonctionnement de la langue, il rétablit l’introspection comme méthodologie acceptable pour la linguistique, faisant ainsi perdre à la syntaxe sa base empirique. Seuls les besoins du traitement automatique de la langue et la numérisation des textes pour le web depuis les années 1990 ont poussé les syntacticiens à se réintéresser à l’adaptation de leurs analyses aux données observables.
}
\exercices{%\label{sec:2.1.9}
    \exercice{1} Lorsqu’un professeur veut faire taire un élève, il a plusieurs possibilités à sa disposition : (1) il peut dire «~\textit{Taisez-vous} !~» (2) ou «~\textit{Chut} !~», (3) ou produire un son particulier «~\textit{Tttt~}» avec les lèvres en avant et un claquement de langue à l’avant du palais, (4) ou encore lever l’index ou (5) ou jeter un regard accusateur. De ces cinq stratégies, lesquelles utilisent des signes ? Lesquels sont des signes linguistiques ? Lesquels ne vérifient pas la double articulation ?

    \exercice{2} Qu’est-ce qui distingue un idéogramme d’une lettre de l’alphabet ?

    \exercice{3} Quelles sont les trois composantes du signe linguistique ?

    \exercice{4} Pourquoi distinguons-nous trois types d’unités minimales ?

    \exercice{5} Quel est le principe de base de l’analyse distributionnelle ?
}
\lecturesadditionnelles{%\label{sec:2.1.10}
    Le livre de lexicologie d’Alain \citet{polguere2003lexicologie} est un bon complément à notre ouvrage de syntaxe. On y trouvera une introduction à la fois simple et riche du signe linguistique et une présentation de la structure du lexique et du sens lexical. L’encyclopédie des sciences du langage d'Oswald Ducrot et Jean-Marie Schaeffer (\citeyear{ducrot1995nouveau}) est un ouvrage que nous recommandons particulièrement, notamment le chapitre consacré au «~Signe~» (p. 253 à 256), qui situe le système linguistique parmi les systèmes sémiotiques en général.

    Nous avons déjà parlé des ouvrages de \citet{saussure1916cours}, \citet{bloomfield1933language} et \citet{humboldt1836uber} au \chapref{sec:1.1}. Concernant les distributionnalistes, on pourra consulter les ouvrages d'Edward \citet{sapir1921language}, Charles F. \citet{hockett1958course} et de Zelig \citet{harris1951methods}. Nous avons déjà parlé d’Edward Sapir au \chapref{sec:1.2} et nous reparlerons dans les \chapref{sec:3.1} et \chapref{sec:3.4} de Charles F. Hockett, dont nous recommandons particulièrement la lecture.

    \FurtherReading{2-1}
}
\citations{%\label{sec:2.1.11}
    \noindent{\sffamily\bfseries Citations de la \sectref{sec:2.1.1}.    }

    \noindent \citet[60]{humboldt1836uber} : 
        \begin{quote}Die Absicht und die Fähigkeit zur Bedeutsamkeit […] macht allein den artikulierten Laut aus, und es lässt sich nichts andres angeben, um seinen Unterschied auf der einen Seite vom tierischen Geschrei, auf der anderen vom musikalischen Ton zu bezeichnen.\end{quote}
}
\corrections{%\label{sec:2.1.12}

    \corrigé{1} Tous sont des signes associant un signifiant vocale ou gestuel à une signification dont l'intention est de faire taire un élève. La question est de savoir où l'on fixe les limites de la langue. Chacun des cinq signes s'éloigne un peu plus du noyau central du français. Les deux premiers, «~\textit{Taisez-vous} !~» et «~\textit{Chut} !~», vérifient la double articulation et se décomposent en phonèmes. Néanmoins \textit{Chut} n’a pas de combinatoire syntaxique (voir la section sur \textit{Locutifs et interjections} du \chapfuturef{17}). Le signe \textit{Tttt} (dont on ne sait pas très bien comment l’orthographier) appartient à un sous-système hors du système phonologique du français. On peut considérer qu'il appartient quand même au français (qui n’existe pas a priori dans d’autres langues) et constitue donc un signe linguistique. Le doigt levé est aussi un signe conventionnel, comme le sont les applaudissements. Même si l'étude des langues vocales comme le français se concentre généralement sur les productions vocales, force est d'admettre que la communication langagière repose aussi sur des gestes et que, dans la mesure où un geste est conventionnel, il est un signe linguistique. Pour finir, le regard accusateur est probablement plus conventionnel qu’on ne le pense a priori : pas sûr que le même regard soit interprété de la même façon dans une autre communauté.

    \corrigé{2} Un idéogramme correspond à un morphème, alors qu’une lettre de l’alphabet est la transcription d’un phonème.

    \corrigé{3} Voir la \sectref{sec:2.1.3} \textit{Signifié, syntactique, signifiant}.

    \corrigé{4} À chacune des trois composantes du signe correspond un découpage particulier. Voir la \sectref{sec:2.1.6} \textit{Sémantème, syntaxème, morphème}.

    \corrigé{5} Le principe de base de l’analyse distributionnelle est le principe de commutation, comme nous allons le voir au chapitre suivant.
}

