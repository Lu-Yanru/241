\section*{Présentation}
\todo[inline]{be sure that the title is segmented like this \part{\gkchapter{Les unités de la langue\\}{Les trois composantes\\ du signe linguistique}}  }

Cette deuxième partie s’intéresse à la délimitation des unités de la langue — les signes linguistiques — selon leur forme, leur combinatoire et leur sens. Le \chapref{sec:2.1} présente rapidement pourquoi nous considérons trois types d’unité. Le \chapref{sec:2.2} présente l’identification des signes linguistiques et plus particulièrement les unités minimales de forme, les morphèmes, et les contrastent avec les unités minimales pour la combinatoire libre, les syntaxèmes. Le \chapref{sec:2.3} présente les unités minimales de sens, les sémantèmes, et les contrastent également avec les syntaxèmes.
