\chapter{\gkchapter{Syntaxèmes et syntagmes}{La délimitation des unités minimales de la syntaxe}}\label{sec:3.1}

\section{Syntaxème, morphème et sémantème}\label{sec:3.1.0}

Nous avons donné une première définition du syntaxème au \chapref{sec:2.3}. L’objectif de ce chapitre est de donner une définition plus précise du syntaxème, de préciser la frontière entre le syntaxème et le syntagme, qui est une combinaison de syntaxèmes, et de donner une définition de ce que nous appelons la syntaxe.

Rappelons que nous appelons \textbf{morphèmes} les signes minimaux du point de vue de la forme (\chapref{sec:2.2}) et \textbf{sémantèmes} les signes minimaux du point de vue du sens (\chapref{sec:2.3}). D’où vient la nécessité de considérer, en plus de ces deux unités, les unités que nous appelons les \textbf{syntaxèmes~}?

Le signe idéal est un \textbf{sémantème simple}, c’est-à-dire une unité minimale de forme et de sens, un sémantème qui est aussi un morphème (\chapref{sec:2.3}). (Nous devrions dire «~un sémantème qui est une acception d’un morphème~», car la quasi-totalité des morphèmes ont plusieurs acceptions.) Comme on l’a vu au \chapref{sec:2.3}, il n’y a pas de correspondance entre unités minimales de sens et unités minimales de forme et un grand nombre de sémantèmes sont complexes, c’est-à-dire sont la combinaison de plusieurs morphèmes. Il convient néanmoins de distinguer parmi les sémantèmes complexes, ceux qui se comportent comme des sémantèmes simples et ceux qui se comportent plutôt comme une combinaison de sémantèmes simples. Ce sont les premiers que nous appelons des syntaxèmes.

Nous avons déjà donné, dans le \chapref{sec:2.1}, l’exemple des sémantèmes \textsc{décapsuleur} et $⌜$\textsc{avoir} \textsc{les} \textsc{pieds} \textsc{sur} \textsc{terre}$⌝$. Les deux sont composés de plusieurs morphèmes, mais, alors que \textsc{décapsuleur} se comporte comme un sémantème simple — comme \textsc{bol} ou \textsc{couteau} —, $⌜$\textsc{avoir} \textsc{les} \textsc{pieds} \textsc{sur} \textsc{terre}$⌝$ ne se comporte pas comme un verbe simple. En fait, $⌜$\textsc{avoir} \textsc{les} \textsc{pieds} \textsc{sur} \textsc{terre}$⌝$ se comporte comme la combinaison libre \textit{avoir les mains sur la table} et est construit de manière \textbf{analogue} à cette expression par la combinaison du verbe \textsc{avoir} et de compléments avec lesquels il s’est figé sémantiquement. À l’inverse, \textsc{décapsuleur} n’est pas construit de manière analogue à une expression libre.

\section{Analogie structurelle}\label{sec:3.1.2}

La notion d’analogie (structurelle) joue un rôle central dans la définition des unités syntaxiques. Voici comment nous la définissons :

\Definition{analogie (structurelle)}
{Une combinaison de signes A+B est dite (\textstyleTermes{structurellement}) \textstyleTermes{analogue} à une combinaison A’+B’ si A a une distribution équivalente à A’, B à B’ et A+B à A’+B’.}

Cette définition va être illustrée dans la \sectref{sec:3.1.4}. Donnons avant cela quelques autres définitions basées sur l’analogie, à commencer par la définition du syntagme :

\Definition{\textstyleTermes{syntagme}}
{Un \textstyleTermes{syntagme} est une combinaison de signes \textbf{structurellement analogue} à une \textbf{combinaison libre}.}

En incluant les combinaisons \textit{analogues} à des combinaisons libres, nous incluons les constructions figées, c’est-à-dire la possibilité qu’un syntagme ne soit pas une combinaison libre, mais se comporte syntaxiquement comme une combinaison libre.

Les syntagmes sont des combinaisons régulières de signes du point de vue de leur \textbf{syntactique} (voir la \sectref{sec:2.1.3} sur \textit{Signifié, syntactique, signifiant}). Autrement dit, A+B forme un syntagme si le syntactique de A+B se calcule de manière régulière à partir des syntactiques de A et de B. Mais la combinaison des signifiants de A et B peut être irrégulière (par exemple être un amalgame comme \textit{au} = \textit{à}+\textit{le} ou \textit{viens} = \textsc{venir}+indicatif+présent+2+singulier) ou la combinaison des signifiés de A et B peut être irrégulière (lorsque A+B est une locution).

La notion de syntaxème peut maintenant être définie de manière plus rigoureuse.

Par définition, tout syntagme X est une combinaison A+B qui est libre ou analogue à une combinaison libre. On peut donc décomposer X en deux signes A et B et une telle \textstyleTermes{décomposition} est dite \textstyleTermes{syntaxique}. Nous opposons une telle décomposition à une \textstyleTermes{décomposition morphologique}, qui serait une décomposition en morphèmes reposant uniquement sur la commutation propre.

La décomposition syntaxique peut être appliquée récursivement. Si les signes A et B sont à nouveau des syntagmes, on pourra les décomposer en deux signes et ainsi de suite, jusqu’à ce qu’on arrive à des signes qui ne peuvent plus être décomposés selon ce principe. C’est ce que nous appelons des \textstyleTermes{syntaxèmes}.

Toute combinaison de deux syntaxèmes est un syntagme et tout syntagme est une combinaison de plusieurs syntaxèmes.

Plus généralement :

\Definition{\textstyleTermes{unité syntaxique}}
{Les \textstyleTermes{unités syntaxiques} sont \textbf{les signes qui commutent librement} dans leur contexte ou qui sont analogues à de tels signes.}

Une unité syntaxique est soit un \textbf{syntagme}, soit un \textbf{syntaxème}. Et donc :

\Definition{\textstyleTermes{syntaxème}}
{Les \textstyleTermes{syntaxèmes} sont donc les \textbf{unités syntaxiques minimales}. Un syntaxème est \textbf{indécomposable en deux unités syntaxiques}.}

\eiffel{Le collimateur et la sellette}{%\label{sec:3.1.3}
    En général, tout syntaxème commute librement dans certains environnements, mais il existe des cas extrêmes comme \textsc{fur}, qui n’apparaît que dans le phrasème $⌜$\textsc{au} \textsc{fur} \textsc{et} \textsc{à} \textsc{mesure}\textrm{$⌝$}. Le fait que \textit{au fur} soit assez clairement analogue à une combinaison libre, en raison notamment de la coordination avec \textit{à mesure}, amène à considérer \textit{fur} comme un syntaxème. Les éléments lexicaux qui ne s’utilisent plus (guère) que dans des phrasèmes sont quand même assez nombreux : (\textit{être}) \textit{aux} \textbf{\textit{aguets}}, (\textit{avancer}) \textit{à la queue} \textbf{\textit{leu leu}}, \textit{à l’}\textbf{\textit{instar}} \textit{de,} (\textit{utiliser qqch}) \textit{à bon/mauvais} \textbf{\textit{escient}}, (\textit{poser une question}) \textit{à brûle-}\textbf{\textit{pourpoint}}, (\textit{rouler}) \textit{à toute} \textbf{\textit{blinde}}, (\textit{aller}) \textit{au diable} \textbf{\textit{vauvert}}, (\textit{avoir qqn}) \textit{dans le} \textbf{\textit{collimateur}}, \textit{de plein} \textbf{\textit{gré}}, (\textit{coeur}) \textit{battre la} \textbf{\textit{chamade}}, \textit{faire de la} \textbf{\textit{charpie}} (\textit{de qqch}), \textit{de} \textbf{\textit{bric}} \textit{et de} \textbf{\textit{broc}}, \textit{de} \textbf{\textit{guingois}}, \textit{de} \textbf{\textit{traviole}}, \textit{en} \textbf{\textit{catimini}}, \textit{en} \textbf{\textit{filigrane}}, \textit{mettre la} \textbf{\textit{sourdine}}, \textit{en un} \textbf{\textit{tournemain}}, \textit{en} \textbf{\textit{vrac}}, \textit{et tout le} \textbf{\textit{bastringue}}, \textit{manger à tous les} \textbf{\textit{râteliers}}, (\textit{mettre qqn}) \textit{sur la} \textbf{\textit{sellette}}, \textit{passer au} \textbf{\textit{crible}}, \textit{prendre la poudre d’}\textbf{\textit{escampette}}, \textit{sans coup} \textbf{\textit{férir}}, \textit{sans} \textbf{\textit{encombre}}, \textit{se faire du} \textbf{\textit{mouron}} (\textit{pour qqn}), \textit{s’en soucier comme d’une} \textbf{\textit{guigne}}, \textit{sonner le} \textbf{\textit{glas}} (\textit{de qqch}), \textit{tailler des} \textbf{\textit{croupières}} (\textit{à qqn}), (\textit{mettre}) \textit{en} \textbf{\textit{exergue}}, \textit{tous} \textbf{\textit{azimuts}}, etc.
}
\section{Syntagme ou syntaxème ?}\label{sec:3.1.4}

Nous allons mettre en pratique nos définitions en étudiant des unités qui sont à la limite entre syntagme et syntaxème.

Commençons en comparant les expressions verbales figées $⌜$\textsc{s’en} \textsc{aller}$⌝$ et $⌜$\textsc{s’enfuir}$⌝$. Dans, les deux expressions, le morphème \textit{en} est à l’origine la cliticisation d’un complément délocatif du type \textit{de quelque part} : \textit{fuir de quelque part~}→ \textit{en fuir}. Il n’y a pas de différence au niveau du figement sémantique entre ces deux expressions. Pourtant elles sont orthographiées différemment et à juste titre. La question est de savoir si les combinaisons \textit{en} + \textit{aller} et \textit{en} + \textit{fuir} se comportent comme un verbe simple ou comme une combinaison libre clitique \textsc{en} ${\oplus}$ Verbe du genre \textit{en partir} (\textit{partir de quelque part}). Nous allons appliquer la définition de l’analogie structurelle (\sectref{sec:3.1.2}) avec A = A’ = \textit{en}, B = \textit{fuir}/\textit{aller} et B’ = \textit{partir}. Il faut chercher les situations où le clitique \textsc{en} ne se comporte pas de la même façon qu’un préfixe \textit{en-}. Il n’y en a que deux : la combinaison avec un auxiliaire qui viendra séparer le clitique et le verbe (\textit{j’}\textbf{\textit{en}} \textit{suis} \textbf{\textit{parti}}) et l’impératif qui inverse l’ordre (\textit{Pars-en} !). On voit que pour $⌜$\textsc{s’enfuir}$⌝$, \textit{en-} se comporte bien comme un affixe en restant solidaire du radical (\textit{je me suis enfui} ; \textit{Enfuis-toi} !), mais que $⌜$\textsc{s’en} \textsc{aller}$⌝$ se comporte encore comme une combinaison \textsc{en} ${\oplus}$ Verbe (\textit{je m’en suis allé~}; \textit{Va-t-en} !). On note quand même que la forme \textit{je m’en suis allé} apparaît comme très soutenue (voire archaïque) et que, dans un style relâché, on pourra avoir~\textsuperscript{?}\textit{je me suis en allé}, les locuteurs évitant au final de produire l’une ou l’autre des formes. En conclusion, \textit{enfuir} est un seul syntaxème, tandis qu’\textit{en aller} est encore la combinaison de deux syntaxèmes, c’est-à-dire un syntagme.

Un autre exemple est celui de \textsc{bonhomme} et $⌜$\textsc{bonne} \textsc{femme}$⌝$. Ici, on a deux combinaisons Adjectif + Nom qui se sont figées. Les conventions orthographiques veulent qu’on écrive \textsc{bonhomme} en un seul mot, bien que pour son pluriel \textit{bonshommes}, il soit possible de faire la liaison (/\textstylePhono{b\~{ɔ}zɔm}/), tandis que $⌜$\textsc{bonne} \textsc{femme}$⌝$ s’écrit en deux mots. Voyons comme précédemment si cette différence d’orthographe est bien motivée. Il s’agit donc de savoir si ces deux sémantèmes se comporte comme un nom simple ou comme une combinaison libre Adjectif ${\oplus}$ Nom telle que \textit{bonne orange}. Cela ne sera possible que s’il existe une différence distributionnelle entre le nom et la combinaison Adjectif ${\oplus}$ Nom. Une telle différence existe bien : le déterminant indéfini pluriel \textsc{des} possède une forme faible \textit{de} qui s’utilise devant un adjectif, tandis que la forme \textit{des} est obligatoire devant un nom (\textit{Pierre a acheté} \textbf{\textit{des}} \textit{oranges} vs \textit{Pierre a acheté} \textbf{\textit{de}} \textit{bonnes oranges}). Or l’énoncé \textit{\textsuperscript{\#}}\textit{Pierre a rencontré de bonnes femmes} est impossible avec $⌜$\textsc{bonne} \textsc{femme}$⌝$ (l’énoncé n’est pas non plus agrammatical, car il est possible avec la combinaison libre \textsc{bon} ${\oplus}$ \textsc{femme}, mais a alors un autre sens que celui attendu). Le sémantème $⌜$\textsc{bonne} \textsc{femme}$⌝$ n’a donc pas la distribution d’une combinaison libre Adjectif ${\oplus}$ Nom, mais celle d’un nom simple. Si l’on tient compte de cette différence distributionnelle, il s’agit donc d’un syntaxème. Cette propriété est confirmé par la combinaison avec un lexème comme \textsc{mini} qui a la propriété de s’accoler au nom : ainsi on peut dire \textit{une bonne mini voiture}, mais pas *\textit{une mini bonne voiture}. À l’inverse, on dira sans problème \textit{une mini bonne femme}.

\chevalier{À chacun son syntagme}{%\label{sec:3.1.5}
    La notion et le terme de \textit{syntagme} sont empruntés à Saussure qui n’en donne pas de définition formelle. Voici comment il introduit le syntagme :

    \begin{quote}
    «~ Dans le discours, les mots contractent entre eux, en vertu de leur enchaînement, des rapports fondés sur le caractère linéaire de la langue, qui exclut la possibilité de prononcer deux éléments à la fois. Ceux-ci se rangent les uns à la suite des autres sur la chaîne de la parole. Ces combinaisons qui ont pour support l’étendue peuvent être appelées \textbf{\textit{syntagmes}}. Le \textbf{syntagme} se compose donc toujours de deux ou plusieurs unités consécutives (par exemple : \textit{re-lire ; contre tous ; la vie humaine ; Dieu est bon ; s’il fait beau temps, nous sortirons}, etc.). Placé dans un \textbf{syntagme}, un terme n’acquiert sa valeur que parce qu’il est opposé à ce qui précède ou ce qui suit, ou à tous les deux.~» (\citealt{saussure1916cours} : 170)
    \end{quote}

    Il est difficile de savoir ce que recouvrait exactement la notion de syntagme dans l’esprit de Saussure. Tout au plus pouvons-nous dire que les exemples donnés ici par Saussure sont compatibles avec notre définition. L’exemple \textit{re-lire} mérite une discussion, car il est à la limite de ce que nous appelons un syntagme. Le morphème \textit{re-} est souvent considéré comme un préfixe (ce que laisse notamment supposer la convention orthographique qui le lie au verbe qui suit). Pourtant, à la différence des préfixes usuels, \textit{re-} se compose très librement avec les verbes, à tel point qu’on est en droit d’en faire un syntaxème. On peut notamment le combiner avec des phrasèmes (\textsuperscript{?}\textit{il a renoyé le poisson} ; \textsuperscript{?}\textit{il a repris le taureau par les cornes}), le dupliquer (\textit{rerefaire, rerelire}) et on trouve des erreurs de production comme \textit{Je revais lui dire} au lieu de \textit{Je vais lui redire}, qui laisse penser qu’il est quasiment un clitique préverbal.

    L’extrait suivant montre clairement que Saussure inclut dans sa définition du syntagme la combinaison entre un lexème et sa flexion, ce qui est aussi notre cas puisqu’il s’agit d’une combinaison libre (voir Partie 4 pour plus de détails) :

    \begin{quote}
    «~Quand quelqu’un dit \textit{marchons} !, il pense inconsciemment à divers groupes d’associations à l’intersection desquels se trouve le \textbf{syntagme} \textit{marchons} ! Celui-ci figure d’une part dans la série \textit{marche} ! \textit{marchez} !, et c’est l’opposition de \textit{marchons} ! avec ces formes qui détermine le choix ; d’autre part, \textit{marchons} ! évoque la série \textit{montons} ! \textit{mangeons} ! etc., au sein de laquelle il est choisi par le même procédé ; dans chaque série, on sait ce qu’il faut faire varier pour obtenir la différenciation propre à l’unité cherchée.~» (\citealt{saussure1916cours} : 179)
    \end{quote}

    Enfin, contrairement à notre définition, Saussure inclut dans sa définition du syntagme certains faits de morphologie constructionnelle, lorsque ceux-ci relèvent de la parole (c’est-à-dire lorsque les productions font preuve, selon les termes mêmes de Saussure, d’une certaine «~liberté de combinaison~») :

    \begin{quote}
    «~Le propre de la parole, c’est \textbf{la liberté des combinaisons}.

    On rencontre d’abord un grand nombre d’expressions qui appartiennent à la langue ; ce sont les locutions toutes faites, auxquelles l’usage interdit de rien changer, même si on peut distinguer, à la réflexion, des parties significatives.

    […] Mais ce n’est pas tout ; il faut attribuer à la langue, non à la parole, tous les types de \textbf{syntagmes} construits sur des formes régulières. […] Quand un mot comme \textit{indécorable} surgit dans la parole, il suppose un type déterminé, et celui-ci à son tour n’est possible que par le souvenir d’un nombre suffisant de mots semblables appartenant à la langue (\textit{impardonnable}, \textit{intolérable}, \textit{infatigable}, etc.). Il en est exactement de même des phrases et des groupes de mots établis sur des patrons réguliers ; des combinaisons comme \textit{la terre tourne}, \textit{que vous dit-il} ? etc., répondent à des types généraux, qui ont à leur tour leur support dans la langue sous forme de souvenirs concrets.

    Mais il faut reconnaître que dans le domaine du \textbf{syntagme}, il n’y a pas de limite tranchée entre le fait de langue, marque de l’usage collectif, et le fait de parole, qui dépend de la liberté individuelle. Dans une foule de cas, il est difficile de classer une combinaison d’unités, parce que l’un et l’autre facteurs ont concouru à la produire, et dans des proportions qu’il est impossible de déterminer.~»\\
    (\citealt{saussure1916cours} : 172)
    \end{quote}

    Notre notion de syntagme, même si elle est plus restrictive que celle de Saussure, couvre davantage de signes que d’autres usages du terme \textit{syntagme} et notamment celui qui est fait par les grammaires dites syntagmatiques (voir \chapref{sec:3.4}). D’une part, nous considérons que l’unité de base de la syntaxe est le syntaxème et nous appelons bien syntagme toute combinaison de syntaxèmes et non de mots. Ainsi considérons-nous qu’un mot comme \textit{avançait} qui combine librement au moins deux syntaxèmes est un syntagme. Par contre, le signe \textit{indécorable} est selon notre définition un syntaxème complexe, et non un syntagme. D’autre part, nous ne considérons pas, à la différence de l’école anglo-saxonne (qui utilise le terme anglais \textit{phrase}, traduit en français par \textit{syntagme}) qu’un syntagme doive être saturé, c’est-à-dire contenir tous les dépendants de chaque portion du syntagme (cf. \chapref{sec:3.3}). Ainsi dans \textit{Pierre doit chercher sa montre}, nous considérons que \textit{doit chercher} est un syntagme au même titre que \textit{chercher sa montre} ou \textit{Pierre doit}. Ce point sera largement précisé dans le \chapref{sec:3.2} qui suit.
}
\section{Syntaxe et morphologie}\label{sec:3.1.6}

Nous pouvons maintenant définir ce que nous entendons par syntaxe :

\Definition{\textstyleTermes{syntaxe}}
{La \textstyleTermes{syntaxe} est l'\textbf{étude} \textbf{des combinaisons libres de signes linguistiques} et des combinaisons analogues à celles-ci.}

Notre définition se distingue des définitions traditionnelles qui voient la syntaxe comme l’étude de l’organisation des mots dans la phrase. Notre définition ne présuppose ni la délimitation préalable d’une unité minimale de la syntaxe (que serait par exemple le mot), ni la délimitation d’une unité maximale de la syntaxe (que serait la phrase). Notre définition induit une unité minimale de la syntaxe, le syntaxème, que nous définissons en même temps que la syntaxe. La question d’une unité maximale est beaucoup plus complexe et sera abordée dans la partie 6. Notre définition peut être rapprochée de celle d’André Martinet dans son ouvrage de \citeyear{martinet1985syntaxe} intitulé \textit{Syntaxe générale}, où la syntaxe est vue comme «~l’étude des combinaisons des unités significatives d’une langue~» tout en précisant immédiatement que «~la syntaxe n’opère pas avec des unités lexicales particulières, mais avec des classes de telles unités~» (p. 17), ce qui revient bien à ne considérer que des combinaisons libres. La définition d’André Martinet néanmoins ne considère pas, contrairement à nous, qu’il puisse y avoir de la syntaxe dans la combinaison des syntaxèmes au sein d’un sémantème, c’est-à-dire dans les combinaisons qui ne sont pas libres, mais seulement analogues à des combinaisons libres.

La \textbf{flexion}, qui est de la combinatoire libre de syntaxèmes, relève, avec notre définition, avant tout de la syntaxe : il s’agit d’une composante de la syntaxe que nous appelons la \textstyleTermes{syntaxe flexionnelle}, et qui est une grande partie de ce que nous appelons la \textstyleTermes{nanosyntaxe}, qui sera étudiée dans la partie 4. Ceci nous distingue de la grammaire traditionnelle qui considère que l’étude des combinaisons au sein du mot relève de la morphologie. La syntaxe flexionnelle est souvent vue comme relevant à la fois de la morphologie et de la syntaxe ; ce qui lui vaut alors le nom de \textit{morphosyntaxe}.

Pour nous, la \textstyleTermes{morphologie} est l’\textbf{étude des formes}, c’est-à-dire des \textbf{signifiants} des signes linguistiques. On peut parler de \textit{morphologie flexionnelle}, mais cela ne concerne alors, selon notre terminologie, que l’étude des combinaisons des signifiants de grammèmes entre eux et avec les lexèmes. L’étude de la formation des lexèmes comples est généralement appelée la \textstyleTermes{morphologie construction\-nelle} et est une branche de la \textstyleTermes{lexicologie}..

\section{Ce que la syntaxe n’est pas}\label{sec:3.1.7}

La syntaxe est traditionnellement définie comme «~l’étude de l’organisation des mots au sein de la phrase~». Notre définition de la syntaxe s’éloigne de cette définition traditionnelle pour trois raisons.

Premièrement, nous considérons, à la suite de Lucien Tesnière, qu’il existe plusieurs types d’organisation des mots dans la phrase et en particulier une \textbf{organisation hiérarchique} et une \textbf{organisation linéaire}. C’est la seule organisation hiérarchique que nous appelons \textbf{syntaxe}. L’organisation linéaire relève d’une étude différente que nous appelons la \textbf{topologie} (\chapref{sec:3.5}).

Deuxièmement, nous considérons qu’il n’est pas possible de définir le mot avant de définir la notion de combinaison libre. C’est pourquoi notre définition de la syntaxe repose sur la notion de combinaison libre, plus primitive à notre sens que celle de mot. Il en découle naturellement que l’unité minimale de la syntaxe est défini en termes de combinaison libre : c’est ce que nous avons appelé le syntaxème. Le \textbf{mot} n’est donc \textbf{pas l’unité} \textbf{minimale de la syntaxe} pour nous.

Le syntaxème est une unité plus petite ou égale ou mot (voir néanmoins l’\encadref{fig:3.1.17} sur les \textit{Syntaxèmes séparables}). Il s’ensuit que le mot sera défini comme une combinaison de syntaxèmes possédant un niveau de cohésion particulier (\chapfuturef{14}). (Bien que nous ne l’avons pas encore défini formellement, nous nous permettrons de parler de mot tout au long de cette partie, puisque tous nos lecteurs en ont une connaissance intuitive qui est suffisante pour l’instant.)

Troisièmement, nous considérons qu'il est difficile de définir les limites maximales de la syntaxe et de déterminer une unité maximale de la syntaxe. Le terme \textit{phrase} est attaché à différentes notions, notamment celle de \textit{phrase graphique} à l'écrit (voir la discussion dans l'\encadref{fig:0.0.11} sur \textit{Notions, termes, concepts et définitions}), qui ne sont pas nécessairement pertinentes. Quoiqu'il en soit, le concept de \textit{phrase} ne peut être défini avant d'avoir défini l'objet de la syntaxe. Nous mènerons la discussion sur les limites maximales de la syntaxe dans le \chapfuturef{21}, à la toute fin de cet ouvrage.

\chevalier{Historique de la notion de syntaxe}{%\label{sec:3.1.8}
    L’idée que la syntaxe est avant tout l’étude des combinaisons n’est pas nouvelle. Dans sa \textit{Grammaire française sur un plan nouveau, avec un Traité de la prononciation des e et un Abrégé des règles de la poésie française} publiée en \citeyear{buffier1709grammaire}, le jésuite \textbf{Claude Buffier} propose la définition suivante (p. 50) : «~La manière de construire un mot avec un autre mot, par rapport à ses diverses terminaisons selon les règles de la Grammaire, s’appelle la syntaxe.~» (Nous modernisons l’orthographe.) Le chapitre sur la syntaxe du même ouvrage (p. 294) est intitulé «~DE LA SYNTAXE Ou la manière de joindre ensemble les parties d’oraison [= parties du discours] selon leurs divers régimes~» et commence par : «~Ces diverses parties font pour ainsi dire, par rapport à une langue, ce que font les matériaux par rapport à un édifice : quelque bien préparés qu’ils soient, ils ne feront jamais un palais ou une maison, si on ne les place conformément aux règles de l’architecture.~»

    Dans l’article «~Construction~» publié en 1754 dans l’\textit{Encyclopédie} de Diderot et D’Alembert, \textbf{Dumarsais} sépare clairement la syntaxe des questions d’ordre linéaire : «~Je crois qu’on ne doit pas confondre \textit{construction} avec syntaxe. \textit{Construction} ne présente que l’idée de combinaison et d’arrangement. Cicéron a dit selon trois combinaisons différentes, \textit{accepi litteras tuas, tuas accepi litteras,} et \textit{litteras accepi tuas :} il y a là trois \textit{constructions}, puisqu’il y a trois différents arrangements de mots ; cependant il n’y a qu’une syntaxe ; car dans chacune de ces \textit{constructions} il y a les mêmes signes des rapports que les mots ont entre eux, ainsi ces rapports sont les mêmes dans chacune de ces phrases.~»

    La stricte séparation de l’ordre structural et de l’ordre linéaire est constitutive de la syntaxe de \textbf{Lucien Tesnière} qui écrit~(\citeyear{tesniere1959elements}, chapitres 4--7) : «~L’ordre structural des mots est celui selon lequel s’établissent les connexions. […] Toute la syntaxe structurale repose sur les rapports qui existent entre l’ordre structural et l’ordre linéaire. […] Parler une langue, c’est en transformer l’ordre structural en ordre linéaire, et inversement comprendre une langue, c’est en transformer l’ordre linéaire en ordre structural.~[…] Il y a donc antinomie entre l’ordre structural, qui est à plusieurs dimensions (réduites à deux dans le stemma) et l’ordre linéaire, qui est à une dimension. Cette antinomie est la «~quadrature du cercle~» du langage. Sa résolution est la condition \textit{sine qua non} de la parole.~»
}
\section{Dimension paradigmatique du syntaxème}\label{sec:3.1.9}

Comme nous l’avons fait pour les morphèmes, nous regroupons au sein d’un syntaxème différentes occurrences de signes. Pour le morphème, nous avions regroupé tous les signes d’une certaine forme (modulo l’allomorphie, voir la \sectref{sec:2.2.20}) qui possédait une proximité sémantique. Pour le regroupement au sein du syntaxème, nous exigeons que leur distribution syntaxique soit similaire. Notre définition du syntaxème reste de ce point de vue un peu floue : si l’on prend en compte l’ensemble du syntactique, notre regroupement se limitera quasiment uniquement aux signes qui appartiennent à un même sémantème (c’est la position d’Igor Mel’čuk pour qui tout lexème à une acception unique). Nous ne voulons pas pour notre part regrouper uniquement les signes dont le syntactique est absolument identique, mais plutôt ceux dont les syntactiques possèdent un certain recouvrement. Si l’on reprend, l’exemple de \textit{avanc-} (voir la \sectref{sec:2.2.9} sur le \textit{Signème}), nous regrouperons au sein d’un même syntaxème les occurrences qui se combinent avec une flexion verbale et constituent le lexème verbal \textsc{avancer} et celles qui se combinent avec une flexion nominale et constitue le lexème nominal \textsc{avance}.

\Definition{\textstyleTermes{extension paradigmatique du syntaxème}}
{Un \textstyleTermes{syntaxème} est un signème dont les signes sont minimaux pour la décomposition syntaxique et possèdent un syntactique similaire.}


\eiffel{Constructions N \textit{de} N : syntaxèmes ou syntagmes}{%\label{sec:3.1.11}
    L’exemple le plus emblématique en français de constructions à la frontière de la morphologie et de la syntaxe est certainement la combinaison de deux noms (N) par la préposition \textit{de}.

    Tout d’abord, il existe des combinaisons libre N \textit{de} N comme \textit{livre de syntaxe}, dont les composantes commutent librement : \textit{livre/bouquin/cahier … de syntaxe/sémantique/géographie} … Or de telles combinaisons sont très cohésives et quasiment inséparables : \textit{un livre de syntaxe intéressant} vs \textsuperscript{??}\textit{un livre intéressant de syntaxe} (voir l'\encadref{fig:3.5.29} sur la \textit{Topologie du groupe substantival en français}). Il résulte de cela qu’il n’y a pas de différences de comportement notables entre un N simple et une combinaison libre N \textit{de} N.

    Le problème est donc que si l’on considère un sémantème de la forme N \textit{de} N, comme \textrm{$⌜$}\textsc{pomme} \textsc{de} \textsc{terre}\textrm{$⌝$}, il est difficile de dire s’il se comporte comme un N simple (et est un syntaxème complexe) ou s’il se comporte comme une combinaison libre N \textit{de} N (et est un phrasème). Un tel sémantème n’est absolument pas séparable : \textit{une pomme de terre germée} vs *\textit{une pomme germée de terre}. Le pluriel n’étant pas prononcé en français (sauf liaison qui ne peut avoir lieu avec \textit{de}), même si les conventions orthographiques veulent que le \textit{{}-s} aille sur \textit{pomme}, on ne peut pas considérer cela comme une différence avec les noms simples linguistiquement pertinente. On peut contraster cette situation avec celle de l’italien. En italien, les pluriels se prononcent : les noms masculins en -\textit{o} ont un pluriel en -\textit{i}. On peut donc s’assurer que le sémantème \textit{pomodoro}, littéralement \textit{pomo d’oro} ‘pomme d’or’, signifiant ‘tomate’, dont le pluriel est \textit{pomodori} (et pas \textit{pomidoro}) est un syntaxème.

    La construction N \textit{à} Vinf est un autre procédé, clairement syntaxique à la base, qui tend aujourd’hui à ne donner que des formes lexicalisées et donc à se morphologiser. La liste des N \textit{à} Vinf est assez longue, mais ne semble plus permettre de combinaisons libres : \textit{poêle à frire, fer à repasser, fer à friser, planche à découper, table à repasser, machine à laver, machine à calculer, graisse à traire,} etc. Il reste néanmoins des combinaisons libres du type \textit{problème/question/… à résoudre/traiter/reprendre …}.
}
\section{Lexème syntagmatique}\label{sec:3.1.12}

Il existe une famille de lexèmes qui tout en étant bien des syntaxèmes, c’est-à-dire en n’étant analogue à aucune combinaison libre, laisse apparaître une structure syntagmatique figée. Il s’agit de lexèmes comme \textit{un lave-linge, un rendez-vous,} (\textit{une idée}) \textit{à la mors-moi le nœud, un je-m’en-foutiste, je ne sais quelle} (\textit{idée}), \textit{il enterre, il atterrit, une bonne femme, parce que}, etc. Dans chacun de ces lexèmes, on reconnaît la structure d’un syntagme (\textit{lave le linge}, \textit{je m’en fous, en terre, à terre}, etc.), mais le lexème n’est pas analogue à ce syntagme, car il ne possède pas la distribution du syntagme libre. Nous appelons de tels syntaxèmes des \textbf{lexèmes syntagmatiques} (ou des \textbf{syntagmes lexématisés}).

\Definition{\textstyleTermes{lexème syntagmatique}}
{Un \textstyleTermes{lexème syntagmatique} est une combinaison A+B qui n’est analogue à aucun syntagme, mais qui est la version figée d’un syntagme A${\oplus}$B ; autrement dit, il existe des combinaisons libres A’${\oplus}$B’, où A et A’ comme B et B’ sont de distribu\-tions équivalentes, mais A+B et A’${\oplus}$B’ ne le sont pas.}

Les signes lexicaux d’un lexème syntagmatique sont très proche de lexèmes, car ils ont conservé leur signifant, une grande partie de leur signifié et une partie de leur syntactique : dans \textit{lave-vaisselle}, la proximité du signe \textit{vaisselle} avec le lexème \textsc{vaisselle} est très importante, puisqu’au niveau sémantique il est bien question de laver la vaisselle et que la combinaison entre \textit{lave} et \textit{vaisselle} s’apparente à la combinaison du verbe avec son objet. Lorsque l’un des éléments correspond à un lexème plus grammatical comme \textit{en} dans \textit{en}+\textit{terr}(\textit{er}), la combinaison syntaxique devient moins prégnante. Et l’est plus du tout quand la combinaison n’est plus transparente et ne respecte plus la syntaxe du français contemporain comme dans \textit{ce}+\textit{pendant}.

Il existe également des cas limites de combinaisons totalement atypiques, comme \textit{à qui mieux mieux, au petit bonheur la chance} ou \textit{cucul la praline}, qui semblent n’obéir à aucun des procédés de construction de la syntaxe ou de la morphologie ou encore des combinaisons qui attestent de constructions syntaxiques disparues comme l’ordre objet-verbe possible en ancien français : \textit{maintenir}, \textit{ce faisant}, \textit{tambour battant, sans coup férir, il faut raison garder}.

\eiffel{Constructions N N : syntaxèmes ou syntagmes ?}{%\label{sec:3.1.13}
    Un autre exemple de constructions à la frontière de la morphologie et de la syntaxe est celui des combinaisons N N de deux noms. D’un côté, il existe des combinaisons nettement liées comme la \textbf{construction N N} \textbf{coordonnée}, qui associe deux N de manière assez symétrique : \textit{un chien-loup, un enseignant-chercheur, une moisonneuse-batteuse, un hôtel-restaurant, une fille-mère, un enfant-martyr}, etc. Ces combinaisons sont bien liées : si on a un \textit{canapé-lit}, on n’a pas un °\textit{fauteuil-lit} ou un °\textit{canapé-couchette~}; si on a la \textit{physique-chimie}, on n’a pas la °\textit{chimie-physique}. De l’autre côté, il existe des combinaisons parfaitement libres, comme la \textbf{construction N N nominative}, qui associe un grand nombre de noms communs avec n’importe quel nom propre : \textit{le docteur Mabuse, le général Lee, les frères Coen, l’avenue Victor Hugo, la bibliothèque François Mitterrand, les usines Renault, la station Châtelet, l’affaire Dreyfus}, jusqu’à des constructions où le nom propre désigne une époque comme \textit{une table Louis XV}. Certaines paraissent plus figées : à côté de la \textit{région Auvergne}, on n’a pas le \textit{*département Cantal} ou la *\textit{ville Saint-Flour}. On peut encore rapprocher des constructions nominatives des combinaisons libres similaires comme un \textit{bébé phoque} ou une \textit{mère kangourou}.

    Entre les deux, on trouve d’autres constructions N N plus ou moins libres. Commençons par la \textbf{construction N N} \textbf{modificative}, qui apparaît comme la réduction d’une construction N Prép N : \textit{un accès pompiers, une borne incendie, une manif étudiants, un fauteuil relax} (= pour la relaxation), \textit{des pommes vapeur, un steak frites, un coin fumeurs}. Dans cette construction asymétrique, l’un des deux noms va potentiellement pouvoir se libérer : par exemple, à côté de \textbf{\textit{accès}  pompiers}, on trouvera \textbf{\textit{accès}} \textit{handicapés}, \textbf{\textit{accès}} \textit{visiteurs}, \textbf{\textit{accès}} \textit{personnel} (= du personnel) et donc le nom \textit{accès} devient ainsi un nom N1 susceptible de régir un N2 nu, et ce régime est également possible pour des noms sémantiquement similaires comme \textit{entrée} ou \textit{porte}. De même, des noms comme \textit{espace} ou \textit{coin} régissent potentiellement un N2 nu : \textbf{\textit{coin}} \textit{repas,} \textbf{\textit{espace}} \textit{repos,} \textbf{\textit{coin}} \textit{fumeurs,} \textbf{\textit{espace}} \textit{enfants,} \textbf{\textit{coin}} \textit{télé,} etc. Avec ces N1 recteurs, le choix de N2 devient libre. 
    Il y aussi des cas où, à l'inverse, avec certains N2, le choix du N1 devient libre : par exemple à côté d’\textit{accès} \textbf{\textit{handicapés}}, on a aussi \textit{un fauteuil} \textbf{\textit{handicapés}}, \textit{une rampe} \textbf{\textit{handicapés}}, \textit{un ascenseur} \textbf{\textit{handicapés}}, etc. Le N2 se comporte ainsi comme un modifieur pouvant modifier librement un nom, à l'image des adjectifs qualificatifs. Il s'agit d'un phénomène de lexicalisation limité à un N2 particulier, comme \textit{maison} dans \textit{une confiture} \textbf{\textit{maison}} ou \textit{une tarte aux pommes} \textbf{\textit{maison}}, qui ne se propage pas à d'autres noms de la classe sémantique d'origine de N2 ; on n’a pas, par exemple, *\textit{une confiture usine} ou *\textit{une tarte aux pommes boulangerie}.

    Les constructions N N coordonnées ne sont pas toujours symétriques : \textit{un poisson-chat} est un poisson qui a l’allure d’un chat et pas l’inverse. Fonctionnent de manière similaire \textit{un requin-marteau, une guerre éclair, une justice escargot} ou \textit{un discours fleuve}. Il nous semble qu’elles relèvent du même procédé de construction que les autres N N coordonnées, avec pour seule différence un usage métaphorique de N2 : un discours fleuve est un discours au sens propre et un fleuve au sens figuré. Dans les \textbf{constructions N N} \textbf{coordonnées asymétriques}, N2 peut devenir un modifieur assez libre : \textit{satellite/avion} \textbf{\textit{espion}}, \textit{personnage/situation} \textbf{\textit{clé}}, \textit{maison/grammaire} \textbf{\textit{jouet}}, \textit{un livre/film} \textbf{\textit{culte}}\textit{/}\textbf{\textit{phare}}\textit{/}\textbf{\textit{événement}}.

    Parmi les constructions qui deviennent totalement productives, outre la construction N N nominative dont nous avons parlé au début, citons la construction qui associe deux aliments, notamment viande~\textrm{${\oplus}$}~légume, qui est un cas particulier de la construction N\textrm{${\oplus}$}N modificative : \textit{une saucisse frites, un steak salade, une truite pommes vapeur, un côte de porc haricots verts}. On a aussi \textit{un œuf} \textit{mayonnaise} ou \textit{un steak sauce au poivre} ou même \textit{un steak sauce poivre}, illustrant la récursivité. Les ingrédients sont des N2 assez productifs ; mais si on a \textit{une crêpe chocolat} ou \textit{une gaufre confiture}, on n’aura pas *\textit{un gâteau pommes} ou *\textit{une glace fraise}. Par contre avec le doublement de l’ingrédient, on a naturellement \textit{une glace vanille-fraise} ou même des combinaisons très complexes (et très naturelles) comme \textit{une glace deux boules citron vert-chololat amer}, où les N modifieurs sont eux-mêmes modifiés. Les matériaux, comme les ingrédients, s’utilisent assez librement comme modifieurs : \textit{une peinture métal, une montre or, une toiture ardoise, une finition bois, un revêtement pierre}. Comme pour les ingrédients, on ne dira pas *\textit{un pantalon coton}, mais on aura \textit{un pantalon lin-coton}.

    Que conclure de tout ça ? Lorsqu’une combinaison N\textrm{${\oplus}$}N est libre, il s’agit soit d’une construction particulière, soit d’un N1 particulier (qui est devenu recteur), soit d’un N2 particulier (qui est devenu un modifieur de nom). Lorsque N2 devient un modifieur complètement libre, comme \textit{une fenêtre} \textbf{\textit{standard}}, \textit{un cas} \textbf{\textit{limite}} ou \textit{une chaise} \textbf{\textit{marron}}, il tend à passer dans la catégorie adjectivale même s’il reste invariable : il peut alors s’employer avec la fonction d’attribut et être modifié par un adverbe (\textit{cette fenêtre est parfaitement standard, ce cas est très limite, cette chaise est complètement marron}).

    En conclusion, la combinaison N+N doit-elle être toujours considérées comme un syntagme ? S’il existe indéniablement des combinaisons libres N\textrm{${\oplus}$}N, d’autres combinaisons N+N, comme \textit{chien-loup}, ne le sont pas et il n’est pas certain qu’elles puissent être considérées comme analogues à des combinaisons libres. Dans la mesure où les combinaisons liées N+N sont des constructions avec des sémantiques associées à la combinaison assez différentes des combinaisons N\textrm{${\oplus}$}N, on est en droit de considérer qu’il s’agit dans ce cas d’un procédé morphologique — la composition nominale — et que le résultat est un lexème syntagmatique. Comme pour les constructions N \textit{de} N, l’absence de différence de comportement entre un N simple et une combinaison libre N\textrm{${\oplus}$}N ne permet pas de trancher de façon définitive le cas des combinaisons liées N+N.
}
\section{Construction syntagmatique isolée}\label{sec:3.1.14}

Un syntagme est par définition analogue à une combinaison libre. Il existe néanmoins quelques cas de combinaisons de sémantèmes qui ne sont pas analogues à des combinaisons libres, au sens strict où nous avons défini l’analogie, mais que nous voulons néanmoins considérer comme des syntagmes. Le français en offre un bel exemple avec les adjectifs utilisés comme compléments de verbe, comme dans l’énoncé \textit{Cette valise} \textbf{\textit{pèse lourd}}. Il ne s’agit pas d’une combinaison libre, puisqu’aucun adjectif ne peut commuter avec \textsc{lourd}, même pas \textsc{léger}. Les combinaisons du type \textsc{peser} + \textsc{lourd} sont très irrégulières et toujours liées : \textit{coûter cher, parler fort, sonner creux, chanter juste, s’habiller jeune, voter utile,} etc. Il s’agit donc d’une construction qui n’est directement analogue à aucune combinaison libre du français. Il existe des combinaisons libres Verbe ${\oplus}$ Adjectif du type \textit{Cette valise} \textbf{\textit{paraît lourde}}, mais elles sont d’un autre type, puisque l’adjectif est un attribut du nom, qu’il s’accorde avec le nom et qu’il commute librement avec d’autres adjectifs. La combinaison \textsc{peser} + \textsc{lourd} est pourtant bien un syntagme. Si l’adjectif \textsc{lourd} ne peut commuter avec un autre adjectif, il peut commuter avec divers groupes nominaux, qui eux commutent librement : \textit{Cette valise pèse trente kilos, Cette valise pèse un sacré poids,} etc. Il s’agit bien d’une commutation car les deux compléments s’excluent mutuellement : *\textit{Cette valise pèse lourd trente kilos}. Même si la combinaison \textsc{peser} + \textsc{lourd} est liée, \textsc{lourd} commute proprement avec les groupes nominaux. Le sens de la combinaison \textsc{peser} + \textsc{lourd} est compositionnel et il s’agit bien d’une combinaison de deux sémantèmes choisis séparément (même si les deux choix sont liés). Par ailleurs, \textsc{peser} et \textsc{lourd} peuvent être séparés et surtout ils peuvent être modifiés indépendamment l’un de l’autre : \textit{Cette valise} \textbf{\textit{ne pèse pas}} \textit{lourd, Cette valise pèse} \textbf{\textit{plus lourd que prévu}}. Le comportement de la combinaison \textsc{peser} + \textsc{lourd} est donc analogue à celui des combinaisons libres et non à celui des syntaxèmes, qui même lorsqu’ils sont complexes, ne peuvent être modifiés que comme un tout.

Nous ajoutons donc à notre définition du syntagme l’extension suivante :

\Definition{syntagme lié}
{Toute \textbf{combinaison de deux sémantèmes} est un syntagme, même si cette combinaison est liée et n’est analogue à aucune combinaison libre.}

\section{Inséparabilité linéaire}\label{sec:3.1.15}

L’inséparabilité linéaire est certainement la propriété la plus remarquable des syntaxèmes. Elle est souvent utilisée comme propriété définitoire, notamment lorsque les syntaxèmes sont définis à partir des mots.

\Definition{séparabilité (linéaire)}
{Une combinaison A+B est (\textstyleTermes{linéairement}) \textstyleTermes{séparable} s’il existe une classe non fermée de syntaxèmes qui peuvent venir s’intercaler entre A et B sans changer la nature de la combinaison entre A et B.}

L’inséparabilité linéaire des syntaxèmes ne peut pas être démontrée (elle peut bien sûr être vérifiée pour les langues déjà décrites, mais pas pour la totalité des syntaxèmes de la totalité des langues). Elle peut simplement être constatée (voir néanmoins les cas limite dans l’\encadref{fig:3.1.16} qui suit). Elle découle très logiquement du fait qu’il n’y a aucune raison qu’un élément vienne séparer un syntaxème. Ce qui devrait donc nous surprendre n’est pas qu’un syntaxème soit inséparable, mais plutôt qu’un sémantème soit séparable ; par exemple, qu’un phrasème comme $⌜$\textsc{noyer} \textsc{le} \textsc{poisson}$⌝$ puisse être séparable : dans \textit{Pierre noyait toujours le poisson}, la flexion \textit{{}-ait} et l’adverbe \textit{toujours} se sont intercalés. Ceci montre le caractère particulier des phrasèmes : l’ensemble forme une unité sémantique, mais seule une partie, ici le verbe \textsc{noyer}, est syntaxiquement accessible et donc les syntaxèmes qui se combinent avec $⌜$\textsc{noyer} \textsc{le} \textsc{poisson}$⌝$ vont en fait se combiner au seul verbe \textsc{noyer} et se positionner par rapport à lui seul.

C’est le caractère inséparable et l’absence d’indépendance de ses composantes qui rend le syntaxème complexe assez différent d’un phrasème. Il faut néanmoins se souvenir que l’inséparabilité n’est pas un critère pour identifier les syntaxèmes : il s’agit d’un indice soit de figement, soit de cohésion syntaxique. (La cohésion syntaxique et les syntagmes inséparables sont étudiés dans la partie 4, consacrée à la \textit{Nanosyntaxe}.)

\globe{Syntaxèmes discontinus}{%\label{sec:3.1.16}
    Des langues comme le français pourraient laisser penser que les syntaxèmes se combinent par \textbf{concaténation}, c’est-à-dire s’accolent toujours les uns derrière les autres ou éventuellement se fusionnent (voir la \sectref{sec:2.2.22} sur \textit{Amalgame, alternance et mégamorphe}). Mais, il existe des langues où les syntaxèmes se combinent en s’imbriquant davantage. Tel est le cas des langues sémitiques ou des langues austronésiennes.

\todo[inline]{Small vskip}
\noindent \textbf{Arabe.} 
En arabe, la plupart des verbes possèdent un radical formé de 3 consonnes, tandis que la flexion verbale vient compléter ce radical discontinu. Considérons le paradigme:

\ea
\begin{multicols}{2}
\ea \textit{kataba zajdun}\\\glt ‘Zayd a écrit’
\ex \textit{jaktubu zajdun}\\\glt   ‘Zayd écrit’
\ex \textit{kutiba kit\=abun}\\\glt   ‘un livre a été écrit’
\ex \textit{ʔakala zajdun}\\\glt   ‘Zayd a mangé’
\ex \textit{jaʔkulu zajdun}\\\glt   ‘Zayd mange’
\ex \textit{ʔukila tuff\=aḥatun}\\\glt   ‘une pomme a été mangée’.
\z
\end{multicols}
\z
    On en déduit que les verbes sont \textsc{k\_t\_b\_} ‘écrire’ et \textsc{ʔ\_k\_l\_} ‘manger’, tandis que les amas flexionnels sont \textit{\_a\_a\_a} = actif.passé.3.sg, \textit{ja\_\,\_u\_u} = actif.présent.3.sg et \textit{\_u\_i\_a} = passif.passé.3.sg.

    De tels exemples ne remettent pas en cause le caractère inséparable des syntaxèmes. De tels syntaxèmes sont bien \textbf{discontinus}, mais les éléments qui viennent s’intercaler entre les différentes parties du lexème sont des syntaxèmes flexionnels, c’est-à-dire des éléments appartenant à une classe fermée et indissociables du lexème. Nous ne parlons de séparabilité que lorsqu’il y a insertion possible d’une classe non fermée d’éléments.

\todo[inline]{Small vskip}
\noindent \textbf{Anglais.} 
    Notons encore l’importance de restreindre, dans la définition de la séparabilité, l’élément intercalaire à une classe non fermée. En anglais britannique oral, l’insertion d’un syntaxème comme \textit{fucking} (\textit{a fucking dog} ‘un putain de chien’) est possible à l’intérieur de mots, aux frontières de morphèmes (\textit{un-fucking-believable} ‘complètement incroyable’) ou bien devant la syllabe accentuée (\textit{abso-fucking-lutely} ‘absolument’). Une telle division de syntaxèmes, appelée une \textstyleTermesapprof{\textup{tmèse}}, est très restreinte lexicalement, et ne met pas en cause l’inséparabilité linéaire du syntaxème.
}
\globe{Syntaxèmes séparables ?}{%\label{sec:3.1.17}
    Certains sémantèmes séparables possèdent une syntaxe suffisamment particulière pour ne pas être analogue à une combinaison libre et être éventuellement déclarés comme des exemples de syntaxèmes séparables. Si de tels éléments existent, ils restent extrêmement marginaux.

    L’exemple qui se rapprocherait le plus d’une telle situation est celui des verbes à particule de l’anglais, de l’allemand et des langues germaniques en général.

\todo[inline]{Small vskip}
   \noindent
   \textbf{Verbes à particules (langues} \textbf{germaniques).} Les verbes à particules de l’allemand ont été longtemps décrits comme des verbes à préfixe séparable. Nous allons commencer par regarder les verbes à particule de l’anglais avant de présenter ceux de l’allemand.

    Les verbes à particule de l’anglais (appelés généralement \textit{phrasal verb}, littéralement ‘verbe syntagmatiques’) sont les tournures verbales du type : \textit{pick up} ‘ramasser’, \textit{go on} ‘continuer’, \textit{take off} ‘enlever’, \textit{figure out} ‘arriver à comprendre’, etc. Une grande partie de ces tournures sont figées. Si elles l’étaient toutes, ceci fournirait un parfait exemple de syntaxèmes séparables, puisque le complément d’objet peut parfois se placer entre le verbe et la particule : \textit{I} \textbf{\textit{picked}} \textit{it} \textbf{\textit{up}} ‘je l’ai ramassé’, \textit{they had to} \textbf{\textit{take}} \textit{his leg} \textbf{\textit{off}} ‘on a dû l’amputer d’une jambe’. Néanmoins, une partie des verbes à particule sont clairement des syntagmes et donc par analogie tous le sont. Il s’agit des combinaisons d’un verbe de déplacement et d’une \textbf{particule directionnelle~}\textit{: go/walk/swim/crawl/drive/pull/put … in/out/up/down/through…} litt. ‘aller/marcher/nager/ramper/conduire/tirer/mettre … dans/hors/vers le haut/vers le bas/à travers …’ mais qu’on traduirait plutôt en français par \textit{entrer/sortir/monter/descendre/traverser … à pied/à la nage/à quatre pattes/en voiture/en tirant …}). Ici, même si certaines combinaisons peuvent se figer (et devenir ainsi polysémiques) comme \textit{go on}, toutes les combinaisons sont à peu près possibles et la commutation est bien libre.

    Dans la tradition allemande, les verbes du même type sont écrits en un seul mot à l’infinitif : \textit{hineingehen, herauslaufen, durchschwimmen,} \textbf{\textit{hochziehen}}, \textit{…} Ces formes sont bien pourtant des combinaisons libres de particules (\textit{hinein/heraus/}\textbf{\textit{hoch}}\textit{/unter/durch} \textit{…} ‘dans / hors / \textbf{vers le haut} / vers le bas / à travers / …’) avec des verbes simples (\textit{gehen/laufen/schwimmen/kriechen/fahren/}\textbf{\textit{ziehen}}\textit{/stellen} \textit{…} ‘aller/marcher/ nager/ramper/conduire/\textbf{tirer}/mettre … ’). À l’infinitif, la particule précède toujours directement l’infinitif, ce qui pourrait expliquer la convention orthographique. (À noter toutefois que, dans les constructions appelées \textit{Zwischenstellung}, l’auxiliaire s’interpose entre la particule et le verbe en présence d’un verbe modal comme dans : \textit{Er kam, weil er das Boot} \textbf{\textit{hoch}} \textit{hat} \textbf{\textit{ziehen}} \textit{wollen} ‘Il est venu parce qu’il a voulu monter le bateau en le tirant’.) Dans les formes finies, l’ordre des mots ressemble plus à celui de l’anglais et la particule peut être très éloignée du lexème verbal: \textit{Er} \textbf{\textit{zieht}} \textit{das Boot} \textbf{\textit{hoch}}, litt. ‘Il \textbf{tire} le bateau \textbf{vers le haut}’, c’est-à-dire ‘Il hisse le bateau’. (Nous développons dans l'\encadref{sec:3.5.37} sur la \textit{Structure topologique de l’allemand} la description du placement du verbe et de la particule en allemand.)

    Notons encore que les particules peuvent se figer totalement et devenir alors un vrai préfixe comme dans \textit{unterstellen}, litt. sous-poser ‘laisser entendre’, \textit{übersetzen}, litt. à travers-mettre ‘traduire’. Cette construction n’est plus analogue à une combinaison libre, puisque le préfixe n’est plus séparable et il s’agit bien alors d’un lexème (complexe) : \textit{Er} \textbf{\textit{unterstellt}} \textit{mir ein Rassist zu sein} ‘Il laisse entendre que je suis raciste’ et \textit{Er} \textbf{\textit{übersetzt}} \textit{den Artikel ins Französische} ‘Il traduit l’article en français’. Ces verbes possèdent toujours une acception compositionnelle avec l’ordre habituel : \textit{Er} \textbf{\textit{stellt}} \textit{das Auto} \textbf{\textit{unter}}. litt. ‘Il pose la voiture dessous’, c’est-à-dire ‘Il met la voiture à l’abri’ ; \textit{Sie} \textbf{\textit{setzte}} \textit{mit dem Schiff} \textbf{\textit{über}}. litt. ‘Elle met avec le bateau à travers’, c’est-à-dire ‘Elle a traversé en bateau’.

\todo[inline]{Small vskip}\noindent
    \textbf{Négation double (français).} La négation \textit{ne … pas} du français est également un candidat intéressant au titre de syntaxème séparable. Le français exprime en effet par deux mots un sens que la majorité des langues expriment par un seul mot ; par exemple, \textit{Je} \textbf{\textit{ne}} \textit{comprends} \textbf{\textit{pas}} se dit \textbf{\textit{No}} \textit{entiendo} en espagnol, \textit{Ich verstehe} \textbf{\textit{nicht}} en allemand, \textit{wǒ} \textbf{\textit{bù}} \textit{dǒng} en chinois, etc. Pourtant, on considère qu’il s’agit de deux syntaxèmes car \textsc{ne} ou \textsc{pas} peuvent fonctionner seul (\textit{Je peux} \textbf{\textit{pas}} \textit{venir,} \textbf{\textit{Pas}} \textit{de ça ici, Je} \textbf{\textit{ne}} \textit{peux accepter ça, Je} \textbf{\textit{ne}} \textit{saurais trop vous conseiller de …}) et que \textsc{ne} peut être associé à d’autres éléments, comme \textsc{plus,} \textsc{personne,} \textsc{rien} ou \textsc{jamais}, ce qui signifie que \textsc{pas} commute librement ici, même si le paradigme est fermé.

\todo[inline]{Small vskip}\noindent
    \textbf{Accord.} Certains linguistes, comme Zelig Harris ou André Martinet, considèrent des séquences d’accords (comme les trois pluriels dans \textit{l}\textbf{\textit{es} }\textit{anim}\textbf{\textit{aux}} \textit{boiv}\textbf{\textit{ent}}) comme un seul morphème séparable. Nous avons dit dans l’\encadref{fig:2.3.4} \textit{Constructions verbales et accords : signes vides} ? qu’ils nous semblaient effectivement judicieux de considérer la séquence comme l’expression d’un unique sémantème, c’est-à-dire d’un unique choix du locuteur, mais de considérer quand même chaque accord comme un syntaxème flexionnel séparé. Ceci est justifié par le fait qu’il n’y a pas nécessité à ce que les différents syntaxèmes commandés par le sémantème en question soient toujours présents ensemble. Considérer de tels objets comme des syntaxèmes reviendrait non seulement à considérer des syntaxèmes séparables, mais qui plus est des syntaxèmes à géométrie variable possédant selon les contextes un, deux, trois ou davantage encore de morceaux.

\todo[inline]{Small vskip}\noindent
    \textbf{Corrélatifs (français).} Il existe en français des phrasèmes dit corrélatifs, car composés de deux éléments corrélés et se comportant de manière similaire. Un premier exemple est \textrm{$⌜$}\textsc{ci} \textsc{…} \textsc{ça}\textrm{$⌝$}~comme dans \textit{Fais pas} \textbf{\textit{ci~}}! \textit{Fais pas} \textbf{\textit{ça~}}\textit{!~}ou \textit{un coup comme} \textbf{\textit{ci}}, \textit{un coup comme} \textbf{\textit{ça}}. Ce sémantème a la particularité d’imposer le dédoublement de l’élément avec lequel il se combine, chacune de ses parties se combinant avec l’un des deux éléments dédoublés. Comme chacune de ses parties se comporte comme un pronom usuel (par exemple comme \textsc{ça}), on est en droit de considérer qu’il s’agit de deux syntaxèmes et pas d’un seul. Néanmoins les deux parties de \textrm{$⌜$}\textsc{ci} \textsc{…} \textsc{ça}\textrm{$⌝$}~ne se combinent pas syntaxiquement entre elles et il ne s’agit donc pas non plus d’un syntagme.

    Un autre exemple de corrélatif est celui de \textrm{$⌜$}\textsc{plus} \textsc{…} \textsc{plus} \textsc{…}\textrm{$⌝$} : \textit{Plus il mange, plus il grossit}. À la différence de \textrm{$⌜$}\textsc{ci} \textsc{…} \textsc{ça}\textrm{$⌝$}~, les deux composantes de \textrm{$⌜$}\textsc{plus} \textsc{…} \textsc{plus} \textsc{…}\textrm{$⌝$}~se comportent de manière assez atypiques, même si d’autres adverbes peuvent venir dans cette position (\textit{Quelquefois il mange~}; \textit{Toujours il grossit}), ce qui pourrait lui valoir le statut de syntaxème séparable. Ce qui nous fait quand même pencher pour le considérer comme un syntagme est que ses composantes, les deux \textsc{plus}, possèdent encore certaines propriétés du comparatif \textsc{plus~}: certes ils n’occupent pas la position du comparatif (\textit{Il mange plus} vs *\textit{Plus il mange}), mais ils peuvent commuter avec \textsc{moins} (\textit{Plus il mange, moins il a d’énergie} ; \textit{Moins il mange, plus il est irritable}) et ils acceptent encore des formes supplétives avec certains adjectifs (\textit{Plus le vin vieillit,} \textbf{\textit{meilleur}} \textit{il est} vs \textit{Plus le vin vieillit,} \textbf{\textit{plus}} \textit{il est} \textbf{\textit{bon}}).
}
\exercices{%\label{sec:3.1.18}
    \exercice{1} Y a-t-il des combinaisons libres structurellement analogues à \textit{haut de gamme} dans \textit{des chaussures haut de gamme} ?

    \exercice{2} Quel problème pose des expressions comme (\textit{mener une affaire}) \textit{tambour battant} ou \textit{sans coup férir} ?

    \exercice{3} Pour les paires suivantes, discuter s’il vous paraît justifié ou non d’écrire chaque signe en un ou deux mots, sachant qu’un syntaxème s’écrit normalement en un mot, mais pas un syntagme comportant plusieurs lexèmes :
    
    \begin{enumerate}[label=\alph*.,font=\upshape]
    \item \itshape s’en aller  {\upshape vs} s’enfuir ;
    \item \itshape parce que   {\upshape vs} puisque ;
    \item \itshape à côté      {\upshape vs} autour ;
    \item \itshape bonne femme {\upshape vs} bonhomme ;
    \item \itshape autre chose {\upshape vs} autrefois.
    \end{enumerate}

    \exercice{4} Est-ce que \textit{homme-grenouille} est un syntaxème ou un syntagme ?

    \exercice{5} Est-ce que \textit{il} et \textit{dort} dans \textit{il dort} sont considérés comme linéairement séparable ?
}
\lecturesadditionnelles{%\label{sec:3.1.19}
    On ne peut pas dire qu’il y ait de consensus actuel sur la question de la frontière entre syntaxe et morphologie. Le fait de considérer la morphologie comme l’étude des formes, et pas seulement l’étude des mots, reprend l’architecture de la Théorie Sens-Texte d’igor Mel’čuk (voir son livre de 1988 ou celui de 2014 avec Jasmina Milićević). Mel’čuk considère néanmoins, à la suite de Tesnière, que le mot est l’unité minimale de la syntaxe, même si, dans les représentations syntaxiques qu’il propose, les mots sont décomposés en syntaxèmes.

    Le livre de Charles Hockett de 1958 reste un des plus bels ouvrages de cette époque. Ici c’est le morphème qui est considéré comme l’unité minimale de l’ «~analyse grammaticale~». Le mot est défini dans un deuxième temps comme dans notre approche. Hockett ne dégage pas le concept de syntaxème, mais s’en approche en donnant une grande importance aux paradigmes syntaxiques.

    Les travaux en grammaire générative et syntaxe X-barre considèrent que la flexion relève de la syntaxe (notamment avec les constituants IP, Inflection Phrase), mais sans aller jusqu’à considérer les syntaxèmes flexionnels comme des signes linguistiques. Les syntaxèmes flexionnels sont traités comme des traits syntaxiques. Voir par exemple l’article de Stephen Anderson (1982).

    La morphologie lexématique est présentée dans le traité de morphologie de Bernard Fradin (2003). Les combinaisons N+N~y sont étudiées (p. 201).

    Les combinaisons figées Verbe-Adjectif, comme \textit{peser lourd} ou \textit{sonner creux} sont étudiées par Pierre Le Goffic (1993: 367).

    Pour ceux qui s’intéressent à la négation double, on pourra consulter le travail d'Otto Jespersen (1917) et les travaux ultérieurs qui y font référence en tant que \textstyleTermes{cycle de Jespersen}\textstyleTermesapprof{} : au cours de l’évolution d’une langue, la négation tend à s’affaiblir, puis à être doublé par un autre syntaxème qui finit par supplanter le syntaxème initial et ainsi de suite.

    \FurtherReading{3-1}
}
\corrections{%\label{sec:3.1.20}
    \corrigé{1} Dans \textit{haut de gamme}, \textit{haut} est un nom (\textit{le haut de la gamme}). Il s’agit donc d’une construction «~N \textit{de} N~» qui se comporte comme le N2 d'une construction N N coordonnée asymétrique, c'est-à-dire des N2 tels que \textit{culte} ou \textit{jouet}. De tels N2 sont lexicalisés et se situe à la frontière entre nom et adjectif. Ceci nous amène à considérer \textit{haut de gamme} dans cet emploi de modifieur comme un \textbf{lexème syntagmatique} (voir \sectref{sec:3.1.12} éponyme).

    \corrigé{2} Les expressions \textit{tambour battant} et \textit{sans coup férir} datent d’une époque où les compléments pouvaient être placés devant le verbe. Si ces constructions sont clairement d’origine syntaxique, elles ne sont plus aujourd’hui analogues à aucune construction libre du français et doivent donc être considérées comme des lexèmes syntagmatiques (voir \sectref{sec:3.1.12} éponyme).

    \corrigé{3} 
    \begin{enumerate}[label=\alph*.]
    \item\textit{enfuir} est un lexème syntagmatique, mais \textit{en aller} a encore des propriétés syntaxiques (voir \sectref{sec:3.1.4} \textit{Syntagme ou syntaxème} ?~pour les détails).

    \item\textit{parce que} est l’exemple type d’un lexème syntagmatique dont l’orthographe n’est pas motivé. L’origine de l’expression est \textit{par ce que} et \textit{parce} n’est pas un lexème du français. À l’inverse, \textit{puisque} est dans le paradigme de \textit{alors que, bien que} ou \textit{lorsque} où un adverbe se combine avec la conjonction \textit{que}. Néanmoins, la très faible diagrammaticité de la combinaison \textit{puis} + \textit{que} et la prononciation particulière de \textit{puis} [pɥis] dans \textit{puisque} justifie tout particulièrement de ne pas décomposer en synchronie.

    \item Les combinaisons entre une préposition et un nom sans déterminant, comme \textit{à côté}, sont à la limite entre syntagme et lexème syntagmatique, car d’un côté la combinaison entre une préposition et un nom est généralement libre, mais il n’existe pas de combinaison libre avec la préposition \textit{à} lorsque le nom est sans déterminant. Dans le cas \textit{autour}, il s’agit plus clairement d’un syntagme, puisque le déterminant est présent et donc la convention orthographique est peu motivée.

    \item \textit{bonhomme} et \textit{bonne femme} sont tous les deux des lexèmes syntagmatiques (voir \sectref{sec:3.1.4} \textit{Syntagme ou syntaxème} ?).

    \item On est ici face à des constructions qui sont clairement d’origine syntaxique, mais qui sont devenues irrégulières aujourd’hui en raison de l’absence d’article. Il est néanmoins clair que la combinatoire de \textit{autrefois} est déterminée par le nom temporel \textit{fois} (voir le \chapfuturef{17}), ce qui justifierait de l’orthographier \textit{autre fois}.
    \end{enumerate}

    \corrigé{4} Comme nous avons pu le voir dans l'\encadref{sec:3.1.13} sur les \textit{Constructions N N}, les N+N coordonnés asymétriques peuvent parfois être libres (\textit{satellite espion, film culte}), mais ce n'est pas avec \textit{grenouille} qui se combine uniquement avec \textit{homme}, donc \textit{homme-grenouille} est un lexème syntagmatique.

    \corrigé{5} Les seuls syntaxèmes qui peuvent s’intercaler entre \textit{il} et \textit{dort} sont des clitiques (\textit{il n’y dort pas}). Il n’y donc pas de classe ouverte de syntaxèmes qui peut venir séparer ces deux mots, qui sont donc considérés comme linéairement inséparables, selon notre définition donnée dans la \sectref{sec:3.1.15} \textit{Inséparabilité linéaire}.
}
