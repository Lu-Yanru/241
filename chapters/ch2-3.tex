\chapter{\gkchapter{Sémantèmes et syntaxèmes}{Unités de sens \textit{vs} unités de combinatoire}}\label{sec:2.3}

\section{Arbitraire du sémantème}\label{sec:2.3.0}

L’étude des sémantèmes est une question de \hi{sémantique} qui dépasse les objectifs de ce livre. Mais, pour bien comprendre ce que sont les syntaxèmes, il nous semble utile de clarifier d’abord ce que sont les sémantèmes. Comme nous le verrons dans ce chapitre, l’extension des syntaxèmes, qu’elle soit prise dans sa dimension syntagmatique ou paradigmatique, est généralement comprise entre celle des morphèmes et celle des sémantèmes (voir notamment l’\encadref{sec:2.3.23} sur les \textit{Extensions paradigmatique et syntagmatique}).

Notre définition des unités sémantiques est plus restrictive que celle des signes linguistiques. Revenons sur notre définition des signes donnée à la \sectref{sec:2.2.2} sur la \textit{Décomposition propre des signes}. Nous avons vu que le signe \textit{broyeur} se décompose proprement en \textit{broy~}${\boxplus}$\textit{~eur}, car \textit{broyeur} est à \textit{broyer} ce que \textit{compresseur} est à \textit{compresser}. Les composantes \textit{broy-} et \textit{{}-eur} de \textit{broyeur} peuvent donc se voir attribuer une contribution sémantique propre et sont ainsi des signes au sens plein du terme. Mais malgré cela, \textit{broyeur} n’est pas compositionnel (voir définition dans la section suivante). Il y a quelque chose d’\hi{arbitraire} dans le signe \textit{broyeur}, tant au niveau du signifiant que du signifié. Au niveau du signifiant, pourquoi utilise-t-on \textit{broyeur} pour un appareil qui sert à broyer et pas \textit{laveur} pour une machine qui sert à laver ? Au niveau du signifié, on ne désigne pas par \textit{broyeur} n’importe quel appareil qui sert à broyer : un hachoir à viande, qui broie plus qu’il ne hache, s’appellera toujours un hachoir.

Ce qui est vrai pour \textit{broyeur} l’est aussi pour des combinaisons syntaxiques, comme \textit{machine à laver}. La combinaison \textit{machine à laver} peut paraître plus libre que \textit{broyeur}, pourtant lorsqu’on fait des commutations sur \textit{machine} ou \textit{laver}, on obtient des combinaisons comme \textit{appareil à laver} ou \textit{machine à broyer} dont la nature est différente de celle de \textit{machine à laver}. Une machine à laver n’est pas n’importe quelle machine qui sert à laver : \textit{machine à laver} désigne un type particulier d’appareil ménager servant à laver le linge. On n’appellera pas \textit{machine à laver} la machine qui sert à laver les voitures dans les stations services. Il y a donc quelque chose d’\hi{arbitraire} dans la relation entre le sens ‘appareil ménager servant à laver le linge’ et le signifiant \textit{machine à laver.}

Les unités telles que \textit{broyeur} ou \textit{machine à laver} dont on peut trouver en partie le sens à partir du sens de leurs composants sont dites \textstyleTermes{transparentes}. Les signes qui composent un signe complexe transparent ne sont pas pour autant nécessairement des unités sémantiques. Lorsqu’on parle de transparence, on raisonne dans le sens de l’analyse, du décodage des unités. Pour bien comprendre la spécificité des unités sémantiques parmi les signes linguistiques, il faut se placer dans le sens de la \hi{synthèse}, de la production d’un énoncé à partir d’un sens.

\section{Chaque sémantème suppose un choix}\label{sec:2.3.1}

Nous proposons de caractériser les unités sémantiques à partir de la notion de \textstyleTermes{choix} introduite par André Martinet. Voici ce qu’il en dit dans le chapitre intitulé \textbf{\textit{Chaque unité suppose un choix}} de ses \textit{Eléments de linguistique générale} (\citeyear{martinet1960elements} : 26) :

\begin{quote}
    «~Soit un énoncé comme \textit{c’est une bonne bière} /\textstylePhono{sɛtynbɔnbiɛr}/. […] Si nous sommes à même de dire quelque chose sur les latitudes combinatoires de /\textstylePhono{bɔn}/, c’est que ce segment de l’énoncé a été reconnu comme représentant une unité particulière distincte de /\textstylePhono{yn}/ et de /\textstylePhono{biɛr}/. Pour arriver à ce résultat, il a fallu constater que /\textstylePhono{bɔn}/, dans ce contexte, correspondait à un \textbf{choix} spécifique entre un certain nombre d’épithètes possibles ; la comparaison d’autres énoncés français a montré que dans les contextes où figure /\textstylePhono{bɔn}/ on trouve aussi /\textstylePhono{eksɛlãt}/ (\textit{excellente}), /\textstylePhono{movɛz}/ (\textit{mauvaise}), etc. Ceci indique que le locuteur a, plus ou moins consciemment, écarté tous les compétiteurs qui auraient pu figurer entre /\textstylePhono{yn}/ et /\textstylePhono{biɛr}/, mais qui ne se trouvaient pas convenir en l’occurrence. Dire de l’auditeur qu’il comprend le français implique qu’il identifie par expérience les choix successifs qu’a dû faire le locuteur, qu’il reconnaît /\textstylePhono{bɔn}/ comme un choix distinct de /\textstylePhono{yn}/ et de celui de /\textstylePhono{biɛr}/, et qu’il n’est pas exclu que le choix de /\textstylePhono{bɔn}/ au lieu de /\textstylePhono{movɛz}/ influence son comportement.~» (Dans cette citation, nous avons modifié les transcriptions phonémiques et utilisé les conventions de cet ouvrage.)
\end{quote}

Prenons l’exemple de l’énoncé \textit{Pierre mange une pomme de terre}. Dans cet énoncé, les mots \textit{pomme}, \textit{de} et \textit{terre} sont des signes linguistiques, mais aucun d’eux ne résulte d’un choix.

\Definition{\textstyleTermes{choix}}
{Par \textstyleTermes{choix}, nous entendons les choix lexicaux et grammaticaux que fait le locuteur lorsqu’il cherche à \hi{exprimer un sens} qu’il veut communiquer.}

Lorsque le locuteur produit \textit{pomme de terre}, \textit{pomme} n’a pas été choisi par opposition à \textit{poire} ou \textit{banane}, \textit{de} n’a pas été choisi par opposition à \textit{à} ou \textit{dans}, et \textit{terre} n’a pas été choisi par opposition à \textit{eau} ou \textit{feu}. C’est bien \textit{pomme de terre} dans son intégralité qui a été choisi par opposition à \textit{carotte, chou-fleur} ou \textit{haricot vert}. Ainsi \textit{pomme de terre, carotte, chou-fleur} et \textit{haricot vert} forment un \textstyleTermes{paradigme de choix} ou \textstyleTermes{système d’opposition} et chacun de ces choix est \textstyleTermes{indivisible}. Il en résulte que \textit{pomme de terre} est ici une unité sémantique et les segments qui la composent n’en sont pas (dans cet énoncé).

\Definition{\textstyleTermes{unité sémantique}}
{Nous appelons \textstyleTermes{unité sémantique} tout signe qui résulte d’un ou plusieurs choix du locuteur.}

Il existe des signes comme \textit{machine} dans \textit{machine à laver} qui ne forment pas une unité sémantique, mais seulement une composante d’une unité sémantique. Il existe aussi des signes comme les syntaxèmes d’accord en genre des adjectifs avec le nom ou le syntaxème d’infinitif qui ont une signification purement grammaticale et ne sont donc ni des unités sémantiques, ni même des composantes d’une unité sémantique. De tels signes ne résultent pas de choix du locuteur ; ils sont entièrement requis par la grammaire de la langue. Nous développons ce point dans la \sectref{sec:2.3.3} sur le \textit{Syntaxème vide}.

\Definition{\textstyleTermes{sémantème}}
{Nous appelons \textstyleTermes{sémantème} tout signe qui résulte d’un \hi{choix indivisible}. Les sémantèmes sont les \hi{unités sémantiques minimales} : ils ne peuvent pas être décomposés en deux unités sémantiques.}

Ainsi \textit{pomme de terre} est une unité sémantique, mais ses composantes \textit{pomme, de} et \textit{terre} n’en sont pas. Il en résulte que \textit{pomme de terre} est un sémantème. Il en va de même pour \textit{broyeur} ou \textit{machine à laver}.

\Definition{\textstyleTermes{compositionalité}}
{Une unité sémantique qui peut être décomposée en deux unités sémantiques est dite \textstyleTermes{compositionnelle}.}

Par exemple, une unité comme \textit{livre de syntaxe} est compositionnelle. Ici \textit{livre} et \textit{syntaxe} sont choisis librement et peuvent commuter avec des quasi-synonymes comme \textit{bouquin} ou \textit{grammaire} (\textit{un bouquin de syntaxe, un livre de} grammaire) sans que la variation de sens soit plus importante que la variation de sens qui existe ailleurs entre \textit{livre} et \textit{bouquin} ou entre \textit{syntaxe} et \textit{grammaire} (\textit{j’ai acheté un livre/bouquin, j’aime la syntaxe/grammaire}).

\section{Motivation du signe}\label{sec:2.3.2}

Un sémantème qui est composé de plusieurs signes est dit \textstyleTermes{complexe} (voir la \sectref{sec:2.2.14} sur \textit{Signème libérable, radical, affixe}). L’existence de sémantèmes complexes, comme \textit{broyeur} ou \textit{machine à laver}, complexifie le système linguistique, puisqu’il n’y a plus correspondance systématique entre les segments minimaux du point de vue du sens (les sémantèmes) et les segments minimaux du point de vue de la forme (les morphèmes). Mais d’un autre côté, l’utilisation de sémantèmes complexes offre une certaine économie en réduisant le caractère \hi{arbitraire} du lien entre la forme et le sens. Cette relative \textstyleTermes{motivation} du signifiant des sémantèmes complexes a été bien dégagée par \citet[180]{saussure1916cours} :

\begin{quote}
    «~Le principe fondamental de l’arbitraire du signe n’empêche pas de distinguer dans chaque langue ce qui est radicalement arbitraire, c’est-à-dire immotivé, de ce qui ne l’est que relativement. Une partie seulement des signes est absolument arbitraire ; chez d’autres intervient un phénomène qui permet de reconnaître des degrés dans l’arbitraire sans le supprimer : \textit{le signe peut être relativement motivé}.

Ainsi \textit{vingt} est immotivé, mais \textit{dix-neuf} ne l’est pas au même degré, parce qu’il évoque des termes dont il se compose et d’autres qui lui sont associés, par exemple \textit{dix}, \textit{neuf}, \textit{vingt-neuf}, \textit{dix-huit}, \textit{soixante-dix}, etc. ; pris séparément \textit{dix} et \textit{neuf} sont sur le même pied que \textit{vingt}, mais \textit{dix-neuf} présente un cas de motivation relative. Il en est de même de \textit{poirier}, qui rappelle le mot simple \textit{poire} et dont le suffixe \textit{{}-ier} fait penser à \textit{cerisier}, \textit{pommier}, etc. ; pour \textit{frêne}, \textit{chêne}, etc., rien de semblable.~»
\end{quote}

Le fait que les sémantèmes complexes soient en partie motivés n’enlève pas totalement le caractère arbitraire de la combinaison de morphèmes qui est retenue parmi toutes celles possibles. Ainsi pour \textit{machine à laver}, les Québécois disent \textit{laveuse} et les Français disent aussi \textit{lave-linge}, mais à côté de \textit{aspirateur}, nous n’avons ni \textit{machine à aspirer}, ni \textit{aspire-poussière}. On peut aussi noter que les sémantèmes choisis pour désigner un même objet peuvent exprimer des points de vue différents sur cet objet, comme l’illustre les «~synonymes~» \textit{téléski}, \textit{remonte-pente} et \textit{tire-fesse} : le même appareil est présenté successivement comme un appareil qui permet de se déplacer avec ses skis, comme un appareil qui permet de se déplacer vers le haut et comme un appareil qui permet de se déplacer par une traction au niveau des fesses. À l’inverse, des quasi «~paraphrases~» comme \textit{ventilateur} et \textit{moulin à vent} désignent des objets très différents. C’est ce côté \hi{arbitraire du choix} de la combinaison retenue parmi tous les possibles qui manifeste le \hi{caractère figé} des sémantèmes.

La motivation ou transparence doit être distinguée de la diagrammaticité (voir la \sectref{sec:2.2.2} sur la \textit{Décomposition propre des signes}). Il s’agit en quelque sorte du point de vue inverse.

\Definition{\textstyleTermes{transparence}}
{Une combinaison AB est \textstyleTermes{transparente} si les sens de A et B permettent d’avoir une idée du sens de AB.}

La transparence permet donc de dire quelque chose de AB à partir de A et de B. Pour la diagrammaticité, c'est l'inverse : la combinaison AB est \hi{diagrammatique} si A et B commutent suffisamment proprement dans la combinaison AB pour qu’on puisse attribuer des sens à A et B. Bien qu’indépendantes, les deux notions sont liées. La diagrammaticité de \textsc{broyeur} et \textsc{compresseur} permet d’attribuer un sens au suffixe -\textit{eur}. Et comme ce suffixe a acquis un sens, les sémantèmes complexes \textsc{broy+eur} et \textsc{compress+eur} en deviennent transparents. On comprend mieux la différence entre ces notions quand on considère un suffixe rare comme celui de \textsc{royaume}. Ce sémantème est très diagrammatique, car il entre dans un paradigme avec \textit{principauté} ou \textit{duché} et on peut donc voir qu’il est la combinaison de \textit{roi} et \textit{{}-aume} et attribuer un sens au suffixe \textit{{}-aume}, mais il n’est pas très transparent, car le suffixe \textit{{}-aume} n’apparaît pas ailleurs et son sens ne préexiste pas à la connaissance du sens de \textit{royaume}. À l’inverse, une locution comme $⌜$\textsc{lever} \textsc{le} \textsc{coude}$⌝$ ‘boire de l’alcool’ est assez motivée (et donc en partie transparente), puisque pour boire il faut lever le coude, mais elle n’est pas du tout diagrammatique, car la connaissance du sens de la locution ne permet pas d’attribuer des parties de ce sens aux composantes de la locution. On peut encore noter que les locuteurs tendent à produire de la diagrammaticité et à projeter des parties du sens d’une locution sur les syntaxèmes qui la composent. C’est pas exemple ce qu’on observe avec une locution comme $⌜$\textsc{prendre} \textsc{le} \textsc{taureau} \textsc{par} \textsc{les} \textsc{cornes}$⌝$ ‘commencer à résoudre un problème’, où \textsc{taureau} peut s’attribuer le sens ‘problème’ comme dans l’exemple suivant : \textstylest{\textit{La ministre de l’Emploi, Monica De Coninck, a manifestement décidé de prendre} le} taureau de l’emploi\textstylest{ \textit{des quinquas par les cornes} (lesoir.be)}. Cette diagrammatisation peut aller jusqu’à rendre l’expression compositionnelle, comme \textit{poser un lapin} ‘ne pas se rendre à un rendez-vous’, où \textsc{lapin} est aujourd’hui devenu un sémantème indépendant (\textit{C’est le deuxième lapin de la semaine} !) signifiant  ‘rendez-vous auquel on ne s'est pas rendu’.

\section{Syntaxème vide}\label{sec:2.3.3}

\Definition{\textstyleTermes{syntaxème vide}}
{Un \textstyleTermes{syntaxème vide} est un syntaxème sans contribution sémantique, qui n’a pas fait l’objet d’un choix, mais a été imposé par la grammaire de la langue. Un tel syntaxème ne fait donc pas partie d’un sémantème.}

Les grammèmes d’accord en genre présentent un cas intéressant de syntaxèmes vides. Dans le syntagme \textit{la petite fille} /lapətitfij/, l’article et l’adjectif s’accordent en genre avec le nom fille. Ces accords, /-a/ pour l’article et /-ə/ pour l’adjectif (c’est-à-dire l’absence de troncation de la consonne finale de l’adjectif, voir l’\encadref{sec:2.2.25} sur \textit{Syntaxème zéro et troncation en français}), sont des syntaxèmes flexionnels, que nous appelons des \textstyleTermes{grammèmes d’accord}. Ces grammèmes ne sont pas des sémantèmes : ils n’expriment pas un choix et ont uniquement un rôle syntaxique de marquage de la dépendance entre deux mots. En français, il existe deux syntaxèmes flexionnels de genre, le féminin et le masculin. Ils se combinent avec les grammèmes de nombre et sont spécifiques aux adjoints du nom, c’est-à-dire les adjectifs et les déterminants. Le signifié des grammèmes d’accord est une signification grammaticale qui n’est pas porteuse de sens. Les grammèmes d’accord sont des syntaxèmes vides. (Voir néanmoins le cas d’accord entre grammèmes dans l’encadré qui suit.)

Rappelons au passage que le genre est un \hi{{trait catégoriel}} du nom qui contrôle sa combinatoire et la nature des syntaxèmes d’accord en genre sur certains éléments avec lesquels il se combine (les adjectifs, les déterminants et les participes) (voir le \chapfuturef{16} sur les \textit{Catégories nanosyntaxiques}). Certains noms possèdent une alternance de «~sexe~» réalisée par un affixe dérivationnel, qui n’est pas un syntaxème (voir l’\encadref{sec:2.2.17} sur les signes \textit{À la limite entre morphème sous-lexical et syntaxème flexionnel}).

La notion de signe vide terme a été introduite par Lucien Tesnière qui parle de \textit{mot vide} (\citeyear{tesniere1959elements} : Chapitre 28 — Mots pleins et mots vides) :
\begin{quote}
    «~1. — Il y a deux espèces de mots essentiels, les mots \textbf{pleins} et les mots \textbf{vides}.

    2. — Les mots \textbf{pleins} sont ceux qui sont \textbf{chargés d’une} \textbf{fonction sémantique}, c’est-à-dire ceux dont la forme est associée directement à une idée, qu’elle a pour fonction de représenter et d’évoquer. […]

    3. — Les mots \textbf{vides} sont ceux qui ne sont pas chargés d’une fonction sémantique. Ce sont de simples \textbf{outils grammaticaux} ([Note] Damourette et Pichon disent très heureusement des «~struments~», d’autres ont proposé «~mots-charnières~».) dont le rôle est uniquement d’indiquer, de préciser ou de transformer la catégorie des mots pleins et de régler leurs rapports entre eux.~»
\end{quote}

Parmi les syntaxèmes vides, outre les grammèmes d’accord, on considère :

\begin{itemize}
\item les \hi{explétifs} comme le sujet \textsc{il} de \textit{il pleut}, qui n’a pas de contribution sémantique et sert uniquement à remplir la contrainte syntaxique de présence d’un sujet ;
\item les \hi{translatifs purs} (voir la section sur \textit{La translation syntaxique} au \chapfuturef{14}) comme la conjonction de subordination \textsc{que} dans \textit{je sais que tu viens} : ce syntaxème sert uniquement à permettre à un verbe d’occuper une position syntaxique où un substantif est attendu ; les pronoms relatifs, comme le \textsc{que} de \textit{le livre que je lis}, sont aussi d’une certaine façon des syntaxèmes vides, comme on le verra dans le \chapfuturef{20} sur l’extraction ;
\item les syntaxèmes marquant les \hi{régimes}, comme le À de \textit{je parle à quelqu’un}, sont aussi généralement considérés comme des syntaxèmes vides, même si ce cas est plus discutable (voir l’encadré qui suit).
\end{itemize}

\loupe[sec:2.3.4]{Constructions verbales et accords : signes vides ?}{%
    Dans \textit{Pierre parle à Marie}, nous considérons que la préposition \textsc{à} n’est pas une unité sémantique, mais qu’elle fait partie du \hi{régime} imposé par \textsc{parler}. Autrement dit, la préposition \textsc{à} ne constitue pas un choix séparé de celui de \textsc{parler~}: plus exactement, le choix fait ici est celui d’une acception particulière de \textsc{parler} qui se construit avec un complément d’objet indirect introduit par \textsc{à}, le verbe et sa construction formant un tout indissociable.

    La version extrême de cette position est de considérer que \textsc{à} est un signe vide, c’est-à-dire sans contribution sémantique (voir la section précédente). La position inverse est au contraire de considérer que le \textsc{parler} de \textit{Pierre parle à Marie} est le même que le \textsc{parler} de \textit{Pierre parle en dormant} ou \textit{Pierre parle anglais} et que la nuance de sens qu’il y a entre ces trois occurrences de \textsc{parler} est due à la \hi{construction} avec laquelle \textsc{parler} se combine. Autrement dit, lorsque \textit{Pierre parle à Marie}, le sens sous-jacent qui est que ‘Pierre s’adresse à Marie’ vient davantage de la construction «~X V \textit{à} Y~», dans laquelle \textsc{s’adresser}, \textsc{parler}, \textsc{téléphoner} ou \textsc{écrire} peuvent occuper la position V, que du verbe qui occupe cette position. Bien sûr cette construction signifiant en elle-même ‘s’adresser à quelqu’un’ n’est compatible qu’avec des verbes du type \textsc{s’adresser} et il est donc difficile de savoir d’où vient exactement la contribution sémantique. En admettant que le lexème \textsc{parler} se combine avec une construction comme il se combine avec sa flexion, il reste que la combinaison d’un lexème et d’une construction est une \hi{combinaison liée~}: par exemple, si on reprend les trois verbes \textsc{écrire}, \textsc{parler} et \textsc{téléphoner} qui se combinent tous avec «~X V à Y~», on voit qu’on peut dire \textit{parler anglais}, mais pas *\textit{téléphoner anglais}, ni *\textit{écrire anglais} (on dit \textit{écrire en anglais}). On peut dire \textit{écrire quelque chose à quelqu’un}, mais pas \textit{parler quelque chose à quelqu’un}. Par contre, on peut \textit{parler de quelque chose}, mais pas *\textit{écrire de quelque chose} ou *\textit{téléphoner de quelque chose}. Donc les V qui se combinent avec une construction donnée ne commutent pas librement avec d’autres V.

    En conclusion, il peut être envisageable de considérer la construction du verbe comme un signe à part entière. On peut même considérer que cette construction est elle-même la combinaison de plusieurs signes, chacun correspondant à la relation entre le verbe et un de ses actants (voir le \chapfuturef{18} sur les \textit{Relations syntaxiques}). La préposition ne serait donc pas ici un mot vide, mais le signifiant d’un signe indiquant une relation de destinataire entre l’action décrite par le verbe et son dépendant : dans \textit{Pierre parle à Marie}, Pierre parle — il profère des paroles — et Marie en est le destinataire. Néanmoins la combinaison entre le verbe et sa construction est une combinaison liée : le paradigme des lexèmes appelant une construction donnée n’est pas régulier et, inversement, pour un lexème donné, les constructions dans lesquels il entre est restreint et peu prévisible. (Voir la \sectref{sec:2.3.10} \textit{Collocation et choix liés} pour les combinaisons liées de deux sémantèmes.)

    Nous avons vu que les syntaxèmes d’accord en genre sont des syntaxèmes vides. Un exemple comme \textit{Les animaux boivent} /\textstylePhonoApprofondissement{l\textbf{ez}anim\textbf{o}bwa\textbf{v}}/, opposé à \textit{L’animal boit} /\textstylePhonoApprofondissement{lanimalbwa}/, est plus problématique. Il y a bien ici une unité sémantique de pluriel, mais il y a trois syntaxèmes exprimant le pluriel /\textstylePhonoApprofondissement{{}-ez}/, /-\textstylePhonoApprofondissement{o}/ et~/-\textstylePhonoApprofondissement{ə}/, associés respectivement à l’article, au nom et au verbe. Un linguiste comme Martinet considère qu’il s’agit d’un sémantème possédant un \hi{signifiant discontinu}, tandis qu’un linguiste comme Mel’čuk, considère 1)~que l’un des trois syntaxèmes est un sémantème et que les deux autres sont des \hi{syntaxèmes d’accord}, et 2) que c’est le nom qui se combine avec le sémantème. Cette dernière proposition est motivée par le fait que le sens du sémantème de pluriel porte effectivement sur le nom et permet de décider si le nom désigne un seul référent ou plusieurs. Par ailleurs, il existe des noms qui requièrent essentiellement le singulier, comme \textsc{eau}, \textsc{chocolat} ou \textsc{peur}, et d’autres toujours le pluriel, comme \textsc{rillettes} ou \textsc{fiançailles}, et on voit dans de tels cas que le nom forme un phrasème (voir définition à la \sectref{sec:2.3.7} qui suit) avec le nombre et que c’est lui qui va déclencher l’accord des autres éléments lexicaux. Mais l’on pourrait rétorquer que la plupart des déterminants portent intrinsèquement un nombre. Une troisième proposition serait donc de considérer que c’est le déterminant qui porte réellement le nombre soit dans son sens lexical, comme pour \textsc{deux}, soit sous la forme d’un grammème de nombre qui se combine avec le déterminant proprement dit, comme pour l’article. Cette option est renforcée par le fait que le nombre n’est généralement marqué à l’oral que sur l’article (\textit{Les vaches broutent} /\textstylePhonoApprofondissement{l\textbf{e}vaʃbrut}/, opposé à \textit{La vache broute} /\textstylePhonoApprofondissement{l\textbf{a}vaʃbrut}/). Pour conclure, nous dirons que les trois analyses s’accordent de toute façon sur le fait qu’il y a un sémantème et plusieurs syntaxèmes. Reste la question de savoir si ce sémantème correspond à l’un de ces syntaxèmes ou à leur combinaison, question dont il n’est pas sure qu’elle soit pertinente.
}
\section{Décomposition en sémantèmes}\label{sec:2.3.5}

La première étape dans l’analyse sémantique d’un énoncé est d’en repérer les sémantèmes. Considérons l’énoncé :

\ea\itshape La moutarde me monte au nez.\z

La production de cet énoncé, dont le sens est à peu près ‘je sens la colère monter en moi’, suppose quatre choix. Cet énoncé contient donc quatre sémantèmes. Nous incitons le lecteur à arrêter un instant sa lecture et à chercher quelles sont ces quatre unités sémantiques.

La première est \textit{me}, choisie par opposition à \textit{te, lui}, etc. Plus exactement, il s’agit du lexème \textsc{moi}, la forme clitique \textit{me} étant imposée par le prédicat verbal qui réalise son premier actant comme objet indirect. Dans une paraphrase comme \textit{Je commence à être en colère}, \textsc{moi} sera réalisé par la forme sujet \textit{je}. La deuxième unité sémantique est la locution \textit{la moutarde mont- au nez}, que nous notons $⌜$\textsc{la} \textsc{moutarde} \textsc{monter} \textsc{au} \textsc{nez}$⌝$, signifiant ‘sentir la colère monter en soi’. Comme dans le cas de \textit{pomme de terre}, aucun des syntaxèmes qui composent cette locution ne peut être commuté individuellement : \#\textit{Le wasabi me monte au nez}, \#\textit{La moutarde me grimpe au nez}, \#\textit{La moutarde me monte aux yeux}. Aucune de ces phrases n’évoque la colère ou alors par un jeu de mots qui permet au locuteur de faire le lien avec \textit{La moutarde me monte au nez}. Le choix de \textit{la moutarde mont- au nez} est un choix unique et indivisible. La troisième unité sémantique est le présent, réalisé par \textit{{}-e} (un syntaxème zéro à l’oral) et choisi par opposition à l’imparfait ou au futur (\textit{La moutarde me montait au nez}, \textit{La moutarde me montera au nez}). La quatrième unité sémantique est non segmentale : c’est la déclarativité. Notre énoncé s’oppose ainsi à \textit{La moutarde me monte-t-elle au nez} ? La déclarativité s’exprime ici par l’utilisation d’une proposition avec un verbe à l’indicatif et une courbe intonative descendante en fin de phrase, typique des énoncés déclaratifs (transcrit à l’écrit par l’utilisation du point final «~.~» par opposition à « ?~» ou « !~»).

Nous étudierons la façon dont les sémantèmes se combinent dans le \chapref{sec:13} sur la \textit{Structure syntaxique profonde}. On peut aussi consulter le \chapref{sec:1.2} où nous avons montré comment \textit{Produire un énoncé} à partir d’une représentation sémantique.

\loupe[sec:2.3.6]{Pronoms}{%
    Les \textstyleTermesapprofondissement{pronoms} sont un autre type d’unités que l’on tend à opposer aux lexèmes pleins (voir la \sectref{sec:2.3.3} sur le \textit{Syntaxème vide}). La plupart des pronoms sont bien des unités ayant une réelle contribution sémantique, mais ils contrastent effectivement avec des sémantèmes comme \textsc{arbre,} \textsc{voir,} \textsc{rouge} ou \textsc{dans.} La plus grande partie des sémantèmes ont un \textstyleTermesapprofondissement{sens dénotationnel}~: ils dénotent certaines entités ou situations du monde, c’est-à-dire que leur sens sélectionne intrinsèquement une partie des entités ou situations envisageables dans le ou les mondes imaginaires que nous construisons avec nos discours. On peut ainsi parmi les entités du monde identifier un sous-ensemble d’entités que l’on peut dénommer des \textit{arbres} ou que l’on peut considérer comme \textit{rouge}. De la même façon, parmi les situations mettant en jeu deux entités X et Y, on peut distinguer celles où X \textit{voit} Y ou celles ou X est \textit{dans} Y et cette décision est indépendante de la situation d’énonciation, c’est-à-dire de qui parle à qui, où et quand.

    Cette «~dénotationalité~» contraste avec la propriété des pronoms, comme \textsc{celui-ci,} \textsc{l’autre,} \textsc{ca,} \textsc{moi} ou \textsc{ici}. Les pronoms sont des unités qui permettent également de sélectionner une partie des entités ou des situations envisageables, mais cette sélection dépend complètement de la situation d’énonciation. Il n’y a pas d’entité du monde dont on puisse dire qu’elle est intrinsèquement \textit{l’autre, ça, ici} ou \textit{moi~}: ‘moi’ est celui qui énonce (\textit{Regarde-}\textbf{\textit{moi~}}!), ‘ici’ est le lieu où a lieu l’énonciation (\textbf{\textit{Ici}}, \textit{il fait plus chaud qu’à Paris}.), ‘ça’ est quelque chose que l’on montre (\textit{Prends} \textbf{\textit{ça~}}!) ou dont on a parlé (\textbf{\textit{Ça}} \textit{m’a plu.}) et ‘l’autre’ se définit par rapport à un ‘celui-là’ déjà identifié (\textit{Je vais plutôt prendre} \textbf{\textit{l’autre}}.).

    On distingue deux fonctions pour les pronoms. Un pronom est dit \textstyleTermesapprofondissement{déi\-ctique} si la ou les entités auxquelles il réfère peuvent être déterminées par la situation d’énonciation, indépendamment du discours qui entoure l’énonciation du pronom : \textsc{moi,} \textsc{toi,} \textsc{ici} ou \textsc{demain} sont typiquement des déictiques ou \textsc{celui-ci} quand il s’agit d’une chose que l’on désigne de la main. Un pronom est dit \textstyleTermesapprofondissement{anaphorique} si la ou les entités auxquelles il réfère sont déterminées par le renvoi à une portion antérieure du discours, que l’on appelle l’\textstyleTermesapprofondissement{antécédent}\textstyleTermes{} du pronom. Dans les exemples suivants, le pronom anaphorique est en gras et son antécédent souligné:
\ea
  \ea \itshape Pierre a pris \uline{un livre}. \textbf{Il}  était sur la table.
  \ex \itshape Pierre a pris \uline{le livre} \textbf{qui}  était sur la table.
  \ex \itshape \uline{Pierre est allé en vacances à la montagne.} \textbf{Ça} lui a plu.
  \z
\z
    On ajoute encore dans la catégorie des pronoms d’autres éléments qui n’ont pas de sens dénotationnel : les pronoms interrogatifs comme \textit{qui} (\textbf{\textit{Qui}} \textit{a parlé à Marie} ? \textit{Pierre, je crois.)} ou les pronoms négatifs comme \textit{rien} (\textit{Il n’a} \textbf{\textit{rien}} \textit{vu}).\textbf{ }

    Les fonctions anaphorique et déictique ne sont pas la spécificité des pronoms :
\ea
  \ea \itshape Prends \textbf{ce livre} !
  \ex \itshape Pierre a pris \uline{un livre}. \textbf{Ce livre}  était sur la table.
  \ex \itshape \uline{Pierre a fait le tour du monde.} \textbf{Le voyage}  lui a plu.
  \z
\z
    Les pronoms partagent beaucoup de propriétés avec les déterminants et un même lexème possède souvent une acception qui est déterminant (\textbf{\textit{Plusieurs personnes}} \textit{sont venues}) et une autre qui est pronom~(\textbf{\textit{Plusieurs}} \textit{sont venus}). La particularité des pronoms est de pouvoir référer sans s’appuyer sur un nom, contrairement aux déterminants qui doivent se combiner avec un nom. (Voir aussi la possibilité de traiter le pronom et le déterminant comme un même sémantème à la \sectref{sec:3.3.26} \textit{Déterminant comme tête} ?).

    Pour les éléments qui ont spécifiquement une valeur anaphorique (ou déictique) mais qui ne reprennent pas un nom, on parle de \textstyleTermesapprofondissement{proforme}. Les verbes peuvent avoir des proformes, mais cela reste marginal dans la diversité des langues ; citons le verbe modal \textsc{do} pour l’anglais :
\ea 
  \ea \itshape Do you plan to come with us?\\
  \glt   ‘Prévois-tu de venir avec nous ?’
  \ex \itshape Yes, I \textbf{do}.\\
  \glt   ‘Oui, je le prévois’, litt. ‘Oui, je fais’.
  \z
\z
}
\section{Phrasème}\label{sec:2.3.7}

Comme nous l’avons vu pour \textit{broyeur} ou \textit{machine à laver}, il est assez courant qu’un sémantème soit réalisé par une combinaison de morphèmes. Il s’agit même de la majorité des sémantèmes.

\Definition{\textstyleTermes{sémantème complexe}}
{Un sémantème composé de plusieurs morphèmes est appelé un \textstyleTermes{sémantème complexe}.}

Un sémantème commute généralement librement avec son contexte (voir le cas particulier des collocations dans la \sectref{sec:2.3.10} \textit{Collocation et choix lié}). Il s’agit donc soit d’un syntaxème, soit d’une combinaison de syntaxèmes.

\Definition{\textstyleTermes{phrasème}, \textstyleTermes{locution}, \textstyleTermes{expression figée ou idiomatique}}
{Un sémantème qui est la combinaison de plusieurs syntaxèmes est appelé un \textstyleTermes{phrasème}. Les phrasèmes sont encore appelés des \textstyleTermes{locutions} ou des \textstyleTermes{expressions figées} dans la tradition grammaticale française et des \textstyleTermes{expressions idiomatiques} (angl. \textit{idioms}) dans la tradition anglo-saxonne. Les délimiteurs $⌜$…$⌝$ servent à indiquer qu’une combinaison de syntaxèmes forme un phrasème.}


On distingue parmi les sémantèmes complexes ceux qui sont des syntaxèmes, comme les lexèmes complexes \textsc{blessure,} \textsc{aspirateur} ou \textsc{construction,} et ceux qui sont des combinaisons de syntaxèmes, comme les phrasèmes $⌜$\textsc{se} \textsc{retourner} \textsc{dans} \textsc{sa} \textsc{tombe}$⌝$ ou $⌜$\textsc{tiré} \textsc{par} \textsc{les} \textsc{cheveux}$⌝$. Les signes qui composent ces deux types de sémantèmes complexes sont de nature différente. Les signes lexicaux qui apparaissent dans un lexème complexe, comme \textit{bless-} dans \textsc{blessure} ou \textit{aspir}{}- dans \textsc{aspirateur}, sont des lexèmes qui ont perdu leur combinatoire (\textit{bless-} et \textit{aspir-} ne se comportent plus comme les verbes \textsc{bless}(\textsc{er}) et \textsc{aspir}(\textsc{er}) dans ce contexte), mais ont en grande partie conservé leur sens : ce sont des \textstyleTermes{lexèmes désyntactisés}. À l’inverse, les signes lexicaux qui apparaissent dans les phrasèmes conservent leur combinatoire syntaxique, mais sont dépossédés de leur contribution sémantique usuelle : ce sont des \textstyleTermes{lexèmes désémantisés}.

Les phrasèmes sont par définition des sémantèmes et font donc l’objet d’un choix indivisible. Il n’y a donc pas de paradigme de choix sur les syntaxèmes qui les composent. C’est ce qu’on nomme le \textstyleTermes{figement} ou la \textstyleTermes{lexicalisation} de la combinaison. Une expression figée a \hi{perdu sa compositionalité}.

Certains phrasèmes sont \textstyleTermes{polymorphes~}: c’est bien sûr le cas lorsque les morphèmes sont eux-mêmes polymorphes comme $⌜$\textsc{s’en} \textsc{aller}$⌝$, où le réfléchi (noté \textsc{se}) comme \textsc{aller} ont de nombreux allomorphes~\textit{: je} \textbf{\textit{m’en} \textit{v}}\textit{ais, nous} \textbf{\textit{nous en i}}\textit{rons}, etc. Mais il y a aussi des phrasèmes qui ne s’expriment pas de manière univoque par une seule combinaison de lexèmes : c’est le cas par exemple de $⌜$\textsc{chier/faire} \textsc{dans} \textsc{son} \textsc{froc/pantalon}$⌝$ ‘manifester une peur intense’ avec deux alternances possibles et donc quatre formes en tout.

\section{Lexie et grammie}\label{sec:2.3.8}

De même que parmi les syntaxèmes nous avons distingué les lexèmes et les grammèmes, nous distinguons parmi les sémantèmes les lexies et les grammies.

\Definition{\textstyleTermes{unité lexicale}, \textstyleTermes{lexie}}
{Un sémantème qui commute avec des lexèmes, comme $⌜$\textsc{se} \textsc{retourner} \textsc{dans} \textsc{sa} \textsc{tombe}$⌝$, est appelé une \textstyleTermes{unité lexicale} ou \textstyleTermes{lexie}.}

Nous distinguons très nettement les termes \textit{lexème} et \textit{lexie}. En général, un lexème possède plusieurs acceptions qui constituent autant de lexies (voir la \sectref{sec:2.3.13} sur la \textit{Délimitation paradigmatique des sémantèmes}). Un lexème peut aussi être partie d’un phrasème et ce sont alors les acceptions de ce phrasème qui constituent des lexies. La lexie est ainsi l’unité de nature lexicale minimale du point de vue de son signifié, tandis que le lexème est l’unité de nature lexicale minimale du point de vue de son syntactique.

\Definition{\textstyleTermes{unité grammaticale}, \textstyleTermes{grammie}}
{Nous appellerons \textstyleTermes{unité grammaticale} ou \textstyleTermes{grammie} tout sémantème qui commute avec des syntaxèmes flexionnels ou grammèmes.}

Il peut s’agir d’acceptions de syntaxèmes flexionnels, mais aussi de combinaisons de grammèmes comme le conditionnel  $⌜${}-\textit{r}{}- + -\textit{i}{}-$⌝$ (dans \textit{Nous aimer}\textbf{\textit{io}}\textit{ns vous rencontrer}) ou de combinaison d’un grammème et d’un lexème comme le passé composé $⌜$\textsc{avoir} + participe\_passé$⌝$.

Certaines lexies ont plutôt un rôle structurel et ne commute pas vraiment avec des lexèmes ou des grammèmes, comme le «~cliveur~» $⌜$\textsc{c’est} \textsc{…} \textsc{qu-}$⌝$ (\textbf{\textit{C’est}} \textit{Pierre} \textbf{\textit{qui}} \textit{viendra}\textbf{, \textit{C’est} à Pierre} \textbf{\textit{que}} \textit{je l’ai dit}) ou le présentatif $⌜$\textsc{il} \textsc{y} \textsc{a} \textsc{…} \textsc{qui}$⌝$ (\textbf{\textit{Il y a}} \textit{quelqu’un} \textbf{\textit{qui}} \textit{a sonné,} \textbf{\textit{Il y a}} \textit{ma voiture} \textbf{\textit{qui}} \textit{est en panne}). Ces lexies contrôlent la construction de la proposition et on leur donne parfois le nom de \textit{constructions de phrase}. Nous les appellerons des \textstyleTermes{sémantèmes constructionnels}.

\loupe[sec:2.3.9]{Verbes supports et unités grammaticales}{%
    Les grammies marquant des rôles syntaxiques sont souvent en concurrences avec des \textstyleTermesapprofondissement{constructions à verbes supports}. On appelle \textstyleTermesapprofondissement{verbe support} un verbe qui n’a \hi{pas de contribution sémantique propre} et joue essentiellement un rôle syntaxique en permettant à un nom prédicatif d’occuper une position verbale. (Les verbes supports sont appelés \textit{light verbs} ‘verbes légers’ dans la littérature anglo-saxonne.)

    Par exemple, pour exprimer le sens ‘gifler’, on a le choix entre le verbe \textsc{gifler} et une construction à verbe support avec le nom \textsc{gifle}. Dans les deux cas, trois diathèses sont possibles :
\ea
\begin{tabularx}{\linewidth}[t]{@{}llQQ@{}}
a. & actif  &       \textit{{Marie gifle Pierre}}     &  \textit{{Marie} \textbf{{donne}}  {une gifle à Pierre}}\\
b. & passif avec agent & \textit{{Marie} \textbf{{est}}  {gifl}\textbf{{ée}}} \linebreak \textit{par Pierre}   &  \textit{{Marie} \textbf{{reçoit}}  {une gifle de Pierre}}\\
c. & passif sans agent & \textit{{Marie} \textbf{{se fait}}  {gifl}\textbf{{er}}}  &   \textit{{Marie} \textbf{{se}} \textbf{{prend}}  {une gifle}}
\end{tabularx}
\z
    Le verbe support n’est pas sémantiquement vide : dans l’action de gifler, il y a bien un donneur et un receveur et les verbes \textsc{donner,} \textsc{recevoir} ou \textsc{prendre} sont sémantiquement motivés et ce sont des sémantèmes. Néanmoins la contribution des verbes supports consiste essentiellement à attribuer les rôles sémantiques aux différents participants comme le font les \textstyleTermesapprof{voix} (voir définition dans le \chapref{sec:13}) : l’actif, le passif (\textsc{être} + participe\_passé) et le passif sans agent (\textsc{se} \textsc{faire} + infinitif). C’est ce qui nous autorise à dire que les verbes supports n’ont pas de contribution sémantique propre.

    Les verbes supports sont un cas particulier de \hi{translatif} (voir la section sur \textit{La translation syntaxique} du \chapfuturef{17}) et un cas particulier de \hi{collocatif} (voir la \sectref{sec:2.3.10} qui suit).
}
\section{Collocation et choix liés}\label{sec:2.3.10}

Si l’unité sémantique AB est une \hi{combinaison libre} A${\oplus}$B, cela signifie que A et B sont nécessairement des unités sémantiques choisies indépendamment l’une de l’autre, dans des paradigmes de choix indépendants, et donc que l’unité AB est compositionnelle. Si l’unité sémantique AB est une \hi{combinaison liée} A~+~B, il est possible que A et B soient tout de même des unités sémantiques, c’est-à-dire que A et B résultent de \hi{choix distincts}. Néanmoins, dans ce cas, comme nous allons le voir, les choix de A et B ne peuvent pas être totalement indépendants.

\Definition{\textstyleTermes{collocation, semi-phrasème}}
{Une \textstyleTermes{collocation} ou \textstyleTermes{semi-phrasème} est une combinaison non libre de deux unités sémantiques.}

Dans l’énoncé \textit{Pierre a peur}, le choix de \textsc{peur} s’oppose à ceux de \textsc{faim,} \textsc{froid} ou \textsc{envie}, tandis que le choix de \textsc{avoir} s’oppose à ceux de \textsc{prendre} ou \textsc{faire}. Le principe de commutation s’applique et la commutation est propre. On a bien deux unités sémantiques. Par contre, \textsc{peur} ne peut pas commuter dans cet énoncé avec ses synonymes \textsc{crainte} ou \textsc{effroi} et à la place de *\textit{avoir crainte} ou de *\textit{avoir effroi}, on dira plutôt \textit{craindre} ou \textit{être effrayé}. De même, le verbe \textsc{avoir} ne peut pas commuter ici avec un autre verbe que \textsc{prendre} ou \textsc{faire}. La combinaison n’est donc pas libre.

\Definition{\textstyleTermes{semi-figement}}
{Dans une collocation, les deux choix ne peuvent être libres et indépendants, sinon la combinaison serait nécessairement libre. Il s’agit de choix liés. C’est pourquoi une collocation est dite \textstyleTermes{semi-figée}.}

Plus exactement, une collocation est toujours \hi{asymétrique~}: l’un des choix est dépendant de l’autre. Par exemple, dans \textsc{avoir} \textsc{peur}, c’est \textsc{peur} qui est \hi{choisi librement}, tandis qu’\textsc{avoir} est choisi pour permettre a \textsc{peur} de jouer un rôle verbal (c’est un verbe support ; voir l’\encadref{sec:2.3.9} qui précède) et ce \hi{choix} est \hi{dépendant} de \textsc{peur}, dans le sens où un autre nom de sentiment entraînerait un choix différent.

\Definition{\textstyleTermes{base}, \textstyleTermes{collocatif}}
{Le sémantème qui est choisi librement est appelé la \textstyleTermes{base} de la collocation. Le sémantème dont le choix dépend de la base est appelé un \textstyleTermes{collocatif} de cette base. On reconnaît un collocatif au fait qu’il ne peut pas se combiner avec certains sémantèmes qui peuvent commuter avec sa base dans d’autres contextes et qui sont pourtant sémantiquement proches.}

Le \tabref{tab:2-3:1} met en parallèle les choix contraints par \textsc{peur} et \textsc{colère} pour l’expression d’un certain nombre de sens très généraux comme ‘causer’, ‘commencer’ ou ‘intense’. (Une case vide indique l’absence de collocatif pour exprimer ce sens avec cette lexie.)

\begin{table}
\caption{Collocatifs pour \textsc{peur} et \textsc{colère}\label{tab:2-3:1}}
\begin{tabularx}{\textwidth}{QQQ}
\lsptoprule
 & {\scshape peur} & {\scshape colère}\\
 \midrule
‘éprouver’ & {\itshape avoir peur} & {\itshape être en colère}\\
‘commencer à éprouver’ & {\itshape prendre peur} & {\itshape se mettre en colère}\\
‘causer’ & {\itshape faire peur à qqn} & {\itshape mettre qqn en colère} {\itshape provoquer la colère de qqn}\\
‘manifester (par un symptôme)’ & \textit{trembler de peur}\\
$⌜$\textit{faire dans son froc}$⌝$ {\itshape être paralysé (par la peur)} & {\itshape suffoquer de colère bouillir (de colère)}\\
‘éprouver intensément’ & {\itshape avoir une peur bleue être vert/mort de peur} & {\itshape être dans une colère noire être rouge de colère}\\
‘éprouver faiblement’ & {\itshape avoir une petite peur} & \\
‘décharger’ & {\itshape se libérer de sa peur} & {\itshape passer sa colère sur qqn}\\
‘qui ne se manifeste pas’ &  & {\itshape colère sourde, rentrée}\\
‘qui n’est pas contrôlé’ & {\itshape peur panique} & {\itshape colère aveugle}\\
\lspbottomrule
\end{tabularx}
\end{table}

On peut constater que les choix possibles sont très différents. On aura noté en particulier le choix totalement arbitraire des adjectifs de couleur, \textit{bleu} et \textit{vert} pour la peur, \textit{rouge} et \textit{noir} pour la colère. Et si on peut avoir \textit{une peur bleue}, on ne peut être \textit{\textsuperscript{\#}}\textit{bleu de peur}. On doit être \textit{vert de peur}, mais on n’aura pas \textit{\textsuperscript{\#}}\textit{une peur verte}. Le fait qu’aucun de ces sémantèmes ne peut se combiner à la fois avec \textsc{peur} et \textsc{colère} montre bien qu’il s’agit de collocatifs.

Si la collocation partage avec le phrasème le caractère arbitraire de son signifiant, une collocation A~+~B se distingue d’un phrasème par la possibilité de commutation sur le collocatif (comme on l’a vu ci-dessus pour \textit{avoir peur}) et par la possibilité de modifier indépendamment A et B (cf. \textbf{\textit{ne pas avoir}} \textit{peur} vs \textit{avoir} \textbf{\textit{une belle peur}}). La transparence sémantique ne permet pas par contre de les distinguer. Comparons par exemple les expressions \textit{noyer le poisson} et \textit{avoir les boules} : à première vue, aucune des deux expressions n’est très transparente, si l’on s’en tient au sens usuel des différents lexèmes qu’elles contiennent. Pourtant, alors que la première ne permet aucune commutation propre, la deuxième permet des commutations propres sur \textsc{avoir}: \textit{foutre les boules à qqn}, \textit{C’est les boules} !, \textit{Putain, les boules} !. Comme par ailleurs, \textit{les boules} ne permet pas de commutation de \textit{les} dans ce contexte (\textit{\textsuperscript{\#}}\textit{avoir la boule}, \textsuperscript{\#}\textit{avoir des boules}), on en déduit que $⌜$\textsc{les} \textsc{boules}$⌝$ est un sémantème et que \textsc{avoir} est un collocatif (un verbe support pour être plus précis). Au final, \textit{avoir les boules} apparaît comme une semi-phrasème, tandis que $⌜$\textsc{noyer} \textsc{le} \textsc{poisson}$⌝$ est vraiment un phrasème.

Nous avons évoqué, dans l’\encadref{sec:2.3.4} \textit{Constructions verbales et accords : signes vides} ?, le cas des régimes qui forment aussi des combinaisons liées. Nous distinguons néanmoins le cas des régimes de celui des collocations discuté ici car les régimes constituent des «~choix~» obligatoires (un lexème ne s’utilise pas sans une construction particulière), alors que le choix d’un collocatif reste toujours optionnel et constitue en cela un véritable choix.

\loupe[sec:2.3.11]{Fonctions lexicales}{%
    Le collocatif étant \hi{choisi en fonction} de la base, on peut modéliser les collocations par des \hi{fonctions}, au sens mathématique du terme, appelées \textstyleTermesapprofondissement{fonctions lexicales}\textstyleTermes{} par Igor Mel’čuk qui les a découvertes à la fin des années 1950 lorsqu’il participait aux premières recherches en traduction automatique en Union Soviétique. À chaque sens qui s’exprime par des collocatifs, on fait correspondre une fonction lexicale qui associe à chaque unité lexicale les collocatifs exprimant ce sens. Le tableau de la section qui précède en est un bon exemple : au sens ‘qui n’est pas contrôlée’ correspond par exemple une fonction lexicale qui associe \textit{panique} à \textsc{peur} et \textit{aveugle} à \textsc{colère}.

    On peut voir une fonction lexicale comme une sorte d’unité lexicale dont le signifiant varierait en fonction du contexte, une sorte de super-signème ; on aurait ainsi une \hi{unité lexicale abstraite} signifiant ‘qui n’est pas contrôlé’ et qui prendrait la valeur \textit{panique} dans un certain contexte, \textit{aveugle} dans un autre ou encore \textit{innocente} en combinaison avec \textsc{joie}.

    Les collocations n’étant pas prédictibles, elles doivent être listées une à une dans le modèle d’une langue. La liste des collocatifs d’un sémantème s’appelle sa \textstyleTermesapprofondissement{combinatoire lexicale restreinte}. La collocation (la co-location en tant que phénomène de semi-figement) s’appelle la \textstyleTermesapprofondissement{cooccurrence lexicale restreinte,} terme exprimant le fait que la cooccurrence de deux unités lexicales est soumise à des \hi{restrictions de sélection}.

    La combinatoire lexicale restreinte d’une lexie fait partie de son \hi{syntactique}, les autres éléments du syntactique contrôlant sa combinatoire libre et son régime (voir la \sectref{sec:2.1.3} sur \textit{Signifé, signifiant, syntactique}). Les collocations constituent ainsi une part très importante de la description du lexique d’une langue. Les dictionnaires monolingues ne font pas une description systématique des collocatifs. Les dictionnaires bilingues en listent généralement davantage du fait qu’un collocatif ne peut pas être traduit indépendamment de sa base. Igor Mel’čuk et son ancien étudiant Alain Polguère ont développé d’importantes bases lexicales pour le français où les collocatifs de chaque entrée lexicale sont décrits à l’aide de fonctions lexicales.
}
\loupe[sec:2.3.12]{Collocations morphologiques}{%
    Nous avons considéré que les combinaisons propres mais liées étaient des collocations dès qu’elles mettaient en jeu des unités syntaxiques. Ces combinaisons comportent deux unités sémantiques : la base, choisie librement, et le collocatif, dont le choix est contraint par celui de la base.

    On est en droit de se demander si les combinaisons propres au sein d’un syntaxème complexe ne sont pas aussi des collocations. Pourquoi des combinaisons très diagrammatiques comme \textit{rapide+ment}, \textit{défend+able}, \textit{guitar+iste} ou \textit{fragil}+\textit{is}(\textit{er}) ne feraient pas aussi l’objet de deux choix : le choix, libre, du radical, puis le choix contraint par le radical du suffixe adéquat.

    C’est fort probable que dans certains cas les choses se passent effectivement comme cela. Des erreurs de production, comme le fameux \textit{bravitude} prononcé par Ségolène Royal alors candidate à la présidence de la république française, tend à prouver que les dérivés de ce type sont parfois construit au moment de la production, avec les écarts à la norme que cela peut entraîner.
}
\section{Délimitation paradigmatique des sémantèmes}\label{sec:2.3.13}

Nous sommes intéressés jusque-là au découpage de la chaîne parlée en sémantèmes, c’est-à-dire à la délimitation des sémantèmes selon l’axe syntagmatique. Intéressons-nous maintenant au découpage selon l’axe paradigmatique (voir la \sectref{sec:2.2.9} sur le \textit{Signème}). Il s’agit de savoir quand deux occurrences d’un même segment de textes appartiennent ou non au même sémantème.

La délimitation des sémantèmes sur l’axe paradigmatique est une question qui dépasse largement le cadre de notre ouvrage et qui est relativement indépendante des questions qui nous intéressent. Nous allons simplement montrer que les principes que nous avons introduits pour la délimitation syntagmatique des unités (principe de commutation, paradigme de choix) permettent aussi une décomposition paradigmatique, ce qui revient à distinguer différentes acceptions d’une même forme.

Nous considérons que deux occurrences de la même forme appartiennent au même sémantème si on peut leur attribuer un \hi{même sens}, si elles appartiennent au \hi{même paradigme de choix} (c’est-à-dire que les éléments qui peuvent commuter avec eux soient les mêmes) et si elles ont une \hi{combinatoire similaire} (et notamment si elles ont les mêmes collocatifs).

Prenons l’exemple du lexème \textsc{bureau}. Il possède au moins quatre acceptions bien séparées (que nous numéroterons de 1 à 4 pour les différencier) :

\begin{itemize}
\item Dans l’énoncé \textit{Le livre est sur le bureau}, le choix de \textsc{bureau1} s’oppose à ceux de \textsc{table} ou \textsc{chaise} ; \textsc{bureau1} dénote un artefact et plus précisément un meuble.
\item Dans l’énoncé \textit{Pierre est dans le bureau}, le choix de \textsc{bureau2} s’oppose à ceux de \textsc{chambre} ou \textsc{cuisine} ; \textsc{bureau2} dénote un lieu et plus précisément une pièce d’habitation.
\item Dans l’énoncé \textit{Pierre est au bureau}, le choix de \textsc{bureau3} s’oppose à ceux de \textsc{maison,} \textsc{école} ou \textsc{toilettes} ; \textsc{bureau3} dénote une activité (et se construit avec le verbe support \textsc{être}).
\item Dans l’énoncé \textit{Le bureau s’est rassemblé à nouveau}, le choix de \textsc{bureau4} s’oppose à ceux de \textsc{équipe,} \textsc{direction} ou \textsc{staff} ; \textsc{bureau4} dénote un groupe de personnes.
\end{itemize}

Comme on le voit, les différents acceptions de \textsc{bureau} appartiennent à des classes sémantiques assez différentes, même s’il s’agit quand même en un certain sens de la même unité \textsc{bureau} dont les différents sens sont liés : un bureau2 contient un bureau1, l’activité bureau3 s’exerce dans un bureau2 et un bureau4 est un groupe de personnes qui se rassemblent dans un bureau2. Chaque acception possède sa combinatoire propre et nous les avons d’ailleurs présentées dans des environnements qui permettent de les discriminer : \textit{être sur, être dans, être à, se rassembler}.

Un autre exemple est fourni par les cinq acceptions suivantes de \textsc{tourner~}:

\begin{itemize}
\item \textsc{tourner1} est lié à \textsc{tour} (\textit{tourner}1 \textit{autour de X} {\textasciitilde} \textit{faire le tour de X}) et à \textsc{contourner} et son dérivé \textsc{contournement~};
\item \textsc{tourner2} est lié à \textsc{tournure} (\textit{ça va mal tourner}2 {\textasciitilde} \textit{ça prend mauvaise tournure}) ;
\item \textsc{tourner3} est lié à \textsc{tournage} (\textit{tourner}3 \textit{un film} {\textasciitilde} \textit{le tournage d’un film}) ;
\item \textsc{tourner4} (\textit{le lait tourne}) ne possède aucun dérivé et ni \textit{tour}, ni °\textit{tournement}, ni \textit{tournure}, ni \textit{tournage} ne pourra désigner l’action de tourner du lait.
\item \textsc{tourner5} (\textit{L’usine tourne à plein régime}) ne possède aucun dérivé non plus et le collocatif \textit{à plein régime} lui est spécifique.
\end{itemize}

Comme on le voit les dérivations et les collocations ont suffit à elles seules à séparer ces différentes acceptions de \textsc{tourner}. Si on reprend l’exemple de \textsc{peur} et de ses intensifieurs, on peut distinguer deux acceptions :

\ea
    \ea \textit{Il a une peur bleue des araignées}.
    \ex \textit{Tu m’as fais une belle peur. Je ne t’avais pas vu.}
\z
\ex
\ea[\textsuperscript{??}]{\textit{Il a une belle peur des araignées.}}
\ex[\textsuperscript{??}]{\textit{Tu m’as fais une peur bleue. Je ne t’avais pas vu.}}
\z
\z

La première acception est une disposition psychique (la peur des araignées), tandis que le deuxième est un sentiment (causé par la surprise). Comme on le voit, ces deux acceptions n’acceptent pas les mêmes intensifieurs, ce qui permet de les distinguer.

Le découpage (selon l’axe paradigmatique) d’un morphème ou d’un syntaxème en sémantèmes donne ce qu’on appelle les acceptions d’une unité.

\Definition{\textstyleTermes{acception}}
{Les \textstyleTermes{acceptions} d’un signème sont des sous-ensembles regroupant des signes de même sens et même distribution. Autrement dit, un signème est découpé en acceptions selon les différents sémantèmes dont il est une composante.}

Les acceptions d’un syntaxème contiennent aussi des \textstyleTermes{signes désémantisés} qui sont des composantes de sémantèmes complexes. Par exemple, on peut ajouter aux acceptions déjà données des lexèmes \textsc{bureau} et \textsc{tourner} leurs occurrences dans les locutions $⌜$\textsc{bureau} \textsc{de} \textsc{tabac}$⌝$ ou $⌜$\textsc{tourner} \textsc{la} \textsc{page}$⌝$.

\loupe[sec:2.3.14]{L’axe paradigmatique}{%
    Il y a deux façons duales d’envisager l’axe paradigmatique (la notion de \textit{dualité} a été formellement définie dans l’\encadref{sec:2.2.8} \textit{Dualité}).

    La première est de considérer l’ensemble des éléments qui peuvent commuter en un point de la chaîne parlée. Autrement dit, on fixe un certain environnement textuel ou structurel qui définit une certaine position structurale et on regarde le paradigme des éléments qui peuvent commuter dans cette position. C’est le point de vue adopté par Saussure lorsqu’il définit les rapports associatifs (voir l’\encadref{sec:2.2.3} sur \textit{La quatrième proportionnelle}), même si la définition de Saussure est plus large et inclut aussi des éléments qui ne peuvent pas occuper la même position, mais peuvent se remplacer au travers d’une restructuration complète de l’énoncé (comme \textsc{durer} et \textsc{pendant} dans l’exemple du \chapref{sec:1.2} : \textit{Zoé a été malade} \textbf{\textit{pendant} deux semaines} vs \textit{La maladie de Zoé a} \textbf{\textit{duré}} \textit{deux semaines}).

    La deuxième consiste non pas à fixer l’environnement, mais à fixer un élément et à regarder le paradigme des environnements dans lequel il peut se trouver. C’est ce que nous faisons ici lorsque nous découpons un signème en syntaxème ou en sémantème. Découper en syntaxèmes selon l’axe paradigmatique, c’est regarder les différents environnements possibles d’une forme et découper l’ensemble des signes ayant cette forme pour signifiant selon les environnements dans lesquels ils peuvent apparaître. C’est ainsi qu’on distinguera \textit{avance} lorsqu’il est une occurrence du nom et \textit{avance} lorsqu’il est une occurrence du verbe. Et c’est ainsi que nous définirons au \chapfuturef{17} les catégories du nom et du verbe.
}
\section{Dimension sémiotique du découpage}\label{sec:2.3.15}

Nous avons présenté deux axes de découpage, les axes syntagmatiques et paradigmatiques. Il existe un \hi{troisième axe}, que nous appelons l’axe sémiotique, à la suite de Louis \citet[66]{hjelmslev1943omkring}.

\Definition{\textstyleTermes{axe sémiotique}}
{L’\textstyleTermes{axe sémiotique} est la dimension qui relie le sens au texte, le contenu à son expression, le signifié au signifiant.}

Parler d’un axe sémiotique, c’est donner une certaine épaisseur à la relation entre le sens et le texte. Le signifié et le signifiant ne se conçoivent pas comme les deux faces d’une feuille, car le signe ne peut pas être seulement considéré dans les relations que ses seuls signifié et signifiant entretiennent avec les signifiés et signifiants d’autres signes. On doit aussi considérer ses relations syntaxiques avec d’autres signes, lesquelles ne sont pas déductibles des relations au niveau des signifiés et des signifiants.

Comme nous l’avons déjà remarqué, les découpages en unités minimales sont différents selon que l’on considère les signifiants, les syntactiques ou les signifiés, ce qui nous donne respectivement les morphèmes, les syntaxèmes et les sémantèmes. On peut voir l’ensemble des productions d’une langue comme un \hi{espace à trois dimensions~}: il s’agit d’un fil (la dimension syntagmatique) qui s’enroule sur lui-même à chaque fois qu’il rencontre une nouvelle occurrence d’un signème (la dimension paradigmatique) et qui est composé de strates diverses (la dimension sémiotique). Décrire une langue, c’est d’abord \hi{découper l’espace} tridimensionnel des productions de cette langue en unités minimales, puis étudier la façon dont ces unités sont liées entre elles.

Prenons l’exemple du morphème \textit{vis-} dont les principales acceptions~sont le nom \textsc{vis} ‘tige fileté en hélice’, les verbes \textsc{visser} ‘mettre une vis’, \textsc{dévisser1} ‘enlever une vis’ et \textsc{dévisser2} ‘lâcher prise et tomber’ (\textit{L’alpiniste a dévissé}), le nom \textsc{vissage} ‘fait de visser1’, l’adjectif \textsc{vissé} (\textit{Il est vissé sur sa chaise toute la journée}) et la locution $⌜$\textsc{serrer} \textsc{la} \textsc{vis}$⌝$ ‘donner moins de liberté’ (\textit{Ses parents ont décidé de lui serrer la vis depuis qu’il a des mauvaises notes}). Dans \textsc{vis} et \textsc{visser}, le morphème \textit{vis-} est un syntaxème. Il s’agit de deux syntaxèmes différents, puisque leurs distributions sont totalement différentes. Lorsque \textit{vis-} est une composante de la $⌜$locution $⌜$\textsc{serrer} \textsc{la} \textsc{vis}$⌝$, il appartient au syntaxème qu’est le nom \textsc{vis}, dont il est une autre acception ; il s’agit d’un signe \hi{désémantisé}, puisqu’il n’a plus de signifié propre. Le signe \textsc{dévisser1} est très diagrammatique (\textit{dévisser} est à \textit{visser}1 ce que \textit{démonter} est à \textit{monter}), de même que \textsc{vissage} (un \textit{vissage} est à \textit{visser}1 ce qu’un \textit{montage} est à \textit{monter}) : les occurrences du morphème \textit{vis-} y sont donc des signes, mais ce ne sont plus des syntaxèmes, puisqu’ils n’ont plus de combinatoire libre. Le syntaxème \textsc{visser} y a été \hi{désyntactisé}. Le signe \textsc{dévisser2} est peu diagrammatique (il entretient simplement une relation métaphorique avec \textsc{dévisser}1 et le préfixe \textit{dé-} n’y est pas vraiment motivé, même si l’on a des cohyponymes comme \textit{déraper} ou \textit{dégringoler}). Le signe \textsc{vissé} est à peine plus diagrammatique. Les occurrences de \textit{vis-} dans ces deux derniers sémantèmes sont donc des quasi-signes : le syntaxème \textsc{visser} y a été à la fois \hi{désyntactisé} et \hi{désémantisé}.

Il y a donc parmi les sept acceptions considérées du morphème \textit{vis-} deux quasi-signes et cinq signes dont deux sémantèmes (\textsc{vis,} \textsc{visser}), deux lexèmes désyntactisés (lorsqu’il est une composante des sémantèmes \textsc{dévisser1} et \textsc{vissage}) et un lexème désémantisé (lorsqu’il est une composante de $⌜$\textsc{serrer} \textsc{la} \textsc{vis}$⌝$).
La figure \ref{fig:vis}
récapitule cela en montrant les différences entre morphème, syntaxème et sémantème. Chaque ligne correspond à un sémantème, l’acception de \textit{vis-} étant entourée d’une bulle. Les syntaxèmes sont indiqués par des rectangles pleins et le morphème \textit{vis-} par un rectangle en pointillé

\begin{figure}
\begin{tikzpicture}
   \matrix (vis)    [matrix of nodes,
                     nodes in empty cells,
                     column sep=5pt,
                     row sep = 1ex,
                     every node/.style={inner sep=0pt,font=\strut},
                     column 1/.style={anchor=base east},
                     column 2/.style={anchor=base west,nodes={draw,ellipse,inner sep=0pt}},
                     row 6 column 2/.style={nodes={dashed}},
                     row 7 column 2/.style={nodes={dashed}},
                     column 3/.style={anchor=base west}]
     {
       \textit{serr(er) la}~~& \textit{vis} &      &     \\
                    & \textit{vis} &      &     \\
                    & \textit{viss}&      &  \textit{age}\\
                    & \textit{viss}&(er)  &     \\
                \textit{dé}~~& \textit{viss}&\textit{(er)1} &     \\
                \textit{dé}~~& \textit{viss}&\textit{(er)2} &     \\
                    & \textit{viss}&      &  \textit{é}  \\
     };
   \node [draw, dashed, fit = (vis-1-2) (vis-6-3) (vis-7-2), inner sep=3pt] {};
   \node [draw, inner sep=1pt, fit = (vis-1-2) (vis-2-2)] {};
   \node [draw, inner sep=1pt, fit = (vis-3-2) (vis-3-4)] {};
   \node [draw, inner sep=1pt, fit = (vis-4-2) (vis-4-3)] {};
   \node [draw, inner sep=1pt, fit = (vis-5-1) (vis-5-2) (vis-5-3)] {};
   \node [draw, inner sep=1pt, fit = (vis-6-1) (vis-6-2) (vis-6-3)] {};
   \node [draw, inner sep=1pt, fit = (vis-7-2) (vis-7-4)] {};
   \draw [-{Triangle[]}] (vis.south west) -- (vis.south east) node [midway,below] {axe syntagmatique};
   \draw [-{Triangle[]}] (vis.south west) -- (vis.north west) node [very near end,left=2ex,rotate=90] {axe paradigmatique};
   \matrix [right=1cm of vis.east] (legend) 
     {
       \node [draw, ellipse,minimum width=1em] {}; & \node {signes};\\
       \node [draw, dashed, ellipse,minimum width=1em] {}; & \node {quasi-signes};\\
       \node [draw, minimum width=1em] {}; & \node {syntaxèmes};\\
       \node [draw, dashed, minimum width=1em] {}; & \node {morphèmes};\\
     };
\end{tikzpicture}
\caption{\label{fig:vis}Découpage paradigmatique du morphème \textit{vis-}}
\end{figure}

\maths[sec:2.3.16]{Extensions paradigmatique et syntagmatique}{%
    Un signème est un ensemble de signes. On peut mesurer l’extension du signème dans deux dimensions : syntagmatique et paradigmatique. (Voir la section suivante pour la troisième dimension, la dimension sémiotique.)

    L’\textstyleTermesapprofondissement{extension syntagmatique} du signème est la portion de texte que couvre le signème, c’est-à-dire l’empan de ses signifiants. Nous notons {\textbar}{\textbar}~<signème>~{\textbar}{\textbar} l’extension syntagmatique d’un signème <signème>. Par exemple, {\textbar}{\textbar} \textsc{dévisser} {\textbar}{\textbar}, l’extension syntagmatique du verbe \textsc{dévisser}, est égale à \textit{déviss-} /devis/. De ce point de vue, les morphèmes sont les signèmes qui ont les extensions syntagmatiques les plus courtes, puisque ce sont les unités de forme minimales. L’extension syntagmatique d’un morphème est toujours contenue dans celle d’un syntaxème (à l’exception des amalgames), qui est elle-même contenue dans celle d’un sémantème. Nous pouvons résumer cela par la formule suivante :
\ea
        {\textbar}{\textbar}~<morphème>~{\textbar}{\textbar} \textrm{${\subseteq}$} {\textbar}{\textbar}~<syntaxème>~{\textbar}{\textbar} \textrm{${\subseteq}$} {\textbar}{\textbar}~<sémantème>~{\textbar}{\textbar}
\z
    (Nous utilisons le signe d’inclusion \textrm{${\subseteq}$}, introduit dans l’\encadref{sec:2.2.7} sur la \textit{Théorie des ensembles}, car l’extension syntagmatique d’un signème est vue comme une partie du texte.) Par exemple :
\ea
        {\textbar}{\textbar}~\textit{vis-}~{\textbar}{\textbar} \textrm{${\subseteq}$} {\textbar}{\textbar}~\textsc{dévisser}~{\textbar}{\textbar} et {\textbar}{\textbar}~\textsc{vis} {\textbar}{\textbar} \textrm{${\subseteq}$} {\textbar}{\textbar}~\textrm{$⌜$}\textsc{serrer} \textsc{la} \textsc{vis}\textrm{$⌝$} {\textbar}{\textbar}
\z
    L’\textstyleTermesapprofondissement{extension paradigmatique} du signème est l’ensemble de ses occurrences possibles, c’est-à-dire l’ensemble des signes et quasi-signes qu’il contient. (Chaque signifié différent du signème donne un signe différent et chaque présence dans un sémantème différent donne un quasi-signe différent.) Nous notons [~<signème>~] l’extension paradigmatique d’un signème <signème>. Par exemple, [ \textit{vis-} ], l’extension paradigmatique du morphème \textit{vis-}, contient les signifiés ‘vis’, ‘visser1’, ‘visser2’, ‘$⌜$serrer la vis$⌝$’, etc. L’extension paradigmatique d’un sémantème est contenue dans celle d’une combinaison particulière de syntaxèmes, qui est elle-même contenue dans celle d’une combinaison particulière de morphèmes (à l’exception des cas, assez courants, d’allomorphie ; voir \sectref{sec:2.2.20}). Nous pouvons résumer cela par la formule suivante :
\ea{}
        [~<sémantème>~] \textrm{${\subseteq}$} [~<syntaxème>~] \textrm{${\subseteq}$} [~<morphème>~]
\z
    Par exemple :
\ea{}
        [~\textrm{$⌜$}\textsc{serrer} \textsc{la} \textsc{vis}\textrm{$⌝$} ] \textrm{${\subseteq}$} [ \textsc{vis} ] \textrm{${\subseteq}$} [ \textit{vis-}~]
\z
}
\section{Syntaxème et faisceau de signes}\label{sec:2.3.17}

Les ensembles de signes qui forment les syntaxèmes possèdent une propriété remarquable qui justifie d’en faire des unités linguistiques à part entière. Il existe entre les signifiés et les signifiants des signes qui appartiennent à un même syntaxème une forme d’indépendance : chaque association entre un sens particulier et une forme particulière du syntaxème est possible et forme un signe.

\Definition{\textstyleTermes{faisceau de signes}}
{Le syntaxème est ainsi un \textstyleTermes{faisceau de signes}, dont les deux faces, le signifié et le signifiant, sont \hi{indépendantes~}: chaque sens peut se combiner avec n’importe quelle forme et chaque forme avec n’importe quel sens.}

Prenons l’exemple du verbe \textsc{aller} qui est à la fois \hi{polymorphique} (/\textstylePhono{v/}, /\textstylePhono{al/}, /\textstylePhono{i/}, /\textstylePhono{aj/}) et \hi{polysémique} (‘se déplacer’ (\textit{il va quelque part}), ‘se sentir’ (\textit{comment allez-vous} ?), ‘\textsc{futur}’ (\textit{il va partir}), etc., acceptions auxquelles il faut encore ajouter toutes les participations à un phrasème, comme dans $⌜$\textsc{aller} \textsc{au} \textsc{charbon}$⌝$ ou $⌜$\textsc{aller} \textsc{se} \textsc{faire} \textsc{cuire} \textsc{un} \textsc{œuf}$⌝$). On peut représenter l’association libre entre les sens et les formes de \textsc{aller} sous la forme d’un faisceau au centre duquel nous plaçons le nom du syntaxème. Comme on le voit, un signe est l’association d’une des formes avec un des sens, tandis que le signème est la réunion de tous ces signes :

\begin{figure}
\begin{tikzpicture}
 \graph [grow down, branch right sep=2cm,edges=black!20,nodes=black!20] 
   {
     {‘se déplacer’ [black], ‘se sentir’, futur/‘\textsc{futur}’}
     -- {ALLER/\textsc{aller} [black,xshift=4cm]}
     -- {al/\textit{/al/-}, v/\textit{/v/-}, i/\textit{/i/-} [black], aj/\textit{\strut/aj/-}
     };
     {[edges=black]‘se déplacer’ -- ALLER -- i};
     {[grow right] futur -!- dots/{\strut …}};
   };
\end{tikzpicture}
\caption{Signe\label{fig:}}
\end{figure}

\begin{figure}
\begin{tikzpicture}
 \graph [grow down, branch right sep=2cm] 
   {
     {‘se déplacer’, ‘se sentir’, futur/‘\textsc{futur}’}
     -- {ALLER/\textsc{aller} [black,xshift=4cm]}
     -- {al/\textit{/al/-}, v/\textit{/v/-}, i/\textit{/i/-}, aj/\textit{\strut/aj/-}
     };
     {[grow right] futur -!- dots/{\strut …}};
   };
\end{tikzpicture}
\caption{Signème}
\end{figure}

La symétrie entre sens et forme n’est pas complète. Le locuteur fait le choix d’un sens et ce choix peut se porter sur n’importe lequel des signifiés du syntaxème. Mais le «~choix~» du signifiant n’en est pas un : il est \hi{totalement imposé par l’environnement}, c’est-à-dire par les choix adjacents du locuteur. Dans le cas de \textsc{aller}, le «~choix~» du signifiant est conditionné par la flexion et donc par le choix du temps et celui du sujet via l’accord.

Nous montrons pour finir le découpage du syntaxème selon la forme, qui nous donne les (allo)morphes, et le découpage selon le sens, qui nous donne les acceptions.

\begin{figure}
\begin{tikzpicture}
 \graph [grow down, branch right sep=2cm,edges=black!20,nodes=black!20] 
   {
     {‘se déplacer’ [black], ‘se sentir’ [black], futur/‘\textsc{futur}’ [black]}
     -- {ALLER/\textsc{aller} [black,xshift=4cm]}
     -- {al/\textit{/al/-}, v/\textit{/v/-} [black], i/\textit{/i/-} , aj/\textit{\strut/aj/-}
     };
     {[edges=black] {‘se déplacer’, ‘se sentir’, futur} -- ALLER -- v};
     {[grow right] futur -!- dots/{\strut …} [black]};
   };
\end{tikzpicture}
    \caption{(Allo)morphe}             
\end{figure}
           
\begin{figure}
\begin{tikzpicture}
 \graph [grow down, branch right sep=2cm,edges=black!20,nodes=black!20] 
   {
     {‘se déplacer’, ‘se sentir’ [black], futur/‘\textsc{futur}’}
     -- {ALLER/\textsc{aller} [black,xshift=4cm]}
     -- {[nodes=black] al/\textit{/al/-}, v/\textit{/v/-}, i/\textit{/i/-} , aj/\textit{\strut/aj/-}
     };
     {[edges=black] {‘se sentir’} -- ALLER -- {al, v, i , aj}};
     {[grow right] futur -!- dots/{\strut …}};
   };
\end{tikzpicture}
    \caption{Acception}
\end{figure}

Il devient alors possible de voir le syntaxème comme un niveau d’articulation entre sens et forme. On peut ainsi scinder la correspondance sens-forme en deux modules et voir les sémantèmes et les morphèmes comme les éléments respectifs de ces deux modules :

\begin{figure}
\begin{tikzpicture}
 \graph [grow down, branch right sep=2cm,edges=black!20,nodes=black!20] 
   {
     {‘se déplacer’, ‘se sentir’, futur/‘\textsc{futur}’}
     -- {ALLER/\textsc{aller} [black,xshift=4cm]}
     -- {al/\textit{/al/-} [black], v/\textit{/v/-} [black], i/\textit{/i/-} [black], aj/\textit{\strut/aj/-} [black]};
     {[edges=black] ALLER -- {al/\textit{/al/-} [black], v/\textit{/v/-} [black], i/\textit{/i/-} [black], aj/\textit{\strut/aj/-} [black]}};
     {[grow right] futur -!- dots/{\strut …}};
   };
\end{tikzpicture}
\caption{Morphème\label{fig:}}
\end{figure}

\begin{figure}
\begin{tikzpicture}
 \graph [grow down, branch right sep=2cm,edges=black!20,nodes=black!20] 
   {
     {‘se déplacer’, ‘se sentir’ [black], futur/‘\textsc{futur}’}
     -- {ALLER/\textsc{aller} [black,xshift=4cm]}
     -- {al/\textit{/al/-}, v/\textit{/v/-}, i/\textit{/i/-}, aj/\textit{\strut/aj/-}};
     {[edges=black] ALLER -- {‘se sentir’ [black]}};
     {[grow right] futur -!- dots/{\strut …}};
   };
\end{tikzpicture}
\caption{Sémantème}    
\end{figure}

En d’autres termes, on peut voir les morphes et morphèmes comme des (ensembles de) \textstyleTermes{demi-signes de surface} dont le signifié est un élément abstrait (représenté par le nom de lexème \textsc{aller} dans nos figures), qui sert lui même de signifiant aux \textstyleTermes{demi-signes profonds} que sont les sémantèmes. Travailler avec des demi-signes est particulièrement intéressant quand on considère des locutions : on peut en effet considérer que le signifiant de sémantèmes complexes comme $⌜$\textsc{àvoir} \textsc{les} \textsc{pieds} \textsc{sur} \textsc{terre}$⌝$ ou comme le passé composé sont des configurations de syntaxèmes.

\loupe[sec:2.3.18]{Signèmes et signes : unités de la langue et unités de la parole}{%
    Depuis les travaux de Saussure, la quasi-totalité des ouvrages de linguistique considèrent que les unités de la langue sont des signes. Ce n’est pas notre cas. Nous considérons que les unités de la langue, celles qui sont dans notre cerveau, sont plutôt des signèmes, c’est-à-dire des \hi{faisceaux de correspondances entre sens et formes}. Lorsqu’un locuteur produit un énoncé, il choisit de tels faisceaux qu’il assemble. Un signème donné est choisi parce que l’un des sens du signème est un sens que le locuteur veut exprimer. Par exemple, si le locuteur veut parler d’un de ses prochains déplacements, il pourra choisir le faisceau \textsc{aller} et produire un énoncé comme «~\textit{Je vais à Paris la semaine prochaine.}~» Le choix de la forme \textit{v-} du faisceau lui est imposé par le contexte : ce sont les sens ‘moi’ réalisé par \textit{je} et ‘\textsc{présent}’ qui vont sélectionner l’allomorphe \textit{v-} parmi les différents morphes de \textsc{aller}. Ce «~choix~» du signifiant est totalement indépendant du sens par lequel le locuteur a sélectionné le signème \textsc{aller}.

    On peut donc dire que ce ne sont pas les signes qui sont organisés en faisceau, mais \hi{les faisceaux} qui \hi{se réalisent par des signes en parole}. Lorsqu’un signème est utilisé dans un énoncé, c’est forcément qu’il a été choisi pour un de ses sens et qu’il est réalisé sous une de ses formes. Donc, dans un énoncé, le signème apparaît sous la forme d’un signe. Selon nous, c’est bien un signème qui a été sélectionné dans notre cerveau et c’est par l’intermédiaire de ce faisceau qu’un signe a été réalisé. En reprenant la terminologie saussurienne, on peut dire que les \hi{signèmes} sont les \hi{unités de la langue} et les \hi{signes} les \hi{unités de la parole}.

    Il faut aussi noter que les signes sont quasiment des observables : on peut dans un texte repérer leur signifiant et mesurer leur contribution sémantique. Les signèmes, eux, sont des constructions théoriques. Leur existence ne peut être prouvée qu’au sein du modèle (tant qu’on n’a pas les moyens de les observer au sein du cerveau).
}
\section{Autonomie de la syntaxe et de la sémantique}\label{sec:2.3.19}

Nous allons conclure ce chapitre en soulignant un point fondamental concernant les unités sémantiques et la syntaxe : le comportement des syntaxèmes (c’est-à-dire leur combinatoire libre avec d’autres syntaxèmes, leur placement ou leur schéma prosodique) dépend assez peu des unités sémantiques auxquelles ils appartiennent. Par exemple, si l’on prend les occurrences de \textsc{prendre} dans n’importe quel phrasème dont il est la tête ($⌜$\textsc{prendre} \textsc{les} \textsc{jambes} \textsc{à} \textsc{son} \textsc{cou}$⌝$, $⌜$\textsc{prendre} \textsc{son} \textsc{pied}$⌝$, $⌜$\textsc{s’en} \textsc{prendre} \textsc{plein} \textsc{la} \textsc{gueule}$⌝$, etc.), \textsc{prendre} y conserve ses propriété habituelles de verbe, prend une flexion verbale, et peut-être modifié par les mêmes types d’éléments (adverbes, etc.). Ses compléments ne peuvent pas être modifiés (c’est l’effet du figement), mais ils se placent comme des compléments libres et obéissent aux mêmes règles de phonétisation (prosodie comprise).

En général, les phrasèmes ont tendance à avoir une combinatoire plus restreinte et à supporter moins facilement les manipulations que les combinaisons libres, mais ce n’est pas une règle. Par exemple, si on compare le sémantème $⌜$\textsc{briser} \textsc{la} \textsc{glace}$⌝$ ‘dissiper la gêne’ avec la combinaison libre \textsc{briser} \textsc{${\oplus}$ glace}, on voit que les deux combinaisons acceptent d’entrer dans des constructions analogues : négation (\textit{Pierre n’a pas brisé la glace}), clivage du sujet \textit{(C’est Pierre qui a brisé la glace le premier}) ou passif (\textit{La glace a été brisée}). Par contre, le clivage de l’objet (\textit{C’est la glace que Pierre a brisé}) n’est possible qu’avec la combinaison libre, c’est-à-dire quand l’objet est une unité sémantique. Le comportement des phrasèmes dépend donc fortement du comportement des syntaxèmes qui les composent, alors que celui des syntaxèmes est relativement indépendant du fait qu’il forme un sémantème ou sont seulement une composante d’un phrasème.

Morphèmes et sémantèmes structurent le lexique de la langue : ce sont respectivement les plus petites et les plus grandes unités minimales de première articulation. Cependant, le fait que les unités minimales du point de vue du signifiant (les morphèmes) ne correspondent pas aux unités minimales du point de vue du signifié (les sémantèmes) entraine que ni les unes, ni les autres ne sont les unités de la syntaxe. Nous allons donc pouvoir maintenant présenter les unités de la syntaxe et cette mise au point sur morphèmes et sémantèmes permettra, nous l’espérons, de bien comprendre ce qui est et n’est pas une unité de la syntaxe.

\exercices{%\label{sec:2.3.20}
    \exercice{1} Étudier le caractère motivé ou non des signifiants des signes désignant les nombres de \textit{dix} à \textit{vingt}.

    \exercice{2} Alors que la grammaire normative indique que le verbe \textsc{pallier} doit être utilisé transitivement (\textit{pallier un problème}), de nombreux locuteurs produisent aujourd’hui \textit{pallier à un problème}. Qu’est-ce qui peut expliquer que la construction de ce verbe change de cette façon ? Qu'est-ce qui permet qu'une construction puisse changer ?

    \exercice{3} L’énoncé «~\textit{Si j’étais toi, je demanderais à} \textbf{\textit{ma}} \textit{mère.}~» est ambigu : \textit{ma mère} peut référer à la mère du locuteur ou à celle de l’interlocuteur. Cela remet-il en question une analyse purement déictique du pronom \textsc{moi~}? Quelle solution proposer ?

    \exercice{4}
    \begin{enumerate}[label=\alph*.]
    \item Montrer que \textit{grièvement blessé} est une collocation.

    \item L’adverbe \textit{grièvement} joue un rôle d’intensifieur auprès de \textit{blessé}. Chercher d’autres exemples d’intensifieurs exprimés par un collocatif pour des adjectifs, mais aussi pour des verbes et des noms.
    \end{enumerate}

    \exercice{5} Identifier les phrasèmes et les collocations des phrases suivantes :

    \begin{enumerate}
    \item  \textit{L’agent de police dormait à poings fermés.}
    \item  \textit{Il est allé faire la sieste sur la plage.}
    \item  \textit{Je ne peux pas le piffrer.}
    \item  \textit{Il a je ne sais quelle idée en tête.}
    \end{enumerate}

    \exercice{6} Déterminer parmi les occurrences suivantes du lexème \textsc{table} différentes acceptions. Pour chacune d’elle vous indiquerez si elle appartient à un phrasème ou à une collocation.
    \begin{enumerate}
    \item  \textit{Cette table a un pied central.}
    \item  \textit{On passera à table à midi.}
    \item  \textit{Tu as intérêt à te mettre à table rapidement si tu ne veux pas te retrouver en tôle.}
    \item  \textit{Consulter la table des matières en début d’ouvrage.}
    \item  \textit{On vous a réservé la meilleure table.}
    \item  \textit{Le résultat est dans la table de la page précédente.}
    \end{enumerate}

    \exercice{7} Un syntaxème comme \textsc{blanc} peut être utilisé dans la formation de différentes expressions : \textit{voter blanc, blanc comme un linge, blanc comme neige, blanchir le linge, blanchisserie}. Quelles composantes du signe sont affectées dans ces différents emplois ?
}
\lecturesadditionnelles{%\label{sec:2.3.21}
    Le \hi{sémantème}, appelé \textit{monème} par André \citet{martinet1960elements}, est défini dans ses \textit{Éléments de linguistique générale}, dont on consultera les sections 1 à 19 du chapitre 1 et le chapitre 4. Les notions introduites par Martinet sont également présentées dans l'ouvrage de Denis Cosatouec et Françoise Guérin (\citeyear{costaouec2007syntaxe}). Le chapitre sur \textit{Les unités significatives} de l’encyclopédie d'Oswald Ducrot et Jean-Marie Schaeffer (\citeyear{ducrot1995nouveau}) complètera cette lecture.

    La syntaxe des \hi{locutions} est étudiée en détail dans la thèse de Marie-Sophie \citet{pause2017structure}. La distinction entre \hi{transparence} et \hi{diagrammaticité} (appelée \textit{analycité}) est bien dégagée dans un article de Marie Helena Svensson de \citeyear{svensson2008very}.

    Le \hi{découpage paradigmatique} en unités lexicales ainsi que les \hi{collocations} sont très bien présentés dans l'ouvrage de lexicologie d'Alain \citet{polguere2003lexicologie}. Voir également l'\textit{Introduction à la lexicologie explicative et combinatoire} de \citeyear{melcuk1995introduction}, co-écrite avec Igor Mel’čuk, pour une introduction plus technique aux \hi{fonctions lexicales}, ainsi que le \textit{Lexique actif du français} de \citeyear{melcuk2007lexique} pour des exemples d’entrées lexicales avec leurs collocatifs. Les \hi{collocations morphologiques} sont présentées dans un article de David Beck de \citeyear{beck2019phraseology}.

    La notion de \hi{faisceau de signes} est introduite dans le premier tome du \textit{Cours de Morphologie Générale} d'Igor \citet{melcuk1988dependency} (voir \chapref{sec:2.1}) lors de la description des grammèmes (p. 278), mais n'est pas exploitée pour les syntaxèmes en général. Les grammèmes chez Mel’čuk n'ont d'ailleurs pas un statut de signe linguistique.

    \FurtherReading{2-3}
}
\corrections{%\label{sec:2.3.22}
    \corrigé{1} Les nombres \textit{dix-sept, dix-huit} et \textit{dix-neuf} sont construits selon la syntaxe régulière des nombres (comme \textit{cent vingt-huit} par exemple). On peut donc considérer qu’ils sont compositionnels. Les nombres \textit{onze, douze, treize, quatorze, quinze, seize} ne sont pas décomposables en synchronie, même s’ils possèdent tous une terminaison en \textit{{}-ze} et un radical en partie transparent. Ils sont donc assez motivés. Les signifants des nombres \textit{dix} et \textit{vingt} sont eux totalement arbitraires.

    \corrigé{2} Il semble que le verbe \textsc{pallier} ait changé de construction en raison de sa proximité sémantique avec \textsc{remédier} ou $⌜$\textsc{s’attaquer}$⌝$ qui se construisent avec X V \textit{à} Y. Le fait qu’un verbe puisse changer de construction montre bien qu’il y a une certaine indépendance de la construction par rapport au lexème lui-même. Voir l’\encadref{sec:2.3.4} \textit{Constructions verbales et accords : signes vides} ?.

    \corrigé{3} Le pronom \textsc{moi} ne réfère pas forcément au locuteur à proprement parler, mais à celui à qui on attribue les paroles. C’est le cas avec le discours rapporté : \textit{Zoé m’a dit : «~Je viendrai~».} C’est aussi le cas ici, où il y a un transfert du ‘toi’ au ‘moi’.

    \corrigé{4} L’adverbe \textsc{grièvement} ne peut modifier que l’adjectif \textsc{blessé}. \textsc{blessé} garde son sens usuel et \textsc{grièvement} fonctionne comme un intensifieur. L’absence de commutation possible sur \textsc{blessé} montre que \textsc{grièvement} est un collocatif. Le choix de \textsc{blessé} est libre et le choix de \textsc{grièvement} est dépendant de ce premier choix. Les intensifieurs sont les plus productifs des collocatifs. En voici quelques uns : \textit{gravement malade, con comme un balai, armé jusqu’au dent, aimer à la folie, applaudir des deux mains, craindre comme la peste, grand ami, gros fumeur, augmentation substantielle, catastrophe épouvantable, désaccord profond, bataille sans merci, forme olympique,} etc.

    \corrigé{5}
    \begin{enumerate}
    \item  Une expression comme \textit{agent de police} se situe à la frontière entre collocation et phrasème. On peut la considérer comme une collocation non standard de base \textsc{police}, le collocatif \textsc{agent} désignant l'un des types de fonctionnaire travaillant à la police. Néanmoins, \textsc{police} parait peu modifiable et il est loin d'être certain que \textsc{agent} soit choisi à partir du choix initial de \textsc{police}. (Le problème est encore compliqué ici par le fait qu’il y a une autre acception de \textsc{agent} qui à elle seule veut dire ‘agent de police’.) \textrm{$⌜$}\textsc{à} \textsc{poings} \textsc{fermés}\textrm{$⌝$} est également un phrasème et ce phrasème est un collocatif de \textsc{dormir} marquant l’intensification.
    \item  \textit{faire la sieste} est un exemple typique de construction à verbe support, où la base \textsc{sieste} est un nom prédicatif et le collocatif \textsc{faire} n’a pas de contribution propre. Il y a une autre collocation moins évidente qui est \textit{sur la plage}. En effet, la préposition locative \textsc{sur} n’est pas totalement prédictible : on peut dire \textit{à/sur la plage}, mais, par exemple, on doit dire \textit{à/dans la campagne} et seulement \textit{dans la forêt.} On considère donc que les prépositions locatives sont des collocatifs dont le choix est restreint par le nom qui suit.
    \item  Le verbe \textsc{piffrer} ne s’utilise que précédé de \textsc{pouvoir} (*\textit{je ne le piffre pas}) et~dans un contexte négatif (*\textit{je peux le piffrer}). Par contre, différentes réalisation de la négation sont possibles : \textit{je n’ai jamais pu le piffrer} ; \textit{personne ne peux le piffrer}. On a donc un phrasème que nous notons \textrm{$⌜$}\textsc{(ne} \textsc{pas)} \textsc{pouvoir} \textsc{piffrer}\textrm{$⌝$} (la parenthèse indique qu’une autre négation peut remplacer \textsc{ne} \textsc{pas}).
    \item  Comme dans les cas précédent, on voit si des commutations sont possibles. On remarque d’abord que le déterminant de \textsc{idée} peut être changé : \textit{avoir une idée en tête}. De plus, \textsc{idée} peut commuter avec d’autres noms : \textit{avoir un problème/une musique en tête.} Ou même un pronom : \textit{avoir quelque chose en tête}. Il nous reste donc l’expression \textit{avoir N en tête}. Or on peut encore commuter le verbe : \textit{on m’a mis cette idée en tête}, \textit{avec ça en tête}. Au final, on a un phrasème \textrm{$⌜$}\textsc{en} \textsc{tête}\textrm{$⌝$} dont \textsc{avoir} est un collocatif (un verbe support). Quant au déterminant \textit{je ne sais quel}, il appartient à un paradigme de pronom : \textit{je ne sais qui/quoi/où/comment}. L’élément \textit{je ne sais} possède lui-même une certaine variabilité : \textit{on ne sait quel, Dieu sait quel}. Au final, on considérera quand même que \textrm{$⌜$}\textsc{je} \textsc{ne} \textsc{sais} \textsc{quel}\textrm{$⌝$} est un phrasème, mais cet exemple montre qu’il y a plusieurs degrés de décompositionalité et que la frontière entre phrasème et syntagme compositionnel n’est pas parfaitement nette.
    \end{enumerate}

    \corrigé{6} Le lexème \textsc{table} est un sémantème dans les exemples 1, 5 et 6. Il s’agit de trois acceptions différentes, puisque \textsc{table} désigne un meuble en 1, un lieu gastronomique en 5 et un type de diagramme en 6. En 2, on a une locution \textrm{$⌜$}\textsc{à} \textsc{table}\textrm{$⌝$}, comme le montre l’absence de commutation possible sur \textsc{table}. Par contre, \textsc{passer}, qui peut commuter avec \textrm{$⌜$}\textsc{se} \textsc{mettre}\textrm{$⌝$} ou \textsc{être}, est un collocatif de \textrm{$⌜$}\textsc{à} \textsc{table}\textrm{$⌝$}. En 3, la commutation de \textrm{$⌜$}\textsc{se} \textsc{mettre}\textrm{$⌝$} avec \textsc{être} n’est plus possible et \textrm{$⌜$}\textsc{se} \textsc{mettre} \textsc{à} \textsc{table}\textrm{$⌝$} forme une locution signifiant ‘passer des aveux’. En 4, \textrm{$⌜$}\textsc{table} \textsc{des} \textsc{matières}\textrm{$⌝$} est également un phrasème. En 5, \textsc{table} est la base de deux collocations : \textsc{réserver} est un collocatif exprimant une étape préliminaire à l'utilisation de la table, tandis que \textsc{bon} (ici sous la forme superlative \textit{meilleur}) est un collocatif intensifiant la qualité de la table. En 6, \textsc{table} est associé au collocatif \textsc{dans}, qui est une préposition exprimant la localisation. On notera que les autres acceptions de \textsc{table}, utilisées en 1 et 5, ne pourraient pas s'utiliser avec cette préposition.

    \corrigé{7} Dans \textit{voter blanc}, \textsc{blanc} est un collocatif de \textsc{voter}. Dans ce contexte, \textsc{blanc} signifie ‘en mettant un bulletin blanc’. Dans \textit{blanc comme un linge}, c’est \textrm{$⌜$}\textsc{comme} \textsc{un} \textsc{linge}\textrm{$⌝$} qui est un collocatif de \textsc{blanc}, mais \textsc{blanc} a une acception particulière ‘qui manifeste un sentiment de peur ou de colère’. Par contre, \textrm{$⌜$}\textsc{blanc} \textsc{comme} \textsc{neige}\textrm{$⌝$} est un phrasème, car \textsc{blanc} seul ne peut avoir le sens ‘sans aucune preuve à charge’ ; \textsc{blanc} y est donc désémantisé, puisque le sens est porté par la locution complète. Dans \textsc{blanchir,} le radical /blāʃ/ est le même morphème que le syntaxème \textsc{blanc}, mais cette acception n’a plus du tout le même syntactique (il s’agit d’un conversion de nom en verbe) et son signifié est également modifié et est devenu ‘laver à haute température pour rendre blanc et pur’. \textsc{blanchisserie} est construit à partir de \textsc{blanchir}, mais le signe a subi une nouvelle désyntactisation et désémantisation, puisque même si \textsc{blanchisserie} est assez diagrammatique, il n’est pas compositionnel pour autant.
}
